\begin{figure*}[t]
\centering
\begin{subfigure}{0.48\textwidth}
\begin{grammar}
  Process names & $p,q$ &$\bnfas$& $\cdots$
  \\
  Name sets
  & $\Names$ 
    & $\bnfas$ & $\emptyNames ~|~ \Names, d ~|~ \Names, p$
  \\
  Process pools
  & $\Procs$ 
  & $\bnfas$ & $\emptyProcs ~|~ \Procs, \ProcNm{p} \proc{e}$
  \\
  Configurations
  & $C$
     & $\bnfas$ & $\Config{\Names}{}{\Procs} $
\end{grammar}
\end{subfigure}%
\begin{subfigure}{0.48\textwidth}
  \begin{grammar}
 Evaluation
  & $E$
 & $\bnfas$ &
$\bullet \bnfalt \ePair{E}{e}{\ell} \bnfalt \ePair{v}{E}{\ell} \bnfalt \eInjj{\ell}{i}{E}$ 
\\ contexts &&& $\bnfaltbrk \eSplitt{\ell}{E}{x_1}{x_2}{e} \bnfalt
\eCase{\ell}{E}{x_1}{e_1}{x_2}{e_2}$
\\ &&& $\bnfaltbrk \eApp{E}{e}{\ell} \bnfalt \eApp{v}{E}{\ell}  \bnfalt \eLet{x}{E}{e}
\bnfalt \eBang{E} \bnfalt \eUnbang{E}$
\\ &&& $\bnfaltbrk \eWr{E}{e} \bnfalt \eWr{v}{E} \bnfalt \eLetRd{E}{x}{e}$
\\ &&& $\bnfaltbrk \eChoicee{E}{x_1}{e_3}{e_2}{x_2}{e_4} \bnfalt
\eChoicee{c}{x_1}{e_3}{E}{x_2}{e_4}$
%\\
% Read contexts
%  & $R$
%%%
%%% This initial definition of R is too limiting: It's not general enough to support the progress theorem:
%%     & $\bnfas$ & $\bullet \bnfalt \eRd{\eChan{c}} \oplus R \bnfalt R \oplus \eRd{\eChan{c}}$
%     & $\bnfas$ & $\bullet \bnfalt e \boxplus R \bnfalt R \boxplus e$
\end{grammar}
\end{subfigure}
\caption{ILC dynamic syntax.}
\label{fig:configs}
\end{figure*}

\begin{figure*}
  \centering
  \begin{subfigure}{0.48\textwidth}
    \judgbox{C_1 \equiv C_2}{~~Configurations~$C_1$ and $C_2$ are equivalent.}
    \begin{mathpar}
      \Infer{permProcs}
            {  \Procs_1 \equiv_\textsf{perm} \Procs_2 }
            { \Config{\Names}{\Store}{\Procs_1} \equiv \Config{\Names}{\Store}{\Procs_2} }
    \end{mathpar}
  \end{subfigure}%
  \begin{subfigure}{0.48\textwidth}
    \judgbox{c_1 \leadsto c_2}{~~Write endpoint~$c_1$ connects to read endpoint $c_2$.}
    \begin{mathpar}
      \Infer{bind}
            { }
            {\mathsf{Write}(d) \leadsto \mathsf{Read}(d)}
    \end{mathpar}    
  \end{subfigure}  
%\caption{Structural congruence.}
%\label{fig:structural-congruence}
%\end{figure*}
%
%\begin{figure*}
\judgbox{C_1 ---> C_2}{~~Configuration~$C_1$ reduces to $C_2$.}
\begin{mathpar}
\Infer{local}{e_1 ---> e_2 }
{ \Config{\Names}{\Store_1}{\Procs, \ProcNm{p} \proc{E[e_1]}} --->
  \Config{\Names}{\Store_2}{\Procs, \ProcNm{p} \proc{E[e_2]}} }
\and
\Infer{fork}{ q \notin \Names }
{ \Config{\Names}{\Store}{\Procs, \ProcNm{p} \proc{E[ \eFork{e_1}{e_2} }] } --->
  \Config{\Names,q}{\Store}{\Procs, \ProcNm{q} \proc{e_1}, \ProcNm{p} \proc{E[ e_2 ]}}}
%\and
%\Infer{par}{
%\Config{\Names}{\Store}{\Procs_1} -> \Config{\Names'}{\Store}{\Procs_1'}}
%{\Config{\Names}{\Store}{\Procs_1, \Procs_2} ->
%  \Config{\Names'}{\Store}{\Procs_1',\Procs_2}}
%
%\Store}{\Procs_1'}}
%\Config{\Names_1',\Names_2}{\Store}{\Procs_1',\Procs_2}}
\and
\Infer{congr}{
C_1 \equiv C_1' 
\\
C_1' ---> C_2'
\\
C_2' \equiv C_2
}
{ C_1 ---> C_2 }
\and
\Infer{nu}{ d \notin \Names}
{ \Config{\Names}{\Store}{\Procs, \ProcNm{p} \proc{E[ \eNu{(x_1, x_2)}{e} ]}} --->
  \Config{\Names, d}{\Store}{\Procs, \ProcNm{p} \proc{E[
        [\mathsf{Read}(d)/x_1][\mathsf{Write}(d)/x_2]e ]}}}
%\Infer{nu}{ c_1, c_2 \notin \Names \\ c_2 \leadsto c_1}
%{ \Config{\Names}{\Store}{\Procs, \ProcNm{p} \proc{E[ \eNu{(x_1, x_2)}{e} ]}} --->
%  \Config{\Names, c_1, c_2}{\Store}{\Procs, \ProcNm{p} \proc{E[ [\eChan{c_1}/x_1][\eChan{c_2}/x_2]e ]}}}
\and
\Infer{rw}
{ c_2 \leadsto c_1 }
{ \Config{\Names}{\Store}{\Procs, \ProcNm{p} E_1[ \eLetRd{\eChan{c_1}}{x}{e} ], \ProcNm{q} E_2[ \eWr{v}{\eChan{c_2}}]} --->
  \Config{\Names}{\Store}{\Procs, \ProcNm{p} E_1[ [\ePair{!v}{\eChan{c_1}}{1}/x]e], \ProcNm{q}
    E_2[ \eUnit ]} }
\and
\Infer{cw}
{ c \leadsto c_i \\ i \in \{1, 2\} }
{ \Config{\Names}{\Store}{\Procs, \ProcNm{p} E_1[\eChoicee{c_1}{x_1}{e_1}{c_2}{x_2}{e_2}], \ProcNm{q} E_2[ \eWr{v}{\eChan{c}}]} --->
  \Config{\Names}{\Store}{\Procs, \ProcNm{p} E_1[ [\ePair{!v}{c_1,
          c_2}{1}/x_i]e_{i}], \ProcNm{q} E_2[ \eUnit ]} }
\end{mathpar}

\judgbox{e_1 ---> e_2}{~~Expression~$e_1$ reduces to~$e_2$.}
\begin{mathpar}
\Infer{let}
{}
{ \eLet{x}{v}{e} ---> [v/x]e }
\and
\Infer{app}
{}
{ \eApp{(\eLamm{\ell}{x}{e})}{v}{\ell} ---> [v/x]e }
\and
\Infer{split}
{ }
{ \eSplitt{\ell}{\ePair{v_1}{v_2}{\ell}}{x_1}{x_2}{e} ---> [v_1/x_1][v_2/x_2]e }
\and
\Infer{case}
{ }
{ \eCase{\ell}{\eInjj{\ell}{i}{v}}{x_1}{e_1}{x_2}{e_2} ---> [v/x_i]e_i }
\and
\Infer{fix}
{ }
{ \eFixx{\ell}{x}{e} ---> [\eFixx{\ell}{x}{e} / x] e }
\and
\Infer{gnab}
{ }
{ \eUnbang{(\eBang{v})} ---> v }
\end{mathpar}
\caption{ILC reduction rules.}
\label{fig:semantics}
\end{figure*}
