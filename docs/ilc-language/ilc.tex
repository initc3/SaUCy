\def\OPTIONConf{0}%
\def\OPTIONArxiv{0}%
%
\documentclass[10pt]{article}
\usepackage[english]{babel}
\usepackage{amsfonts}
\usepackage{amsmath}
\usepackage{amssymb}
\usepackage{color}
\usepackage{comment}
\usepackage{enumitem}
\usepackage[margin=1in]{geometry}
\usepackage{mathpartir}
\usepackage{joshuadunfield}
\usepackage{natbib}
\usepackage{semantic}
\usepackage{soul}
\usepackage{tgpagella}
\usepackage{titling}
\usepackage{xspace}

\setlength{\droptitle}{-.6in}
\pretitle{\begin{center}\Large\bfseries}
\posttitle{\par\end{center}\vspace{-.4cm}}
\preauthor{\begin{center}}
\postauthor{\par\end{center}\vspace{-.5cm}}
\predate{\begin{center}\itshape}
\postdate{\end{center}\vspace{-.6cm}}

\newcommand{\runonboldsf}{\sffamily\bfseries\selectfont}
\newcommand{\boldsf}[1]{\text{\runonboldsf #1}}

% ILC Values
\newcommand{\vUnit}{\texttt{()}}

% ILC Types
\newcommand{\tyUnit}{\tyname{unit}}
\newcommand{\tyNat}{\tyname{nat}}

% ILC Computation types
\newcommand{\xF}{\boldsf{F}}
\newcommand{\xU}{\boldsf{U}}
\newcommand{\xM}{\boldsf{M}}
\newcommand{\tyFp}[1]{\xF(#1)}
\newcommand{\tyUp}[1]{\xU(#1)}
\newcommand{\tyMp}[1]{\xM(#1)}
\newcommand{\tyF}[1]{\xF\,#1}
\newcommand{\tyU}[1]{\xU\,#1}
\newcommand{\tyM}[1]{\xM\,#1}

% ILC Read and Write channel types
\newcommand{\xRd}{\boldsf{Rd}}
\newcommand{\xWr}{\boldsf{Wr}}
\newcommand{\tyRdp}[1]{\xRd(#1)}
\newcommand{\tyWrp}[1]{\xWr(#1)}
\newcommand{\tyRd}[1]{\xRd\,#1}
\newcommand{\tyWr}[1]{\xWr\,#1}

% ILC Modes
\newcommand{\Rm}{\textsf{R}}
\newcommand{\Wm}{\textsf{W}}
\newcommand{\Vm}{\textsf{V}}


\newcommand{\xFork}{\mathrel{|\rhd}}

%% Math ligatures (thanks to the semantic package) that make it
%% easier to typeset math using readable LaTeX text.
%\mathlig{|-->}{\longmapsto}
\mathlig{::=}{\bnfas}
\mathlig{|}{\;|\;}
% \mathlig{[[}{\mbsf{[}}
% \mathlig{]]}{\mbsf{]}}
\mathlig{[[}{\texttt{\upshape[}}
\mathlig{]]}{\texttt{\upshape]}}
\mathlig{**}{\times}
\mathlig{|>}{\rhd}
\mathlig{->}{\arr}
\mathlig{-->}{\rightarrow}
\mathlig{--->}{\longrightarrow}
\mathlig{=>}{\Rightarrow}
\mathlig{*!}{\boldsf{!}}
\mathlig{||}{\mathrel{|\!|}}
\mathlig{;;}{\mathrel{;}}
\mathlig{*&&}{\mathrel{\xFork}}
\mathlig{*||}{\mathrel{\oplus}}

\mathlig{@>}{\arr_{\namesort}}

\newcommand{\e}{\epsilon}
\newcommand{\hist}{\mathcal{H}}

\newcommand{\ambns}{M}

\newcommand{\disjoint}{\mathrel{\bot}}

\newcommand{\tbrack}[1]{\texttt{\upshape[}{#1}\texttt{\upshape]}}

\newcommand{\rootname}{\textvtt{root}}

%\newcommand{\tv}{tv}
\newcommand{\Mv}{V}
\newcommand{\ntevalsym}{\Downarrow_{\mathsf{M}}}
\newcommand{\nteval}{\mathrel{\ntevalsym}}

\newcommand{\xrefv}{\keyword{ref}}
\newcommand{\xthunk}{\keyword{thunk}}
\newcommand{\xunthunk}{\keyword{unthunk}}
\newcommand{\xname}{\keyword{name}}
%\newcommand{\xName}{\tyname{Name}}
\newcommand{\xName}{\tyname{Nm}}

\newcommand{\xRef}{\tyname{Ref}}
%\newcommand{\xThunk}{\tyname{Thunk}}
\newcommand{\xThunk}{\tyname{Thk}}
\newcommand{\xUnthunk}{\tyname{Unthunk}}

\newcommand{\refv}[1]{\xrefv\;{#1}}
\newcommand{\thunk}[1]{\xthunk\;{#1}}
\newcommand{\unthunk}[1]{\xunthunk\;{#1}}
\newcommand{\name}[1]{\xname\;{#1}}
\newcommand{\Name}[1]{\xName\tbrack{#1}}


\newcommand{\Thk}[1]{\xThunk\tbrack{#1}\,}
\newcommand{\Unthk}[1]{\xUnthunk\tbrack{#1}\,}
\newcommand{\Thunk}[2]{\keyword{thunk}\Lparen{#1},{#2}\Rparen}
\newcommand{\Unthunk}[1]{\keyword{unthunk}\Lparen{#1}\Rparen}
\newcommand{\Ref}[1]{\keyword{ref}\Lparen{#1}\Rparen}
\newcommand{\Deref}[1]{\keyword{deref}\Lparen{#1}\Rparen}
\newcommand{\Assign}[2]{\keyword{assign}\Lparen{#1},{#2}\Rparen}
\newcommand{\Seq}[2]{\keyword{seq}\Lparen{#1},{#2}\Rparen}

\newcommand{\List}[3]{\tyname{List}\tbrack{#1;#2}~{#3}}
\newcommand{\Art}[2]{\tyname{Art}\tbrack{#1}~{#2}}
\newcommand{\Int}[0]{\tyname{Int}}

\let\Force\undefined
\newcommand{\eForce}[1]{\keyword{force}\Lparen{#1}\Rparen}

\newcommand{\App}[2]{#1\,#2}
\newcommand{\eApp}[2]{#1\,#2}

\newcommand{\Lparen}{\texttt{(}}
\newcommand{\Rparen}{\texttt{)}}

\newcommand{\Ret}[1]{\keyword{ret}\Lparen{#1}\Rparen}
\newcommand{\Let}[3]{\keyword{let}\Lparen{#1},{#2}.{#3}\Rparen}
\newcommand{\eLet}[3]{\keyword{let}\Lparen{#1},{#2}.{#3}\Rparen}
\newcommand{\LetBang}[3]{\keyword{let!}\Lparen{#1},{#2}.{#3}\Rparen}

\newcommand{\fix}[2]{\keyword{fix}\Lparen{#1}.{#2}\Rparen}

% Channel read and Channel write expressions
\newcommand{\eRd}[1]{\keyword{rd}\Lparen{#1}\Rparen}
\newcommand{\eWr}[2]{\keyword{wr}\Lparen{#1},{#2}\Rparen}
\newcommand{\eNu}[2]{\nu#1.\,#2}

\newcommand{\xstoretype}{\textsf{has-store-type}}
\newcommand{\storetype}{\;\xstoretype\;}

\newcommand{\xterminal}{\textsf{terminal}}
\newcommand{\terminal}{\;\xterminal}

\newcommand{\Config}[2]{\left<#1;#2\right>}

\newcommand{\Chans}{\Sigma}
\newcommand{\vChan}[1]{\texttt{chan}\Lparen#1\Rparen}
\newcommand{\emptyChans}{\varepsilon}

\newcommand{\emptyctxt}{\cdot}

\newcommand{\Procs}{\pi}
\newcommand{\proc}{e}
\newcommand{\emptyProcs}{\varepsilon}


\begin{document}

\title{Interactive Lambda Calculus (ILC) Language Definition}
\author{Kevin Liao, Andrew Miller}
\date{University of Illinois, Urbana-Champaign}

\maketitle
\thispagestyle{empty}

\begin{figure}[htbp]
{
  \centering

\begin{grammar}
  Modes & $m,n,p$ &$\bnfas$& $\Wm \bnfalt \Rm \bnfalt \Vm$ & (Write, Read and Value) 
\end{grammar}

\judgbox{m || n => p}{~~The parallel composition of modes $m$ and $n$ is mode~$p$.}
\begin{mathpar}
\Infer{sym}{m || n => p}{n || m => p}
\and \Infer{wv}{ }{\Wm || \Vm => \Wm}
\and \Infer{wr}{ }{\Wm || \Rm => \Wm}
\and \Infer{rr}{ }{\Rm || \Rm => \Rm}
\end{mathpar}
\\[2mm]
\judgbox{m ;; n => p}{~~The sequential composition of modes $m$ and $n$ is mode~$p$.}
\begin{mathpar}
\and \Infer{v$\ast$}{ }{\Vm ;; n => n}
\and \Infer{wv}{ }{\Wm ;; \Vm => \Wm}
\and \Infer{r$\ast$}{ }{\Rm ;; n => \Rm}
\and \Infer{wr}{ }{\Wm ;; \Rm => \Wm}
\end{mathpar}
}
Note that in particular, the following mode compositions are \emph{not derivable}:
\begin{itemize}
\item $\Wm || \Wm => p$ is \emph{not} derivable for any mode~$p$
\item $\Wm ;; \Wm => p$ is \emph{not} derivable for any mode~$p$
\end{itemize}
\caption{Syntax of modes; sequential and parallel mode composition.}
\label{fig:expr}
\end{figure}

\begin{figure}[htbp]
\centering
\judgbox{\Delta ; \Gamma |- e : \tau |> m}{~~Under $\Delta$ and $\Gamma$, expression~$e$ has type $\tau$ and mode $m$.}
\begin{mathpar}
%
\Infer{taut}
{x:\sigma |> m \in \Delta; \Gamma}
{\Delta; \Gamma |- x:\sigma |> m}
%
\and
%
\Infer{inst}
{\sigma_1 > \sigma\\\\
\Delta; \Gamma |- e:\sigma_1 |> m}
{\Delta; \Gamma |- e:\sigma |> m}
%
\and
%
\Infer{gen}
{ \alpha \not \in {\sf fv}(\Delta; \Gamma)\\\\
\Delta; \Gamma |- e:\sigma |> m}
{\Delta; \Gamma |- e:\forall\alpha\sigma}
%
\and
%
\Infer{lam}
{\Delta ; \Gamma, x:\tau |-         e : \tau_1      |> m}
{\Delta ; \Gamma      |- \lam{x} e : \tau -> \tau_1 |> m}
%
\and
%
\Infer{app}
{\Delta_1 ; \Gamma |- e_1 : \tau_1 -> \tau_2 |> m\\\\
 \Delta_2 ; \Gamma |- e_2 : \tau_1        |> \Vm}
{\Delta_1, \Delta_2 ; \Gamma |- e_1 e_2 : \tau_2 |> m}
%
\and
%
\Infer{let}
{ m_1 ;; m_2 => m_3\\\\
\Delta_1 ; \Gamma |- e_1 : \sigma |> m_1 \\\\
\Delta_2 ; \Gamma, x:\sigma |- e_2 : \tau |> m_2
}
{\Delta_1, \Delta_2 ; \Gamma |- \Let{e_1}{x}{e_2} : \tau |> m_3}
%
\and
%
\Infer{nu}
{\Delta; \Gamma, x: {\sf Ch}\ \tau_1 |- e : \tau |> m}
{\Delta; \Gamma |- \eNu{x}{e} : \tau |> m}
%
\\
%
\Infer{rd}
{\Delta; \Gamma |- e : {\sf Ch}\ \tau |> \Vm}
{\Delta    |- \eRd{e} : \tau |> \Rm}
%
\and
%
\Infer{wr}
{\Delta_1; \Gamma   |- e_1 : \tau |> m \\
 \Delta_2; \Gamma   |- e_2 : {\sf Ch}\ \tau |> \Vm}
{\Delta_1, \Delta_2 |- \eWr{e_1}{e_2} : \Unit |> \Wm}
%
\\
%
\Infer{fork}
{
 m_1 || m_2 => m_3
 \\\\
 \Delta_1; \Gamma |- e_1 : \tau_1 |> m_1
 \\\\
 \Delta_2; \Gamma |- e_2 : \tau |> m_2
}
{\Delta_1, \Delta_2 |- e_1 \xFork e_2 : \tau |> m_3}
%
\end{mathpar}
\end{figure}

\bibliographystyle{unsrt}
\bibliography{references}

\end{document}
