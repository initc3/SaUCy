\begin{abstract}
  The universal composability (UC) framework is the established standard for
  analyzing cryptographic protocols in a modular way, such that security is
  preserved under concurrent composition with arbitrary other protocols.
  However, although UC is widely used for on-paper proofs, prior attempts at
  systemizing it have fallen short, either by using a symbolic model (thereby
  ruling out computational reduction proofs), or by limiting its expressiveness.

  In this paper, we lay the groundwork for building a concrete, executable
  implementation of the UC framework. Our main contribution is a process
  calculus, dubbed the Interactive Lambda Calculus (ILC). ILC faithfully
  captures the computational model underlying UC---interactive Turing machines
  (ITMs)---by adapting ITMs to a subset of the $\pi$-calculus through an affine
  typing discipline. In other words, \emph{well-typed ILC programs are
    expressible as ITMs.} In turn, ILC's strong confluence property enables
  reasoning about cryptographic security reductions.  We use ILC to develop a
  simplified implementation of UC called SaUCy.
\end{abstract}
