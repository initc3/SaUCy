\begin{abstract}
  The universal composability (UC) framework is an established gold standard for
  analyzing cryptographic protocols in a modular way from simpler building
  blocks. In particular, security is preserved under general \emph{concurrent}
  composition, so a UC-secure protocol remains so when dropped into arbitary
  contexts. But although UC is widely used for on-paper proofs, its modularity
  benefits have not been enjoyed in practical implementations.

  In this paper, we lay the groundwork for building a concrete implementation of
  the UC framework called SaUCy in the form of a process calculus called the
  Interactive Lambda Calculus (ILC). In particular, ILC captures the
  computational model underlying UC---interactive Turing machines (ITMs)---by
  adapting ITMs to a subset of the $pi$-calculus through its type system. In
  other words, \emph{well-typed ILC programs are expressible as ITMs.} Using
  ILC, we concretize the UC framework in SaUCy. \todo{Edit later.}
\end{abstract}
