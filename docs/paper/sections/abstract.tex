\begin{abstract}
  The universal composability (UC) framework is an established gold standard for
  analyzing cryptographic protocols in a modular way from simpler building
  blocks. In the framework, security is preserved under \emph{concurrent}
  composition with arbitrary other protocols, so a UC-secure protocol remains so
  when dropped into any context.

  But although UC is widely used for on-paper proofs, its modularity benefits
  have not been enjoyed in practical implementations. This is due in part to the
  fact that proof artifacts (which are essentially programs) are traditionally
  written in a combination of prose and pseudocode, which makes them error-prone
  and difficult to verify. Moreover, previous work on systematizing
  cryptographic analysis falls short of either permitting computational security
  proofs or being expressive enough to represent distributed protocols.

  In this paper, we lay the groundwork for building a concrete, executable
  implementation of the UC framework (dubbed SaUCy) in the form of a process
  calculus called the Interactive Lambda Calculus (ILC). In particular, ILC
  faithfully captures the computational model underlying UC---interactive Turing
  machines (ITMs)---by adapting ITMs to a subset of the $\pi$-calculus through its
  type system. In other words, \emph{well-typed ILC programs are expressible as
    ITMs.} And being a faithful embedding of ITMs is crucial---it is necessary for
  computational security proofs to go through. \todo{Discuss SaUCy.}
\end{abstract}
