\section{Interactive Lambda Calculus}
\label{sec:ilc}

\subsection{Program Syntax}
\label{subsec:syntax}

Figure~\ref{fig:ilc-syntax} gives the syntax of expressions. Expressions $e$
include variables $x$, primitive values $v$, channels $c$, the unit value $()$,
pairs (with elimination form \textsf{split}), sums (with elimination form
\textsf{case}), reference cells (with elimination form \textsf{get} and mutable
update \textsf{set}), thunks (with elimination form \textsf{force}), fixed
points, let binding, lambda abstraction, and function application. For
communication, expressions include restriction ($\eNu{(x_1, x_2)}{e}$), which
binds a a read channel $x_1$ and write channel $x_2$ in $e$; send
($\eWr{e_1}{e_2}$), which writes the result of evaluating $e_1$ on the channel
result of evaluating $e_2$; receive ($\eRd{e}$), which reads from the channel
result of evaluating $e$ and returns a pair consisting of the read value and the
read channel itself (see Section~\ref{subsec:types} for details); fork ($e_1 *&&
e_2$), which creates a separate process $e_1$ and continues as $e_2$; external
choice ($e_1 *|| e_2$), which allows a process to evolve as either $e_1$ or
$e_2$ (based on some initial event in each of the processes); and sequential
composition ($e_1 ; e_2$).

\begingroup
\setlength\intextsep{0pt}
\begin{wrapfigure}{r}{0.35\textwidth}
  \lstinputlisting[style=myilc]{listings/loop.ilc}
\end{wrapfigure}
Note that replication ($!e$ in the $\pi$-calculus), which allows a process to
spawn repeatedly, is not included for reasons we discuss in
Section~\ref{subsec:types}. Instead, replication can be achieved through
recursive definitions. For example, the recursive function \textsf{loop} (right)
takes as arguments a read channel \textsf{c} and a function \textsf{f}. In the
definition of \textsf{loop}, the let expression binds the value read from
\textsf{c} to the variable \textsf{v} and rebinds the read channel to
\textsf{c}; the body of the let expression applies the function \textsf{f} to
the value \textsf{v}, and then repeats \textsf{loop}.\mypar
\endgroup

\begin{figure*}
  \begin{grammar}
    Expressions
    & $e$
        &$\bnfas$&
        $\eVar{x} \bnfalt \eUnit \bnfalt \ePair{e_1}{e_2} \bnfalt \eInj{i}{e}
    \bnfalt \eRef{e} \bnfalt \eSplit{e_1}{x_1}{x_2}{e_2} \bnfalt
    \eCase{e}{x_1}{e_1}{x_2}{e_2}$
    \\ &&& $\bnfaltbrk \eGet{e} \bnfalt \eSet{e_1}{e_2} \bnfalt \eFix{x}{e}
    \bnfalt \eLet{x}{e_1}{e_2} \bnfalt \eLetBang{x}{e_1}{e_2} \bnfalt \eBang{e}$
    \\ &&& $\bnfaltbrk \eLam{x}{e} \bnfalt \eApp{e_1}{e_2} \bnfalt \eNu{(x_1,
      x_2)}{e} \bnfalt \eWr{e_1}{e_2} \bnfalt \eLetRd{x_1}{x_2}{e_1}{e_2}$
    \\ &&& $\bnfaltbrk \eFork{e_1}{e_2} \bnfalt \eChoice{e_1}{e_2} \bnfalt
    \eSeq{e_1}{e_2}$
  \end{grammar}
  \caption{Syntax of expressions.}
  \label{fig:ilc-syntax}
\end{figure*}


\subsection{Type System}
\label{subsec:types}

\begin{figure*}
  \begin{grammar}
    Types
    & $A,B$
    &$\bnfas$& $\tyUnit \bnfalt \tyProd{A}{B} \bnfalt \tySum{A}{B} \bnfalt \tyRd{A} \bnfalt
    \tyWr{A} \bnfalt \tyRef{A} \bnfalt \tyArr{A}{\footnotesize $m$}{B}$
    \\
    Linear Typing Contexts
    & $\Delta$
    &$\bnfas$& $\emptyctxt \bnfalt \Delta,x:A$
    \\
    Intuitionistic Typing Contexts
    & $\Gamma$
    &$\bnfas$& $\emptyctxt \bnfalt \Gamma,x:A$
  \end{grammar}
\caption{Syntax of types.}
\label{fig:syntax--types}
\end{figure*}


\begin{figure*}[htbp]
  \centering
  \begin{grammar}
    Modes & $m,n,p$ &$\bnfas$& $\Wm \bnfalt \Rm \bnfalt \Vm$
  \end{grammar}

  \judgbox{m || n => p}{~~The parallel composition of modes $m$ and $n$ is mode~$p$.}
  \begin{mathpar}
  \Infer{sym}{m || n => p}{n || m => p}
  \and \Infer{wv}{ }{\Wm || \Vm => \Wm}
  \and \Infer{wr}{ }{\Wm || \Rm => \Wm}
  \and \Infer{rv}{ }{\Rm || \Vm => \Rm}
  \and \Infer{rr}{ }{\Rm || \Rm => \Rm}
  \and \Infer{vv}{ }{\Vm || \Vm => \Vm}
  \end{mathpar}
  \\[2mm]
  \judgbox{m ;; n => p}{~~The sequential composition of modes $m$ and $n$ is mode~$p$.}
  \begin{mathpar}
  \Infer{w*}{ }{\Wm ;; n => n}
  \and \Infer{r$\ast$}{ }{\Rm ;; n => \Rm}
  \and \Infer{v$\ast$}{ }{\Vm ;; n => n}
  \end{mathpar}
  \\[2mm]
  \judgbox{m || n\ \slashed{=>}\ p}{~~The parallel composition of modes $m$ and $n$ is
    \emph{not derivable} for any mode~$p$.}
  \begin{mathpar}
  \Infer{ }{ }{\Wm || \Wm\ \slashed{=>}\ p}
  \end{mathpar}
\caption{Syntax of modes; sequential and parallel mode composition.}
\label{fig:syntax-modes}
\end{figure*}


\begin{figure*}[htbp]
\centering
\judgbox{\Delta ; \Gamma |- e : A |> m}{~~Under $\Delta$ and $\Gamma$, expression~$e$ has
  intuitionistic type $A$ and mode $m$.}
\begin{mathpar}
\Infer{var}
{\Gamma(x) = A}
{\Delta; \Gamma |- x: A}
%
\and
%
\Infer{unit}
{ }
{\Delta ; \Gamma |- \eUnit : \tyUnit}
%
\and
%
\Infer{pair}
{\Delta_1; \Gamma |- e_1 : A_1\\\\
\Delta_2; \Gamma |- e_2 : A_2}
{\Delta_1, \Delta_2; \Gamma |- \ePair{e_1}{e_2} : \tyProd{A_1}{A_2}}
%
\and
%
\Infer{inj}
{i \in \{1, 2\}\\\\
\Delta; \Gamma |- e : A_i}
{\Delta ; \Gamma |- \eInj{i}{e}  : \tySum{A_1}{A_2}}
%
\and
%
\Infer{ref}
{\Delta; \Gamma |- e : A}
{\Delta; \Gamma |- \eRef{e} : \tyRef{A}}
%
\and
%
\Infer{split}
{\Delta_1; \Gamma |- e_1 : A_1 \times A_2\\\\
\Delta_2; \Gamma,x_1:A_1, x_2:A_2 |- e_2 : B |> m}
{\Delta_1, \Delta_2; \Gamma |- \eSplit{e_1}{x_1}{x_2}{e_2} : B |> m}
%
\and
%
\Infer{case}
{\Delta_1; \Gamma |- e : A_1 + A_2\\\\
\Delta_2; \Gamma,x_1:A_1 |- e_1 : B |> m\\\\
\Delta_2; \Gamma,x_2:A_2 |- e_2 : B |> m}
{\Delta_1, \Delta_2; \Gamma |- \eCase{e}{x_1}{e_1}{x_2}{e_2} : B |> m}
%
\and
%
\Infer{get}
{\Delta; \Gamma |- e : \tyRef{A}}
{\Delta; \Gamma |- \eGet{e} : A}
%
\and
%
\Infer{set}
{\Delta_1, \Gamma |- e_1 : \tyRef{A} \\ \Delta_2, \Gamma |- e_2 : A }
{\Delta_1, \Delta_2; \Gamma |- \eSet{e_1}{e_2} : \tyUnit}
%
\and
%
\Infer{fix}
{\Delta; \Gamma, x : \tyArr{A}{\footnotesize m}{A} |- e : \tyArr{A}{\footnotesize m}{A}}
{\Delta; \Gamma |- \eFix{x}{e} : \tyArr{A}{\footnotesize m}{A}}
%
\and
%
\Infer{let}
{m_1 ;; m_2 => m_3\\\\
\Delta_1 ; \Gamma |- e_1 : A |> m_1 \\
\Delta_2 ; \Gamma, x:A |- e_2 : B |> m_2
}
{\Delta_1, \Delta_2 ; \Gamma |- \eLet{x}{e_1}{e_2} : B |> m_3}
%
\and
%
\Infer{let!}
{m_1 ;; m_2 => m_3\\\\
\Delta_1 ; \Gamma |- e_1 : \tyBang{A} |> m_1 \\
\Delta_2 ; \Gamma, x:A |- e_2 : B |> m_2
}
{\Delta_1, \Delta_2 ; \Gamma |- \eLetBang{x}{e_1}{e_2} : B |> m_3}
%
\and
%
\Infer{lam}
{\Delta ; \Gamma, x:A |- e : B |> m}
{\Delta ; \Gamma |- \eLam{x}{e} : \tyArr{A}{\footnotesize m}{B}}
%
\and
%
\Infer{app}
{\Delta_1 ; \Gamma |- e_2 : A\\\\
\Delta_2 ; \Gamma |- e_1 : \tyArr{A}{\footnotesize m}{B}}
{\Delta_1, \Delta_2 ; \Gamma |- \eApp{e_1}{e_2} : B |> m}
%
\and
%
\Infer{nu}
{\Delta, x_1:\tyRd{A}; \Gamma, x_2:\tyWr{A} |- e : B |> m}
{\Delta ; \Gamma |- \eNu{(x_1, x_2)}{e} : B |> m}
%
\and
%
\Infer{wr}
{\Delta_1; \Gamma   |- e_1 : A\\\\
\Delta_2; \Gamma   |- e_2 : \tyWr{A}}
{\Delta_1, \Delta_2; \Gamma |- \eWr{e_1}{e_2} : \tyUnit |> \Wm}
%
\and
%
\Infer{letrd}
{m_1 ;; m_2 => m_3\\\\
\Delta_1 ; \Gamma |- e_1 : \tyRd{A} |> m_1 \\
\Delta_2,x_1:\tyBang{A},x_2:\tyRd{A} ; \Gamma |- e_2 : B |> m_2
}
{\Delta_1, \Delta_2 ; \Gamma |- \eLetRd{x_1}{x_2}{e_1}{e_2} : B |> m_3}
%
\and
%
\Infer{fork}
{m_1 || m_2 => m_3\\\\
\Delta_1; \Gamma |- e_1 : A |> m_1\\
\Delta_2; \Gamma |- e_2 : B |> m_2}
{\Delta_1, \Delta_2; \Gamma |- \eFork{e_1}{e_2} : B |> m_3}
%
\and
%
\Infer{choice}
{\Delta_1; \Gamma |- e_1 : A |> \Rm\\\\
\Delta_2; \Gamma |- e_2 : A |> \Rm
}
{\Delta_1, \Delta_2; \Gamma |- \eChoice{e_1}{e_2} : A |> \Rm}
%
\and
%
\Infer{seq}
{m_1 ;; m_2 => m_3\\\\
\Delta_1; \Gamma |- e_1 : A |> m_1\\
\Delta_2; \Gamma |- e_2 : B |> m_2}
{\Delta_1, \Delta_2; \Gamma |- \eSeq{e_1}{e_2} : B |> m_3}
\end{mathpar}
\judgbox{\Delta ; \Gamma |- e : X |> m}{~~Under $\Delta$ and $\Gamma$, expression~$e$ has linear
  type $X$ and mode $m$.}
\begin{mathpar}
\Infer{lvar}
{\Delta(x) = X}
{\Delta; \Gamma |- x: X}
%
\and
%
\Infer{tensor}
{\Delta_1; \Gamma |- e_1 : X_1\\\\
\Delta_2; \Gamma |- e_2 : X_2}
{\Delta_1, \Delta_2; \Gamma |- \ePair{e_1}{e_2} : \tyTensor{X_1}{X_2}}
%
\and
%
\Infer{bang}
{\Delta ; \Gamma |- e : A }
{\Delta ; \Gamma |- \eBang{e} : \tyBang{A}}
%
\and
%
\Infer{lfix}
{\Delta, x : \tyLolli{X}{\footnotesize m}{X}; \Gamma |- e : \tyLolli{X}{\footnotesize m}{X}}
{\Delta; \Gamma |- \eLfix{x}{e} : \tyLolli{X}{\footnotesize m}{X}}
%
\and
\Infer{lolli}
{\Delta,x:X ; \Gamma |- e : Y |> m}
{\Delta ; \Gamma |- \eLAM{x}{e} : \tyLolli{X}{\footnotesize m}{Y}}
\end{mathpar}
\caption{Expression typing.}
\label{fig:type-expressions}
\end{figure*}


At a high level, ILC's type system adapts ITMs to a subset of the $\pi$-calculus.
The invariants maintained by the type system ensure that the only
non-determinism in an ILC program is due to random coinflips taken by processes,
which have a well-defined distribution. This is essential to ensure that the
normalization of an ILC program has a computational interpretation, as is
necessary in cryptographic reduction proofs. It guarantees that any apparent
concurrency hazards, such as adversarial scheduling of messages in an
asynchronous network, are due to an explicit adversary process $\mc{A}$ rather
than uncertainty built into the model itself.

Typing rules in ILC have the judgement form $\Delta ; \Gamma |- e : A |> m$, read as
``Under $\Delta$ and $\Gamma$, expression~$e$ has type $A$ and mode $m$''.  Here, $\Delta$
denotes a linear typing context, $\Gamma$ denotes a non-linear typing context, and
$m$ is one of three modes: value (\Vm), read (\Rm), and write
(\Wm). Importantly, these typing rules maintain the following two invariants:

\begin{enumerate}
\item \emph{No duplication of read channel ends.} In ILC, each channel (or
  ``tape'' in ITM parlance) has a read end and a write end. The read end of the
  channel is protected against duplication by binding it in the linear
  context $\Delta$. The notation $\Delta_1, \Delta_2$ denotes a partitioning of the read
  channels in $\Delta$. This ensures that no confusion (non-determinism) arises at the receiving end of
  a communication.

\item \emph{No parallel composition of write mode processes.} The typing rules
  do not allow parallel composition of two write mode processes ($\Wm ||
  \Wm$). This ensures that no confusion (non-determinism) arises at the sending
  end of a communication.

\item \emph{No sequential composition of write mode processes.} The typing rules
  do not allow sequential composition of two write mode processes ($\Wm ;;
  \Wm$). This prevents the programmer from writing a process that gets ``stuck''
  trying to perform writes in sequence---a writing process becomes inactive after
  writing and can only reactivate when written to.
\end{enumerate}

\noindent Figure~\ref{fig:syntax--types} gives the syntax of types,
Figure~\ref{fig:syntax-modes} gives the syntax of modes and rules for mode
composition, and Figure~\ref{fig:type-expressions} gives the typing of
expressions (eliding value mode derivations).

\todo{Discuss syntax of types, linear typing rules and let!} The rule nu types a restriction as
$B |> m$ provided that body of the restriction ($e$) has type $B |> m$ under the
assumptions that $x_1$ is a read channel of type $\tyRd{A}$ in the linear
context and $x_2$ is a write channel of type $\tyWr{A}$ in the intuitionistic
context. The rule wr types a write expression as $\tyUnit |> \Wm$ provided that
the value being sent is compatible with the type of the write channel being sent
on. Additionally, the linear context $\Delta$ must be partitioned into contexts $\Delta_1$
and $\Delta_2$, which are used to type the subexpressions $e_1$ and $e_2$
respectively. The same pattern holds for the other typing rules as well. The rule rd types a read expression as $(A ** \tyRd{A}) |> \Rm$, since it
returns a pair containing the value from the channel and the channel itself, so
that it can be rebound. Since read channels are typed linearly, returning and
rebinding read channels allows them to be used more than once. The rule fork is
types a fork as $B |> m_3$, where the type of the right process is $B$ (the type
of the left is ignored) and the mode $m_3$ is derived as the parallel
composition of the modes of the left and right processes ($m_1 || m_2 =>
m_3$). The rule choice is types an external choice as $A |> \Rm$ provided that
it is the type of the left and right processes. The rule seq types a sequence
similarly to the rule fork, except the mode $m_3$ is derived by sequential
composition of the modes of its sub-expressions.

The rule var looks up the binding of $x$ in the non-linear typing context $\Gamma$,
and the rule lvar looks up the binding of $x$ in the linear typing context
$\Delta$. The rules unit, pair, inj, and ref, get, and set are standard. The rule
thunk captures both the type and mode of the suspended expression, which is
typed as $\tyUp{A ** m}$, and the rule force forces the evaluation of the thunk,
which is typed as $A |> m$. The rules split and case are standard, except for
the fact that the body of a \textsf{split} expression ($e_2$) and the branches
of a \textsf{case} expression ($e_1$ and $e_2$) need not be value mode
expressions (i.e., they can include communication). Similarly, the rule fix
allows fixed point expressions to include communication. The rule let is the
standard rule for typing let bindings, except its mode is derived as $m1 ;; m2
=> m3$, which is the sequential composition of the mode of the bound expression
($e_1$) with the mode of the body expression ($e_2$). The rule lam and app are
standard, except a function can include communication and arguments to a
function must be value mode expressions.

\todo{How do you type let (v, c) = rd c?, Why replication violates invariants.}

\lstinputlisting[style=myilc]{listings/repl.ilc}

\subsection{Dynamic Semantics}
\label{subsec:semantics}

\begin{figure*}
\centering
\begin{grammar}
  Channel names & $c$   &$\bnfas$& $\cdots$
  \\
  Process names & $p,q$ &$\bnfas$& $\cdots$
  \\
  Store locations & $\ell$ & $\bnfas$ & $\cdots$
  \\[1mm]
  Name sets
  & $\Names$ 
    & $\bnfas$ & $\emptyNames ~|~ \Names, c ~|~ \Names, p$
  \\
  Stores & $\Store$ & $\bnfas$ & $\Store ~|~ \Store, \ell{:}v$
  \\
  Process pools
  & $\Procs$ 
    & $\bnfas$ & $\emptyProcs ~|~ \Procs, p{:}\proc{e}$
    \\[1mm]
  Configurations
  & $C$
     & $\bnfas$ & $\Config{\Names}{\Store}{\Procs} $
     \\[1mm]
\end{grammar}
\begin{grammar}
 Evaluation
  & $E$
     & $\bnfas$ & 
 $\bullet \bnfalt \eLet{x}{E}{e} \bnfalt \eLetBang{x}{E}{e}$
\\ contexts &&& $\bnfaltbrk \eApp{E}{e} \bnfalt \eApp{v}{E} \bnfalt \eRef{E} \bnfalt \eGet{E}$
     \\ &&& $\bnfaltbrk \eSet{E}{e} \bnfalt \eSet{\ell}{E}$
\\ &&& $\bnfaltbrk \eSplit{E}{x_1}{x_2}{e} \bnfalt \eCase{E}{x_1}{e_1}{x_2}{e_2}$
\\[1mm]
 Read contexts
  & $R$
     & $\bnfas$ & $\bullet \bnfalt \eRd{\eChan{c}} \oplus R \bnfalt R \oplus \eRd{\eChan{c}}$
\end{grammar}
\caption{Channel names, process names, configurations and evaluation contexts. \todo{Arrange grammars side-by-side, for space.}}
\label{fig:configs}
\end{figure*}

\begin{figure*}
\centering
\judgbox{C_1 \equiv C_2}{~~Configurations~$C_1$ and $C_2$ are equivalent.}
\begin{mathpar}
\Infer{permProcs}
{  \Procs_1 \equiv_\textsf{perm} \Procs_2 }
{ \Config{\Names}{\Store}{\Procs_1} \equiv \Config{\Names}{\Store}{\Procs_2} }
\end{mathpar}
%\caption{Structural congruence.}
%\label{fig:structural-congruence}
%\end{figure*}
%
%\begin{figure*}
\judgbox{C_1 ---> C_2}{~~Configuration~$C_1$ reduces to $C_2$.}
\begin{mathpar}
\Infer{local}{ \Store_1 ; e_1 ---> \Store_2 ; e_2 }
{ \Config{\Names}{\Store_1}{\Procs, \ProcNm{p} \proc{E[e_1]}} --->
  \Config{\Names}{\Store_2}{\Procs, \ProcNm{p} \proc{E[e_2]}} }
\and
\Infer{fork}{ q \notin \Names }
{ \Config{\Names}{\Store}{\Procs, \ProcNm{p} \proc{E[ \eFork{e_1}{e_2} }] } --->
  \Config{\Names,q}{\Store}{\Procs, \ProcNm{q} \proc{e_1}, \ProcNm{p} \proc{E[ e_2 ]}}}
\and
\Infer{congr}{
C_1 \equiv C_1' 
\\
C_1' ---> C_2'
\\
C_2' \equiv C_2
}
{ C_1 ---> C_2 }
\and
\Infer{nu}{ c_1, c_2 \notin \Names }
{ \Config{\Names}{\Store}{\Procs, \ProcNm{p} \proc{E[ \eNu{(x_1, x_2)}{e} ]}} --->
  \Config{\Names, c_1, c_2}{\Store}{\Procs, \ProcNm{p} \proc{E[ [\eChan{c_1}/x_1][\eChan{c_2}/x_2]e ]}}}
\and
\Infer{rw}
{ c_2 \leadsto c_1 }
{ \Config{\Names}{\Store}{\Procs, \ProcNm{p} E_1[R[ \eRd{\eChan{c_1}}] ], \ProcNm{q} E_2[ \eWr{\eChan{c_2}}{v}]} --->
  \Config{\Names}{\Store}{\Procs, \ProcNm{p} E_1[ (v, \eChan{c_1})], \ProcNm{q} E_2[ () ]} }
\\
\end{mathpar}

\judgbox{\Store_1 ; e_1 ---> \Store_2 ; e_2}{~~Under store $\Store_1$,
  expression~$e_1$ reduces to~$\Store_2 ; e_2$.}
\begin{mathpar}
\Infer{let}
{}
{ \Store ; \eLet{x}{v}{e} ---> \Store ; [v/x]e }
\and
\Infer{let!}
{}
{ \Store ; \eLetBang{x}{v}{e} ---> \Store ; [v/x]e }
\and
\Infer{app}
{}
{ \Store ; \eApp{(\eLam{x}{e})}{v} ---> \Store ; [v/x]e }
\and
\Infer{split}
{ }
{ \Store ; \eSplit{\ePair{v_1}{v_2}}{x_1}{x_2}{e} ---> \Store ; [v_1/x_1][v_2/x_2]e }
\and
\Infer{case}
{ }
{ \Store ; \eCase{\eInj{i}{v}}{x_1}{e_1}{x_2}{e_2} ---> \Store ; [v/x_i]e_i }
\and
\Infer{fix}
{ }
{ \Store ; \eFix{x}{e} -> \Store ; [\eFix{x}{e} / x] e }
\and
\Infer{ref}
{ \ell \not \in dom(\Store) }
{ \Store ; \eRef{v} -> \Store, \ell : v ; \ell }
\and
\Infer{get}
{ \Store(\ell) = v }
{ \Store ; \eGet{\ell} -> \Store; v }
\and
\Infer{set}
{ }
{ \Store ; \eSet{\ell}{v} -> [\ell \mapsto v]\Store; \eUnit }
\end{mathpar}
\caption{Reduction semantics.}
\label{fig:semantics}
\end{figure*}


Figure~\ref{fig:semantics} gives the reduction semantics for ILC, which defines
a transition relation for \emph{configurations}. A configuration $C$ consists of
a set of communication channels $\Sigma$ and a process pool $\pi$. The main judgement
$C_1 \longrightarrow C_2$ can be read as ``configuration $C_1$ reduces to $C_2$.''
Configuration reduction uses an ancillary judgement for local reduction, which
covers cases in which the configuration does not change. The local reduction
judgement $\sigma_1; \e_1 \longrightarrow \sigma_2; \e_2$ can be read as ``under store $\sigma_1$,
expression $e_1$ reduces to $\sigma_{2}; e_2$. A store $\sigma$ consists of a finite map
from locations $\ell$ to to values. These rules are standard.

In the fork rule, a process with store $\sigma$ and redex $e_1 \xFork e_2$ in
evaluation context $E$ spawns a new process $\sigma; e_1$ and reduces to $E[e_2]$. In
the nu rule, the term $E[ \eNu{(x_1, x_2)}{e} ]$ reduces to $E[
  [\vChan{c_1}/x_1][\vChan{c_2}/x_2]e ]$, where $c_1$ and $c_2$ are fresh
channels added to $\Sigma$. In the rw rule, given that $c_2$ is the corresponding
write channel of $c_1$, denoted $c_2 \leadsto c_1$, the processes $\sigma_1 ; E_1[R[
    \eRd{\vChan{c_1}}] ]$ and $\sigma_2 ; E_2[ \eWr{\vChan{c_2}}{v}]$ reduce to the
terms $\sigma_1 ; E_1[ (v, \vChan{c_1})]$ and $\sigma_2 ; E_2[ \vUnit]$, respectively.

\begin{comment}
\lstinputlisting[style=myilc]{listings/loop_seq_rd.ilc}


\lstinputlisting[style=myilc]{listings/async.ilc}

\begin{func}[$\Func_{\textsc{crs}}$]
    $\Func_{\textsc{crs}}$ proceeds as follows, when parameterized by a distribution $D$.
    \begin{enumerate}
        \item When activated for the first time on input $(\texttt{value}, sid)$, choose a value $d \xleftarrow[]{R} D$ and send $d$ back to the activating party. In each other activation return the value $d$ to the activating party.
    \end{enumerate}
\end{func}

%\begin{ilc}[CRS]
%\lstinputlisting[style=ilc]{listings/F_crs.ilc}
%\end{ilc}

\begin{func}[$\Func_{\textsc{com}}$]
    $\Func_{\textsc{com}}$ proceeds as follows, running with parties $P_1, \ldots, P_n$ and an adversary $S$.
    \begin{enumerate}
        \item Upon receiving a value $(\texttt{Commit}, sid, P_i, P_j, b)$ from $P_i$, where $b \in \{ 0, 1 \}$, record the value $b$ and send the message $(\texttt{Receipt}, sid, P_i, P_j)$ to $P_j$ and $S$. Ignore any subsequent \texttt{Commit} messages.

        \item Upon receiving a value $(\texttt{Open}, sid, P_i, P_j)$ from $P_i$, proceed as follows: If some value $b$ was previously recorded, then send the message $(\texttt{Open}, sid, P_i, P_j, b)$ to $P_j$ and $S$ and halt. Otherwise halt.
    \end{enumerate}
\end{func}

%\begin{ilc}[COM]
%\lstinputlisting[style=ilc]{listings/F_com.ilc}
%\end{ilc}

\end{comment}

\begin{figure}[h!]
\begin{boxedminipage}{\columnwidth}
\begin{centering}
\textbf{$\execUC{\Z}{\pi}{\A}{\F{}}$} \\
\end{centering}
\small
\begin{itemize}[leftmargin=2mm]
\item[] $\nu ~\chan{z2p}~ \chan{z2f}~ \chan{z2a}~ \chan{p2f}~ \chan{p2a}~ \chan{a2f}.$
\item[] \emph{// The environment chooses \msf{SID}, \msf{conf}, and corrupted parties}
\item[] let $(\msf{Corrupted},\msf{SID},\msf{conf}) = \Z\{\chan{z2p},\chan{z2a},\chan{z2f}\}$
\item[] \emph{// The protocol determines \msf{conf'}}
\item[] let $\msf{conf'} = \pi.\mtt{cmap}(\msf{SID},\msf{conf})$
\item[] $|$ $\A{}\{\msf{SID},\msf{conf},\msf{Corrupted},\chan{a2z},\chan{a2p},\chan{a2f}\}$
\item[] $|$ $\F{}\{\msf{SID},\msf{conf'},\msf{Corrupted},\chan{f2z},\chan{f2p},\chan{f2a}\}$
\item[] \emph{// Create instances of parties on demand}
\item[] let $\msf{partyMap} = \msf{ref}~\msf{empty}$
\item[] let $\msf{newParty} \msf{PID} = $ do
  \begin{itemize}[leftmargin=3mm]
  \item[] $\nu ~\chan{f2pp}~ \chan{z2pp}.$
  \item[] $@\msf{partyMap}[\msf{PID}].\msf{f2p} := \chan{f2pp}$
  \item[] $@\msf{partyMap}[\msf{PID}].\msf{z2p} := \chan{z2pp}$
  \item[] $|$ forever do $\{ m \leftarrow \chan{pp2f}; (\msf{PID},m) \rightarrow \chan{f2p}\}$
  \item[] $|$ forever do $\{ m \leftarrow \chan{pp2z}; (\msf{PID},m) \rightarrow \chan{z2p} \}$
  \item[] $|$ $\pi\{\msf{SID},\msf{conf},\chan{p2f}/\chan{pp2z},\chan{p2z}/\chan{pp2z}\}$
  \end{itemize}
\item[] let $\msf{getParty}~\msf{PID} =$
  \begin{itemize}
  \item[] if $\msf{PID} \notin \msf{partyMap}$ then $\msf{newParty}~\msf{PID}$
  \item[] return $@\msf{partyMap}[\msf{PID}]$
  \end{itemize}
\item[] $|$ forever do
  \begin{itemize}[leftmargin=3mm]
  \item[] $(\msf{PID}, m) \leftarrow \chan{z2p}$
  \item[] if $\msf{PID} \in \msf{Corrupted}$ then $\mtt{Z2P}(PID,m) \rightarrow \chan{p2a}$
  \item[] else $m \rightarrow (\msf{getParty}~\msf{PID}).\chan{z2p}$
  \end{itemize}
\item[] $|$ forever do
  \begin{itemize}[leftmargin=3mm]
  \item[] $(\msf{PID}, m) \leftarrow \chan{f2p}$
  \item[] if $\msf{PID} \in \msf{Corrupted}$ then $\mtt{F2P}(PID,m) \rightarrow \chan{p2a}$
  \item[] else $m \rightarrow (\msf{getParty}~\msf{PID}).\chan{f2p}$
  \end{itemize}
\item[] $|$ forever do
  \begin{itemize}[leftmargin=3mm]
  \item[] $|~ \mtt{A2P2F}(\msf{PID}, m) \leftarrow \chan{a2p} $
    \begin{itemize}[leftmargin=2mm]
    \item[] if $\msf{PID} \in \msf{Corrupted}$ then $(\msf{PID},m) \rightarrow \chan{p2f}$
    \end{itemize}
  \item[] $|~ \mtt{A2P2Z}(\msf{PID}, m) \leftarrow \chan{a2p} $
    \begin{itemize}[leftmargin=2mm]
    \item[] if $\msf{PID} \in \msf{Corrupted}$ then $(\msf{PID},m) \rightarrow \chan{p2z}$
    \end{itemize}
  \end{itemize}
\end{itemize}
\end{boxedminipage}
\caption{
\label{fig:execuc}
Definition of the SaUCy execution model. The environment, are run as concurrent processes. A new instance of the protocol $\pi$ is created, on demand, for each party $\msf{PID}$. Messages sent to honest parties are routed according to their \msf{PID}; messages sent to corrupted parties are instead diverted to the adversary.
}
\end{figure}

