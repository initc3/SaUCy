\section{Introduction}
\label{sec:introduction}

The universal composability (UC) framework provides a method for analyzing the
security of cryptographic tasks, while ensuring that security is maintained
under concurrent general composition. More specifically, the security
requirements of a task can be expressed as a self-contained specification called
an \emph{ideal functionality}. If a protocol $\pi$ realizes some ideal
functionality $\mc{F}$ in the UC framework, then $\pi$ will ``behave just like
$\mc{F}$'' in arbitrary contexts.

In addition to strong composition guarantees, another strength of the UC
framework is its generality. Practically any cryptographic task can be expressed
as an ideal functionality, and practically any network model (e.g.,
authenticated vs. unauthenticated channels, synchronous vs. asynchronous
messaging) can be expressed in the UC framework. However, this generality comes
at the cost of UC formalization being quite complicated.

In this paper, \ldots

Adapting theory from UC into a concrete implementation turns out to be
difficult. To begin with, the computational model underlying UC, called
Interacting Turing Machines (ITMs), is not a good fit to existing distributed
languages. In the ITM model, execution is essentially single threaded, passing
control from one process to another each time a message is sent, such that
exactly one process is active at a given time. This is in contrast to the
standard $\pi$-calculus~\cite{milner1999communicating}, for which there may be
many possible execution paths that lead to different outcomes.

Therefore, we have developed a core process calculus, called Interactive Lambda
Calculus (ILC), that adapts ITMs to a subset of the $\pi$-calculus through its
type system. In particular, the invariants maintained by the type system ensures
that any non-determinism is due to random coinflips taken by processes, which
have a well-defined distribution. This is essential to ensure that the
normalization of an ILC program has a computational interpretation, as is
necessary in cryptographic reduction proofs.

%Using ILC, we will implement the UC network model (asynchronous communication,
%Byzantine corruptions, etc.) and executable UC implementations (ideal
%functionalities and simulators) for various primitives in the distributed
%systems and cryptography literature (e.g., distributed consensus, multiparty
%computation, and zero-knowledge proofs).

\begin{comment}
The success of blockchains and cryptocurrencies have raised interest in building
secure software systems that combine consensus protocols~\cite{miller2016honey},
zero-knowledge proofs~\cite{kosba2016hawk}, multiparty
computation~\cite{bentov2017instantaneous}, and other advanced techniques from
distributed computing and cryptography.  However, these primitives are known to
be error-prone and difficult to compose securely.  To the average developer,
reasoning about asynchronous, distributed, and adversarial deployment
environments is unnatural. On top of this, the security of a software system is
generally a whole-system property, but vulnerabilities often arise from
misunderstandings and mismatches as components are
integrated~\cite{chong2016report}.

Our solution is to develop a module system, \saucy, that will simplify the task
of composing distributed protocols and cryptographic primitives.  The novel
design idea of \saucy is to include with each module a rich behavioral
specification in the form of an \emph{ideal functionality}, which serves as a
self-contained specification of all desired security and liveness properties.
This idea is rooted in the theory of \emph{universal composability}
(UC)~\cite{canetti2001universally}, which is widely used in cryptography for
on-paper proofs, but has not yet been adapted for software engineering.  Based
on our prior experience providing formal specifications for smart contract and
blockchain protocols~\cite{bentov2017instantaneous, kosba2016hawk,
  miller2017sprites}, ideal functionalities are well-suited for modular design
of complex security-oriented applications for several reasons:

\begin{enumerate}
\item The UC framework is an established standard for modeling distributed and
  cryptographic protocols, so we can draw on existing literature for ideal
  functionality models.
\item Ideal functionalities are executable specifications, so they are amenable
  to property-based testing and machine-checkable proofs.
\item UC provides the strongest notion of security under concurrent
  composition. When we substitute an ideal functionality for a distributed
  protocol that realizes it, all the properties of the ideal functionality are
  preserved. Hence a developer's understanding of the ideal functionality
  carries over to the distributed implementation.
\end{enumerate}

\subsection{Our Approach}
\label{subsec:approach}

To reap the expected benefits of the \saucy module system, in this work, we will
develop infrastructure to help authors write, test, and verify distributed
protocol and cryptographic primitives.  This work will take place over three
main tasks: The first task focuses on designing a new high-level language called
ILC for expressing protocols and implementing the UC framework, the second
task focuses on developing testing techniques for detecting security and
liveness violations in the presence of Byzantine failures, and the third task
focuses on building a verification tool for mechanizing security and liveness
proofs in the UC framework.\smallskip

\subsection{Organization}
\label{subsec:org}

This paper is organized as follows. Section~\ref{sec:background} provides an
overview of the UC framework and potential applications.
Section~\ref{sec:challenges} highlights several challenges in using
UC. Section~\ref{sec:ilc} describes the design of our programming language
Interactive Lambda Calculus (ILC). Section~\ref{sec:session} describes an
extension of ILC with session types that will enable a form of verification for
ILC programs. Section~\ref{sec:testing} details our plan to develop new
techniques for testing security and liveness of ILC procotols in the presence of
Byzantine failures. Section~\ref{sec:verification} details our plan to develop a
proof assistant for mechanizing UC proofs.
\end{comment}
