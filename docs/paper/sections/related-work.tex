\section{Related Work}
\label{sec:related}

Symbolic UC~\cite{bohl2016symbolic} adapts the UC framework, which is defined in
the context of computational model of cryptography, to the symbolic model. In
particular, they show that certain aspects of the UC framework, such as ideal
functionality specifications and UC composition, still carry over fruitfully to
the symbolic model. They are also able to show that certain results, such as the
impossibility of UC commitments in the standard, can still be observed in the
symbolic model.

In Symbolic UC, protocols are written in an untyped applied $\pi$-calculus, so it
is possible to write protocols in which processes get stuck, for
example. \todo{Can you write protocols in which there is confusion as to which
  party is being written to?}

RSIM~\cite{backes2007reactive}.
