\section{Related Work}
\label{sec:related}

\begin{enumerate}
  \item Symbolic UC~\cite{bohl2016symbolic} transports ideas from the UC
    framework to the symbolic model of cryptography, in which cryptographic
    operations are abstracted as a term process algebra (specifically, a variant
    of the applied $\pi$-calculus) and adversary capabilities are defined by
    deduction rules over these terms. In particular, they show that certain
    aspects of the UC framework, such as ideal functionality specifications and
    UC composition, still carry over to the symbolic model. They are also able
    to show that certain results, such as the impossibility of UC commitments in
    the standard model of cryptography, can still be observed in the symbolic
    model. Although this abstract vantage point leads to simpler security proofs
    that can be amenable to automated reasoning, security guarantees derived
    from symbolic analyses are not as strong as those from computational
    analyses considered in UC and in cryptography more broadly.
  \item RSIM~\cite{backes2007reactive}
  \item CertiCrypt~\cite{barthe2009formal}
  \item EasyCrypt~\cite{barthe2011computer}
  \item ProVerif~\cite{blanchet2010proverif}
  \item CryptoVerif~\cite{blanchet2007cryptoverif}
  \item Cryptol~\cite{lewis2003cryptol}
  \item $\text{F}^{\star}$~\cite{swamy2016dependent}
  \item Spi calculus~\cite{abadi1999calculus} is an extension of the
    $\pi$-calculus with cryptographic primitives. In the spi calculus, as in the
    $\pi$-calculus, channels can be passed over channels, and its scoping rules
    guarantee that an attacker cannot access a channel it is not explicitly
    given (scoping is the basis of security). \todo{Why do we prevent passing channels
    over channels in SaUCy?} The spi calculus allows expressing security
    guarantees as equivalences between spi calculus processes. For example, we
    can say that a protocol maintains the secrecy of a value $x$ by stating that
    the protocol with $x$ is equivalent to the protocol with $x'$, for every
    $x'$. Here, equivalence means equivalence in the eyes of an arbitrary
    environment that interacts with the protocol. \todo{They cannot take the
      standard bisimilarity relation as our notion of equivalence. Why?}

    Although equivalence makes reference to the environment, we do not need to
    give a model of the environment explicitly. Instead, the environment can be
    an arbitrary pi calculus process. In sum, their approach uses the powerful
    scoping constructs of the $\pi$-calculus, the definition of an environment as
    an arbitrary spi calculus process, and the representation of security
    properties (both integrity and security) as equivalences. However, the spi
    calculus does not include any notion of probability or complexity, so it can
    be a useful foundation for symbolic cryptography, but not computational
    cryptography.

    In $\pi$-calculus, the scope of a channel can change during a
    computation. When a process sends a restricted channel to a process outside
    the scope of the restriction, the scope is said to extrude. Why do we
    disallow extrusion in SaUCy? A central idea in spi calculus is to use
    restriction and extrusion to keep track of secret values.

    Another difference is that channels are bidirectional. 
  \item Applied $\pi$-calculus~\cite{abadi2001mobile} is a similar extension to
    the $pi$-calculus. Here, there is no need to craft a special calculus and
    develop its proof techniques for each choice of cryptographic
    operation. Includes name restriction and variable restriction (as in the spi
    calculus), so fresh channels, nonces, and keys can be represented as new
    names. Attacks against protocols rely on equational properties.

    Example for one-way hash functions: Represent hash functions with a unary
    function symbol \textsf{h} (without any equational properties). The absence
    of an inverse for \textsf{h} models one-wayness. In comparison with spi
    calculus, the applied pi calculus permits a more uniform and versatile
    treatment of cryptographic functions (e.g., one-way hash functions,
    encryption/decryption, signatures, XOR), their variants, and their
    properties. \todo{Not clear to me why this is the case.} The spi calculus
    developed the idea that the context represents an active attacker, and
    equivalences capture authenticity and secrecy properties (same as here).
  \item Wysteria?~\cite{rastogi2014wysteria}
  \item Lambda auth?~\cite{miller2014authenticated}
  \item Fowler \etal~\cite{fowler2018session} develop a session typed
    programming language that is confluent. Only allows for fixed and two-party
    communications.
\end{enumerate}
