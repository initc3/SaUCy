\section{Related Work}
\label{sec:related}

Symbolic UC~\cite{bohl2016symbolic} adapts the UC framework, which is defined in
the context of computational model of cryptography, to the symbolic model. In
particular, they show that certain aspects of the UC framework, such as ideal
functionality specifications and UC composition, still carry over to the
symbolic model. They are also able to show that certain results, such as the
impossibility of UC commitments in the standard model of cryptography, can still
be observed in the symbolic model.

%, in which cryptographic operations are abstracted
%as a term process algebra (specifically, a variant of the applied $\pi$-calculus) and adversary capabilities
%are defined by deduction rules over these terms. Although this highly abstract
%view of protocols leads to simpler security proofs that can be amenable to
%automated reasoning,
%
%While some
%existing process calculi enjoy confluence properties, they are insufficient as a
%foundation for UC, because they either rule out nondeterminism entirely or are
%restricted to two-party
%communications~\cite{kobayashi1999linearity,bohl2016symbolic,fowler2018session}. 

RSIM~\cite{backes2007reactive}. CertiCrypt, EasyCrypt, Proverif, Cryptoverif,
Cryptol, $\text{F}^{\star}$.
