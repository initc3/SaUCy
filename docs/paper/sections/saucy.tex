\section{SaUCy}
\label{sec:saucy}

We build a concrete, executable implementation of the UC framework in ILC,
dubbed SaUCy. The \textsf{execUC} function takes as arguments an environment
\textsf{Z}, protocol parties \textsf{P} and \textsf{Q}, an adversary \textsf{A},
and a functionality \textsf{F}.

For a real world execution, programs for the actual protocol participants will be
used for \textsf{P} and \textsf{Q}, a real world adversary for \textsf{A}, and
an ideal functionality for \textsf{F} (to model \textsf{F}-hybrid
protocols). For an ideal world execution, dummy parties will be used for
\textsf{P} and \textsf{Q}, a simulator for \textsf{A}, and an ideal
functionality for \textsf{F}.

\begin{algorithm}
\SetAlgorithmName{Protocol}{protocol}{List of Protocols}
\DontPrintSemicolon

\SetKwBlock{Parameters}{\textnormal{\textsf{Public strings}:}}{}
\Parameters{
  $\sigma$: Random string in $\{0,1\}^{4n}$\;
  ${pk}_0, {pk}_1$: Keys for generator $G_{k} \colon \{0,1\}^n \to \{0,1\}^{4n}$
}\smallskip
\SetKwBlock{Commit}{\textnormal{\textsf{Commit}($b$):}}{}
\Commit{
  $r \leftarrow \{0, 1\}^n$\;
  $x \coloneqq G_{{pk}_b}(r)$\;
  if $b=1$ then $x \coloneqq x \oplus \sigma$\;
  Send $(\mathsf{Commit}, x)$ to receiver.\;
  Upon receiving $(\mathsf{Commit}, x)$ from $A$, $B$ outputs $(\mathsf{Receipt})$
}\smallskip

\SetKwBlock{Decommit}{\textnormal{\textsf{Decommit}($x$):}}{}
\Decommit{
  Send $(b, r)$ to receiver.\;
  Receiver checks $x = G_{{pk}_b}(r)$ for $b = 0$, or $x = G_{{pk}_b}(r) \oplus \sigma$
  for $b = 1$. If verification succeeds, then $B$ outputs $(\mathsf{Open}, b)$.
}
\caption{Universally Composable Commitment}
\label{alg:com}
\end{algorithm}

To model the cryptography needed in universally composable commitments, we
introduce several new syntactic forms---\textsf{kgen}, \textsf{tdp}, and
\textsf{hc}---with the static and dynamic semantics shown in
Figure~\ref{fig:extended-ilc}.

The key generation function \textsf{keygen} generates, on input $1^n$ (security
parameter), a random public key $v_{pk}$ and a trapdoor $v_{td}$. The trapdoor
permutation function \textsf{tdp} computes, on input key $v_k$ and bitstring
$v_{in}$, a bitstring $v_{out}$. The hardcore predicate function \textsf{hc}
generates, on input trapdoor permutation $f_{v_k}$.

\begin{figure*}
  \begin{grammar}
    Expressions
    & $e$
        &$\bnfas$&
        $\eKGen{e} \bnfalt \eTdp{e_1}{e_2} \bnfalt \eHc{e}$
  \end{grammar}
  
  \judgbox{\Delta ; \Gamma |- e : A |> m}{~~Under $\Delta$ and $\Gamma$, expression~$e$ has
  intuitionistic type $A$ and mode $m$.}
  \begin{mathpar}
  \Infer{kgen}
  {\Delta ; \Gamma |- e : [\tyBit]}
  {\Delta; \Gamma |- \eKGen{e}: [\tyBit]}
  %
  \and
  %
  \Infer{eTdp}
  {\Delta_1; \Gamma |- e_1 : [\tyBit]\\
   \Delta_2; \Gamma |- e_2 : [\tyBit]}
  {\Delta_1, \Delta_2; \Gamma |- \eTdp{e_1}{e_2}: [\tyBit]}
  %
  \and
  %
  \Infer{hc}
  {\Delta; \Gamma |- e : \tyArr{[\tyBit]}{}{\tyArr{[\tyBit]}{}{[\tyBit]}}}
  {\Delta; \Gamma |- \eHc{e}: \tyBit}
  \end{mathpar}
  
  \judgbox{\Store_1 ; e_1 ---> \Store_2 ; e_2}{~~Under store $\Store_1$,
    expression~$e_1$ reduces to~$\Store_2 ; e_2$.}
  \begin{mathpar}
  \Infer{kgen}
  {\keyword{\textbf{Gen}}(n) = (v_{pk}, v_{td}) \\ v_{pk}, v_{td} \in \{0,1\}^n}
  { \Store ; \eKGen{n} ---> \Store ; (v_{pk}, v_{td})}
  \and
  \Infer{tdp}
  {\mathbf{f}_{v_k}(v_i) = v_o \\ \mathbf{f} \colon \{0,1\}^n -> \{0,1\}^n -> \{0,1\}^n}
  { \Store ; \eTdp{v_k}{v_i} ---> \Store ; v_o }
  \and
  \Infer{hc}
  {\keyword{\textbf{Hc}}(\mathbf{f}_{v_k}) = v \\ \keyword{\textbf{Hc}} \colon
  (\{0,1\}^n -> \{0,1\}^n) -> \{0, 1\}}
  { \Store ; \eHc{f_{v_k}} ---> \Store ; v}
  \end{mathpar}
%  \begin{mathpar}
%    G_{pk}(r) = (f^{(3n)}_{pk}(r), B(f^{(3n-1)}_{pk}(r)), \ldots, B(f_{pk}(r)), B(r))
%  \end{mathpar}
  \caption{ILC extended with trapdoor permutations.}
  \label{fig:extended-ilc}
\end{figure*}

