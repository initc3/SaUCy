\section{SaUCy}
\label{sec:saucy}

Using ILC, we build a concrete, executable implementation of the UC framework,
dubbed SaUCy. We adapt theory from UC to SaUCy and flesh out our commitments
example.

\subsection{SaUCy?}
\label{subsec:saucy}

The \textsf{execUC} function defines the UC model of protocol execution. It is
parameterized by an environment \textsf{z}, protocol parties \textsf{p} and
\textsf{q}, an adversary \textsf{a}, a functionality \textsf{f}, a security
parameter \textsf{k}, a random bitstring \textsf{r}, and a corruption model
\textsf{crupt}.

\begin{itemize}[leftmargin=*]
  \item \emph{Environment.} The environment's program defines interactions with
    the protocol parties and the adversary, which have different programs in the
    real world and the ideal world (see below). Its job is to distinguish which
    of the worlds it is interacting with.
  \item \emph{Protocol.} In the real world, the program of the protocol parties
    correspond to actual programs for running the protocol. In the ideal world,
    the protocol parties are simply dummy parties, which relay messages between
    the environment and the functionality.
  \item \emph{Adversary.} In the real world, the adversary is simply the dummy
    adversary, which relays messages between the environment and either the
    functionality or a corrupted party. In the ideal world, the adversary is a
    simulator, which must mimic the attack of any real world adversary, but in
    the ideal world.
  \item \emph{Functionality.} In the real world, the functionality is any
    functionality that the real world protocol makes calls to (if any). In the
    ideal world, the functionality is the specification for the protocol under
    analysis.
  \item \emph{Security parameter.} Each process is handed a security parameter
    (a natural number), and must run in a number of steps polynomial in this
    security parameter. We have more to say on this later.
  \item \emph{Corruptions.} The possible corruption models are either party
    \textsf{p} is corrupt, party \textsf{q} is corrupt, or no parties are
    corrupt, which are defined in the following datatype:
    \lstinputlisting[style=myilc]{listings/crupt.ilc}
\end{itemize}

%For a real world execution, the protocol parties contain code for running the
%actual protocol under analysis, the adversary is the dummy adversary, and the
%ideal functionality is any functionality that the protocol makes calls to (if
%any). For an ideal world execution, the protocol parties are simply dummy
%parties, the adversary is a simulator, and the ideal functionality is a
%specification for the protocol under analysis. The environment has the ability
%to interact with each of the executions as they evolve. For the simulation to be
%good, the environment should not be able to distinguish which of the executions
%it is interacting with.

\subsection{Adapting UC Theory to SaUCy}
\label{subsec:uc}

The typing rules do not guarantee termination, let alone polynomial time
normalization, so we define this additional notion.

\begin{definition}[Polynomial time normalization]
  Consider a term \textsf{e} that has the type
  \[\Delta, \Gamma |- \mathsf{e} : \tyNat ->[\tyBit]\ {->}_m\ \tyBit,\]
  where the first argument is a security parameter \textsf{k} and the second
  argument is a random bitstring \textsf{r}.\footnote{The definition of
    polynomial time normalization applies similarly to a term \textsf{e} of type
    $\tyBit$ where the security parameter and random bitstring are free
    variables in \textsf{e}.} We say that \textsf{e} is polynomial time
  normalizable, written \textsf{poly(e)}, if for all random bitstrings
  \textsf{r}, where the length of \textsf{r} is polynomial in the security
  parameter \textsf{k}, the term \textsf{e k r} normalizes to a value \textsf{v}
  in a polynomial (in \textsf{k}) number of steps.
  \todo{The term must be closed to normalize, so no $\Delta,\Gamma$}
\end{definition}

\begin{definition}[Value Distribution] 
  Because processes are confluent, we know that if $\mathsf{e~k~r} \leadsto \mathsf{v}$ then the value $\mathsf{v}$ is unique.
  We can therefore define the probability distribution ensemble $D(\mathsf{e}) = \{ D_{\mathsf{e,k}} \}_\mathsf{k}$ of values given a uniform distribution $U_k$ over $\mathsf{k}$-bit strings $\mathsf{r}$, where
\[
D_{\mathsf{e},\mathsf{k}}(\mathsf{v}) = \sum_{\mathsf{r} \in R} U_{\mathsf{k}}(\mathsf{r}), \quad \textnormal{for~} R = \{ \mathsf{r} ~|~ \mathsf{e~k~r} \leadsto \mathsf{v} \}
\]
\end{definition}

\begin{definition}[Polynomial time in UC]
  The judgement polyUC is defined as 
  \begin{mathpar}
    \Infer{polyUC,}
    {\Delta' ; \Gamma |-
      \keyword{poly}(\keyword{execUC}\ \mc{F}\ \mc{Z}\ \pi\ \mc{A}\ \mc{F}) : \tyBit}
    {\Delta ; \Gamma |- \keyword{polyUC}(\mc{Z}, \pi, \mc{F}, \mc{A})}
  \end{mathpar}
  where $\Delta' = \Delta, \mathsf{k} : \tyBang{\tyNat}, \mathsf{r} :
  \tyBang{[\tyBit]}$. It says that the term
  \[\keyword{execUC}\ \mc{F}\ \mc{Z}\ \pi\ \mc{A}\ \mc{F}\]
  is a polynomial time normalizing term. \todo{Elaborate more.}
  \todo{The term with \textsf{execUC} should be closed, so no $\Delta,\Gamma$}  
\end{definition}

Protocol emulation defines what it means for a protocol to emulate an idealized
version of itself, which relies on an ideal functionality to do all the
work. Stated informally, we say that a protocol $\pi$ emulates a protocol $\phi$ if no
environment $\mc{Z}$ can distinguish between interactions with a known adversary
$\mc{A}$ and $\pi$ versus a simulated adversary $\mc{S}$ and $\phi$. Stated in SaUCy,
the definition is this:
\begin{definition}[Emulation]
  The protocol-functionality pair $(\pi_0, \mc{F}_0)$ emulates $(\pi_1, \mc{F}_1)$
  if for all adversaries $\mc{A}$, there exists a simulator $\mc{S}$
  such that for all environments $\mc{Z}$, if $|- \keyword{polyUC}(\mc{Z}, \pi_0,
  \mc{F}_0, \mc{A})$ then $|- \keyword{polyUC}(\mc{Z}, \pi_1, \mc{F}_1, \mc{S})$,
  and
  \[ D(\mathsf{execUC}\ \mc{Z}\ \pi_0\ \mc{F}_0\ \mc{A}) \approx D(\mathsf{execUC}\ \mc{Z}\ \pi_1\ \mc{F}_1\ \mc{S}).\]
\end{definition}

We can now state useful composition operators, simplifying lemmas, and
notation. For brevity, we pack several notions from UC into a single alternate
definition: UC-realizes.

\begin{definition}[UC-realizes]
  Protocol $\pi$ with $\mc{F}_0$ realizes $\mc{F}_1$, written $\mc{F}_0
  \yrightarrow{$\pi$} \mc{F}_1$, if for all environments $\mc{Z}$, if $|-
  \keyword{polyUC}(\mc{Z}, \pi, \mc{F}_0, \mc{A}_{\mathbbm{1}})$, then
  $|- \keyword{polyUC}(\mc{Z}, \pi_{\mathbbm{1}}, \mc{F}_1, \mc{S})$, and the
  following statistical indistinguishability relation holds
  \[ D(\mathsf{execUC}\ \mc{Z}\ \pi\ \mc{F}_0\ \mc{A}_{\mathbbm{1}}) \approx D(\mathsf{execUC}\ \mc{Z}\ \pi_{\mathbbm{1}}\ \mc{F}_1\ \mc{S}).\]
\end{definition}

\subsection{Universal Composition}
\label{subsec:composition}

\todo{The idea is that the environment can arbitrarily and adaptively choose inputs to the protocol $\phi$. So when we compose with an arbitrary other protocol $\rho$, it is not any worse than what the environment could have done. It essentially creates a proof obligation to ensure isolation of the protocols.}
While the intended use of universal composition is to replace an ideal
functionality with a protocol that securely realizes it, we define universal
composition \todo{Let's just call it protocol composition, not universal composition. It's not really universal.} more generally in terms of replacing one subroutine protocol with
another. Given a protocol $\phi$, a protocol $\rho$ that makes calls to $\phi$,
\todo{instead of ``makes calls'', specify that the \textsf{toF} channel of $\rho$ is connected to the \textsf{fromZ} channel of  $\phi$. }
and a
protocol $\pi$ that emulates $\phi$, then $\rho^{\phi -> \pi}$ is identical to $\rho$ with the
following modifications:
\begin{itemize}[leftmargin=*]
  \item When $\rho$ writes to $\phi$, $\rho^{\phi -> \pi}$ writes to $\pi$.
  \item When $\rho^{\phi -> \pi}$ receives a message from $\pi$, proceed as $\rho$ would when
    it receives the same message from $\phi$.
\end{itemize}

\begin{figure}
\lstinputlisting[style=myilc]{listings/compose.ilc}
\caption{Universal composition operator.}
\label{fig:composition-operator}
\end{figure}

From UC-realizes, we can conclude the original UC formulation for arbitrary
composition:
\begin{theorem}[Composition Theorem]
  If $\mc{F}_0 \yrightarrow{$\pi$} \mc{F}_1$, then for all $\rho$ we
  have that $(\rho^{\pi}, \mc{F}_0)$ emulates $(\rho, \mc{F}_1)$.
\end{theorem}

\subsection{Commitments Example}
\label{subsec:example}

To model the cryptography needed in universally composable commitments, we
introduce several new syntactic forms---\textsf{kgen}, \textsf{tdp}, and
\textsf{hc}---with the static and dynamic semantics shown in
Figure~\ref{fig:extended-ilc}.

The key generation function \textsf{keygen} generates, on input $1^n$ (security
parameter), a random public key $v_{pk}$ and a trapdoor $v_{td}$. The trapdoor
permutation function \textsf{tdp} computes, on input key $v_k$ and bitstring
$v_{in}$, a bitstring $v_{out}$. The hardcore predicate function \textsf{hc}
generates, on input trapdoor permutation $f_{v_k}$.

\begin{figure*}
  \begin{grammar}
    Expressions
    & $e$
        &$\bnfas$&
        $\eKGen{e} \bnfalt \eTdp{e_1}{e_2} \bnfalt \eHc{e}$
  \end{grammar}
  
  \judgbox{\Delta ; \Gamma |- e : A |> m}{~~Under $\Delta$ and $\Gamma$, expression~$e$ has
  intuitionistic type $A$ and mode $m$.}
  \begin{mathpar}
  \Infer{kgen}
  {\Delta ; \Gamma |- e : [\tyBit]}
  {\Delta; \Gamma |- \eKGen{e}: [\tyBit]}
  %
  \and
  %
  \Infer{eTdp}
  {\Delta_1; \Gamma |- e_1 : [\tyBit]\\
   \Delta_2; \Gamma |- e_2 : [\tyBit]}
  {\Delta_1, \Delta_2; \Gamma |- \eTdp{e_1}{e_2}: [\tyBit]}
  %
  \and
  %
  \Infer{hc}
  {\Delta; \Gamma |- e : \tyArr{[\tyBit]}{}{\tyArr{[\tyBit]}{}{[\tyBit]}}}
  {\Delta; \Gamma |- \eHc{e}: \tyBit}
  \end{mathpar}
  
  \judgbox{\Store_1 ; e_1 ---> \Store_2 ; e_2}{~~Under store $\Store_1$,
    expression~$e_1$ reduces to~$\Store_2 ; e_2$.}
  \begin{mathpar}
  \Infer{kgen}
  {\keyword{\textbf{Gen}}(n) = (v_{pk}, v_{td}) \\ v_{pk}, v_{td} \in \{0,1\}^n}
  { \Store ; \eKGen{n} ---> \Store ; (v_{pk}, v_{td})}
  \and
  \Infer{tdp}
  {\mathbf{f}_{v_k}(v_i) = v_o \\ \mathbf{f} \colon \{0,1\}^n -> \{0,1\}^n -> \{0,1\}^n}
  { \Store ; \eTdp{v_k}{v_i} ---> \Store ; v_o }
  \and
  \Infer{hc}
  {\keyword{\textbf{Hc}}(\mathbf{f}_{v_k}) = v \\ \keyword{\textbf{Hc}} \colon
  (\{0,1\}^n -> \{0,1\}^n) -> \{0, 1\}}
  { \Store ; \eHc{f_{v_k}} ---> \Store ; v}
  \end{mathpar}
%  \begin{mathpar}
%    G_{pk}(r) = (f^{(3n)}_{pk}(r), B(f^{(3n-1)}_{pk}(r)), \ldots, B(f_{pk}(r)), B(r))
%  \end{mathpar}
  \caption{ILC extended with trapdoor permutations.}
  \label{fig:extended-ilc}
\end{figure*}


\begin{algorithm}
\SetAlgorithmName{Protocol}{protocol}{List of Protocols}
\DontPrintSemicolon

\SetKwBlock{Parameters}{\textnormal{\textsf{Public strings}:}}{}
\Parameters{
  $\sigma$: Random string in $\{0,1\}^{4n}$\;
  ${pk}_0, {pk}_1$: Keys for generator $G_{k} \colon \{0,1\}^n \to \{0,1\}^{4n}$
}\smallskip
\SetKwBlock{Commit}{\textnormal{\textsf{Commit}($b$):}}{}
\Commit{
  $r \leftarrow \{0, 1\}^n$\;
  $x \coloneqq G_{{pk}_b}(r)$\;
  if $b=1$ then $x \coloneqq x \oplus \sigma$\;
  Send $(\mathsf{Commit}, x)$ to receiver.\;
  Upon receiving $(\mathsf{Commit}, x)$ from $A$, $B$ outputs $(\mathsf{Receipt})$.
}\smallskip

\SetKwBlock{Decommit}{\textnormal{\textsf{Decommit}($x$):}}{}
\Decommit{
  Send $(b, r)$ to receiver.\;
  Receiver checks $x = G_{{pk}_b}(r)$ for $b = 0$, or $x = G_{{pk}_b}(r) \oplus \sigma$
  for $b = 1$. If verification succeeds, then $B$ outputs $(\mathsf{Open}, b)$.
}
\caption{Universally Composable Commitment}
\label{alg:com}
\end{algorithm}
