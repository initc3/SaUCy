\section{UC Theory}
\label{sec:uc}

The typing rules do not guarantee termination, let alone polynomial time
normalization, so we define this additional notion.

\begin{definition}[Polynomial time normalization]
  Consider a term \textsf{e} that has the type
  \[\Delta, \Gamma |- \mathsf{e} : \tyNat ->[\tyBit]\ {->}_m\ \tyBit,\]
  where the first argument is a security parameter \textsf{k} and the second
  argument is a random bitstring \textsf{r}.\footnote{The definition of
    polynomial time normalization applies similarly to a term \textsf{e} of type
    $\tyBit$ where the security parameter and random bitstring are free
    variables in \textsf{e}.} We say that \textsf{e} is polynomial time
  normalizable, written \textsf{poly(e)}, if for all random bitstrings
  \textsf{r}, where the length of \textsf{r} is polynomial in the security
  parameter \textsf{k}, the term \textsf{e k r} normalizes to a value \textsf{v}
  in a polynomial (in \textsf{k}) number of steps.
\end{definition}

Because processes are confluent, we know that there is only one such value
\textsf{v = e k r} for each bitstring \textsf{r}, and that \textsf{e} is a
polynomial time computable function. Hence, a uniform distribution of bitstrings
gives us a distribution of \textsf{v} as $D_{\mathsf{e}}(\mathsf{v})$, which is
the probability that \textsf{e k r} normalizes to \textsf{v} for a particular
\textsf{k} and a uniform choice of \textsf{r}. \todo{Need to elaborate further.}

\begin{definition}[Polynomial time in UC]
  The judgement polyUC is defined as 
  \begin{mathpar}
    \Infer{polyUC,}
    {\Delta' ; \Gamma |-
      \keyword{poly}(\keyword{execUC}\ \mc{F}\ \mc{Z}\ \pi\ \mc{A}\ \mc{F}) : \tyBit}
    {\Delta ; \Gamma |- \keyword{polyUC}(\mc{Z}, \pi, \mc{F}, \mc{A})}
  \end{mathpar}
  where $\Delta' = \Delta, \mathsf{k} : \tyBang{\tyNat}, \mathsf{r} :
  \tyBang{[\tyBit]}$. It says that the term
  \[\keyword{execUC}\ \mc{F}\ \mc{Z}\ \pi\ \mc{A}\ \mc{F}\]
  is a polynomial time normalizing term. \todo{Elaborate more.}
\end{definition}

Protocol emulation defines what it means for a protocol to emulate an idealized
version of itself, which relies on an ideal functionality to do all the
work. Stated informally, we say that a protocol $\pi$ emulates a protocol $\phi$ if no
environment $\mc{Z}$ can distinguish between interactions with a known adversary
$\mc{A}$ and $\pi$ versus a simulated adversary $\mc{S}$ and $\phi$. Stated in SaUCy,
the definition is this:
\begin{definition}[Emulation]
  The protocol-functionality pair $(\pi_0, \mc{F}_0)$ emulates $(\pi_1, \mc{F}_1)$
  if for all adversaries $\mc{A}$, there exists a simulator $\mc{S}$
  such that for all environments $\mc{Z}$, if $|- \keyword{polyUC}(\mc{Z}, \pi_0,
  \mc{F}_0, \mc{A})$ then $|- \keyword{polyUC}(\mc{Z}, \pi_1, \mc{F}_1, \mc{S})$,
  and
  \[ D(\mathsf{execUC}\ \mc{Z}\ \pi_0\ \mc{F}_0\ \mc{A}) \approx D(\mathsf{execUC}\ \mc{Z}\ \pi_1\ \mc{F}_1\ \mc{S}).\]
\end{definition}

We can now state useful composition operators, simplifying lemmas, and
notation. For brevity, we pack several notions from UC into a single alternate
definition: UC-realizes.

\begin{definition}[UC-realizes]
  Protocol $\pi$ with $\mc{F}_0$ realizes $\mc{F}_1$, written $\mc{F}_0
  \yrightarrow{$\pi$} \mc{F}_1$, if for all environments $\mc{Z}$, if $|-
  \keyword{polyUC}(\mc{Z}, \pi, \mc{F}_0, \mc{A}_{\mathbbm{1}})$, then
  $|- \keyword{polyUC}(\mc{Z}, \pi_{\mathbbm{1}}, \mc{F}_1, \mc{S})$, and the
  following statistical indistinguishability relation holds
  \[ D(\mathsf{execUC}\ \mc{Z}\ \pi\ \mc{F}_0\ \mc{A}_{\mathbbm{1}}) \approx D(\mathsf{execUC}\ \mc{Z}\ \pi_{\mathbbm{1}}\ \mc{F}_1\ \mc{S}).\]
\end{definition}

\subsection{Universal Composition}
\label{subsec:composition}

While the intended use of universal composition is to replace an ideal
functionality with a protocol that securely realizes it, we define universal
composition more generally in terms of replacing one subroutine protocol with
another. Given a protocol $\phi$, a protocol $\rho$ that makes calls to $\phi$, and a
protocol $\pi$ that emulates $\phi$, then $\rho^{\phi -> \pi}$ is identical to $\rho$ with the
following modifications:
\begin{itemize}[leftmargin=*]
  \item When $\rho$ writes to $\phi$, $\rho^{\phi -> \pi}$ writes to $\pi$.
  \item When $\rho^{\phi -> \pi}$ receives a message from $\pi$, proceed as $\rho$ would when
    it receives the same message from $\phi$.
\end{itemize}

\begin{figure}
\lstinputlisting[style=myilc]{listings/compose.ilc}
\caption{Universal composition operator.}
\label{fig:composition-operator}
\end{figure}

From UC-realizes, we can conclude the original UC formulation for arbitrary
composition:
\begin{theorem}[Composition Theorem]
  If $\mc{F}_0 \yrightarrow{$\pi$} \mc{F}_1$, then for all $\rho$ we
  have that $(\rho^{\pi}, \mc{F}_0)$ emulates $(\rho, \mc{F}_1)$.
\end{theorem}

\subsection{Application to Commitments}
\label{subsec:application}

To model the cryptography needed in universally composable commitments, we
introduce several new syntactic forms---\textsf{kgen}, \textsf{tdp}, and
\textsf{hc}---with the static and dynamic semantics shown in
Figure~\ref{fig:extended-ilc}.

The key generation function \textsf{keygen} generates, on input $1^n$ (security
parameter), a random public key $v_{pk}$ and a trapdoor $v_{td}$. The trapdoor
permutation function \textsf{tdp} computes, on input key $v_k$ and bitstring
$v_{in}$, a bitstring $v_{out}$. The hardcore predicate function \textsf{hc}
generates, on input trapdoor permutation $f_{v_k}$.

\begin{figure*}
  \begin{grammar}
    Expressions
    & $e$
        &$\bnfas$&
        $\eKGen{e} \bnfalt \eTdp{e_1}{e_2} \bnfalt \eHc{e}$
  \end{grammar}
  
  \judgbox{\Delta ; \Gamma |- e : A |> m}{~~Under $\Delta$ and $\Gamma$, expression~$e$ has
  intuitionistic type $A$ and mode $m$.}
  \begin{mathpar}
  \Infer{kgen}
  {\Delta ; \Gamma |- e : [\tyBit]}
  {\Delta; \Gamma |- \eKGen{e}: [\tyBit]}
  %
  \and
  %
  \Infer{eTdp}
  {\Delta_1; \Gamma |- e_1 : [\tyBit]\\
   \Delta_2; \Gamma |- e_2 : [\tyBit]}
  {\Delta_1, \Delta_2; \Gamma |- \eTdp{e_1}{e_2}: [\tyBit]}
  %
  \and
  %
  \Infer{hc}
  {\Delta; \Gamma |- e : \tyArr{[\tyBit]}{}{\tyArr{[\tyBit]}{}{[\tyBit]}}}
  {\Delta; \Gamma |- \eHc{e}: \tyBit}
  \end{mathpar}
  
  \judgbox{\Store_1 ; e_1 ---> \Store_2 ; e_2}{~~Under store $\Store_1$,
    expression~$e_1$ reduces to~$\Store_2 ; e_2$.}
  \begin{mathpar}
  \Infer{kgen}
  {\keyword{\textbf{Gen}}(n) = (v_{pk}, v_{td}) \\ v_{pk}, v_{td} \in \{0,1\}^n}
  { \Store ; \eKGen{n} ---> \Store ; (v_{pk}, v_{td})}
  \and
  \Infer{tdp}
  {\mathbf{f}_{v_k}(v_i) = v_o \\ \mathbf{f} \colon \{0,1\}^n -> \{0,1\}^n -> \{0,1\}^n}
  { \Store ; \eTdp{v_k}{v_i} ---> \Store ; v_o }
  \and
  \Infer{hc}
  {\keyword{\textbf{Hc}}(\mathbf{f}_{v_k}) = v \\ \keyword{\textbf{Hc}} \colon
  (\{0,1\}^n -> \{0,1\}^n) -> \{0, 1\}}
  { \Store ; \eHc{f_{v_k}} ---> \Store ; v}
  \end{mathpar}
%  \begin{mathpar}
%    G_{pk}(r) = (f^{(3n)}_{pk}(r), B(f^{(3n-1)}_{pk}(r)), \ldots, B(f_{pk}(r)), B(r))
%  \end{mathpar}
  \caption{ILC extended with trapdoor permutations.}
  \label{fig:extended-ilc}
\end{figure*}


\begin{comment}
\subsection{Brain Dump}

\begin{definition}[Protocol Emulation]
Let $\pi$ and $\phi$ be probabilistic polynomial time (p.p.t) protocols. We say
that $\pi$ UC-emulates $\phi$ if for any p.p.t. adversary $\mc{A}$ there exists a
p.p.t. ideal-process adversary $\mc{S}$ such that for any balanced PPT environment
$\mc{Z}$ we have:
\begin{equation*}
\textsc{Exec}_{\phi, \mc{S}, \mc{Z}} \approx \textsc{Exec}_{\pi, \mc{A}, \mc{Z}}.
\end{equation*}
\end{definition}

\begin{lemma}[Protocol Emulation w.r.t. the Dummy Adversary]
Let $\pi$ and $\phi$ be probabilistic polynomial time (p.p.t) protocols. We say
that $\pi$ UC-emulates $\phi$ if for the dummy adversary $\mc{D}$ there exists a
p.p.t. ideal-process adversary $\mc{S}$ such that for any balanced PPT environment
$\mc{Z}$ we have:
\begin{equation*}
\textsc{Exec}_{\phi, \mc{S}, \mc{Z}} \approx \textsc{Exec}_{\pi, \mc{D}, \mc{Z}}.
\end{equation*}
\end{lemma}

\begin{theorem}[Universal Composition]
  Let $\rho$, $\pi$, and $\phi$, be p.p.t protocols such that $\pi$ UC-emulates $\phi$ and
  both $\phi$ and $\pi$ are subroutine respecting. Then protocol $\rho^{\phi -> \pi}$
  UC-emulates protocol $\rho$.
\end{theorem}

\begin{corollary}
  Let $\rho$, $\pi$ be p.p.t protocols such that $\pi$ UC-realizes a p.p.t ideal
  functionality $\mc{F}$, and both $\rho$ and $\pi$ are subroutine respecting. Then
  protocol $\rho^{\pi/\mc{F}}$ UC-emulates protocol $\rho$.
\end{corollary}

\begin{corollary}[Universal Composition: Realizing Functionalities]
  Let $\mc{F}$, $\mc{G}$ be ideal functionalities such that $\mc{F}$ is
  p.p.t. Let $\rho$ be a subroutine respecting protocol that UC-realizes $\mc{G}$,
  and let $\pi$ be a subroutine respecting protocol that UC-realizes
  $\mc{F}$. Then the composed protocol $\rho^{\pi/\mc{F}}$ securely realizes $\mc{G}$.
\end{corollary}

\begin{theorem}
  Protocol $\Pi_{\textsc{com}}$ securely realizes functionality
  $\Func_{\textsc{com}}$ in the CRS model.
\end{theorem}

Let {\sf Bit} be the type of single bits (i.e., 0 or 1), and let {\sf Inf} be
the type of infinite bitstrings. The meaning of an ILC term $\tau$ is given by the
denotation $[\![\tau]\!]\sigma$, which returns, for an infinite bitstring $\sigma{:}{\sf Inf}$, a value
$v{:}{\sf Bit}$. The denotation $[\![\tau]\!]$, then, returns a binary
distribution $d$ over the types of return values for all infinite
bitstrings. Let $\Delta(d_1, d_2)$ denote the statistical distance between two
distributions $d_1$ and $d_2$.
%\[ \Delta(d_1, d_2) \defeq max_{A}|d_1 A - d_2 A|\]

\begin{definition}[$\epsilon$-indistinguishability of ILC Terms]
Let $\tau_1$ and $\tau_2$ be ILC terms, which are closed except for an infinite
bitstream free variable $\sigma{:}{\sf Inf}$. Additionally, for any such $\sigma$,
$[\![t_1]\!]\sigma{:}{\sf Bit}$ and $[\![\tau_2]\!]\sigma{:}{\sf Bit}$. We say that $\tau_1$ and
$\tau_2$ are $\epsilon$-indistinguishable iff $\Delta([\![\tau_1]\!], [\![\tau_2]\!]) \leq \epsilon$.
\end{definition}

\begin{definition}[Probability Distribution Ensemble]
An \emph{ensemble} of probability distributions is a family of probability
distributions $\{ X_{\lambda, z} \}_{\lambda \in \mathbb{N}, z \in {\{0,1\}}^{*}}$ with index
set $\mathbb{N} \times \{0,1\}^{*}$.  The ensembles considered in this work are binary probability
distribution ensembles, which describe single bit outputs of computations, where
$\lambda \in \mathbb{N}$ represents the security parameter, and $z \in \{0,1\}^{*}$
represents input.
\end{definition}

\begin{definition}[Indistinguishability]
Let $X$ and $Y$ be two binary probability distribution
ensembles. We say that $X$ and $Y$ are indistinguishable
(written $X \approx Y$) if for any $c, d \in \mathbb{N}$, there exists
$\lambda_0 \in \mathbb{N}$ such that for all $\lambda > \lambda_0$ and all $z \in \cup_{\lambda \leq \lambda^d}\{0,1\}^{\lambda}$,
\[ | \Pr[X_{\lambda, z} = 1] - \Pr[Y_{\lambda, z} = 1] | < \lambda^{-c}. \]
\end{definition}

\begin{definition}[Bit Producing ILC Term]
Let $\tau$ be an ILC term. We say that $\tau$ is bit producing if it is closed except
for an infinite bitstream free variable $\sigma{:}{\sf Inf}$ and $\sigma{:}{\sf Inf} \vdash
\tau{:}{\sf Bit}$.
%security parameter in judgement
%and converging
\end{definition}

\noindent The denotation $[\![\tau]\!]\sigma$, in which a particular $\sigma$ is
given, evaluates to a value of type {\sf Bit}, and the denotation $[\![\tau]\!]$,
in which no $\sigma$ is specified, evaluates to a binary probability distribution
ensemble over types or values?

\begin{definition}[Indistinguishability of Bit Producing ILC Terms]
Let $\tau_1$ and $\tau_2$ be bit producing ILC terms. We say that $\tau_1$ and $\tau_2$ are
indistinguishable terms if the binary probability distribution ensembles
$[\![\tau_1]\!]$ and $[\![\tau_2]\!]$ are indistinguishable.
\end{definition}
%tau is an ILC+ stream w/ infinite streams and security parameter

\begin{definition}[Protocol Emulation in ILC]
Let $(\pi_1, \mc{F}_1)$ and $(\pi_2, \mc{F}_2)$ be two protocol-functionality
pairs. We say that $(\pi_1, \mc{F}_1)$ UC-emulates $(\pi_2, \mc{F}_2)$ iff for all
adversaries $\mc{A}$, there exists an ideal-process adversary $\mc{S}$ such that
for any environment $\mc{Z}$,
$\textsc{ExecUC}_{\mc{Z}, \mc{A}, \pi_1, \mc{F}_1}$ and
$\textsc{ExecUC}_{\mc{Z}, \mc{S}, \pi_2, \mc{F}_2}$ are bit producing and
indistinguishable terms.
% ExecUC should have sigma and parameter as free variables
% bit respecting adversaries and environments
% first define constraints pi and F for divergence wrt A and Z
\end{definition}

\begin{definition}[Protocol Emulation in ILC]
Let $\pi$ and $\phi$ be probabilistic polynomial time (p.p.t.) protocols. We say that
$\pi$ UC-emulates $\phi$ if for any p.p.t. adversary 
$\mc{A}$, there exists a p.p.t. ideal-process adversary $\mc{S}$
such that for any balanced p.p.t. environment $\mc{Z}$,
$\textsc{ExecUC}_{\phi, \mc{S}, \mc{Z}}$ and $\textsc{ExecUC}_{\pi, \mc{A}, \mc{Z}}$
are indistinguishable bit producing terms.
\end{definition}

\begin{definition}[Protocol Emulation in ILC]
Let $\pi$ and $\phi$ be protocols. We say that $\pi$ UC-emulates $\phi$ iff for all
adversaries $\mc{A}$, there exists an ideal-process adversary $\mc{S}$ such that
for any environment $\mc{Z}$,
$\textsc{ExecUC}[\pi, \mc{A}, \mc{Z}, \lambda, \sigma]$ and
$\textsc{ExecUC}[\phi, \mc{S}, \mc{Z}, \lambda, \sigma]$ are bit producing and
indistinguishable terms.
\end{definition}

\begin{definition}[Balanced Environment]
An environment $\mc{Z}$ is balanced if the overall length of inputs given by
$\mc{Z}$ to the parties of the main instance $\pi$ is at most $k$ times the length
of the input to the adversary.
\end{definition}

Environment should activate the adversary to allow sending of messages.
\end{comment}
