\section{Conclusion and Future Work}
\label{sec:conclusion}

The universal composability (UC) framework is widely used in cryptography
for proofs.
SaUCy takes a step towards mechanizing UC as a programming
framework for constructing and analyzing large systems.
%
%In this work, we have taken the first steps towards putting UC on a proper
%footing by building a concrete, executable implementation of the framework
%called SaUCy. We set apart our work from previous work in a major way: We
%implement SaUCy in a newly designed process calculus called the Interactive
%Lambda Calculus (ILC). Importantly, ILC faithfully abstracts interactive Turing
%machines, the computational model underlying UC, which grants the compelling
%benefit of computational security analysis. \todo{Finish later.}
%
We envision using SaUCy to tackle, for example, applications involving blockchains and smart contracts~\cite{dziembowski2018general,miller2017sprites,dziembowski2017perun}, which comprise an array of cryptography and distributed computing components and suffer from increasingly unwieldy formalisms.
%  Define convenient operators for composition in functionalities,
%  may simplify the process of proofs, simplify what are needed in
%  wrappers.

We can view ILC typechecking of simulators in SaUCy as a partial
mechanization of UC proofs, though the indistinguishability analysis
is still on paper.
Even partial mechanization is useful for catching
bugs; we imagine using SaUCy to systematically implement
functionalities and protocols from the literature and fuzz test
them.
Future work would be to embed ILC within a mechanized proof system,
such as $\textnormal{F}^{*}$ or EasyCrypt.

