\section{Implementation}
\label{sec:implementation}

We have implemented an ILC interpreter in Haskell, which (at present) consists
of 2.3K source lines of code.  For economy of use, our implementation performs
polymorphic type inference, bounded polymorphic mode inference, and affinity
inference. Polymorphism on modes is bounded precisely due to our restriction on
parallel write mode composition. Moreover, a consequence of any kind of mode
polymorphism at all is that the modes of higher order functions can be dependent
on the modes of its function arguments. We give a taste of this below.

This first example features full mode polymorphism and no dependent modes.
\lstinputlisting[style=myilc]{listings/loop.ilc}
The \textsf{loop} function
takes as arguments a read channel \textsf{c} and an intuitionistic function
wrapped in a bang!, which is unpacked as \textsf{f}.  It says to first read from
the channel \textsf{c}, unpack the read value as \textsf{v}, apply the function
\textsf{f} to \textsf{v}, and then recursively call \textsf{loop} again. Here,
the mode $m$ carried over the function argument is fully polymorphic, since no
parallel compositions take place, and the function's mode $\Rm$ is monomorphic.

This next example features bounded mode polymorphism and dependent modes.
\lstinputlisting[style=myilc]{listings/par.ilc}
The \textsf{par} function takes as arguments two functions \textsf{f} and
\textsf{g} and a third argument \textsf{v}. It says to compute \textsf{f v} and
\textsf{g v} in parallel. Here, because write mode expressions cannot be
composed in parallel, at most one of the modes carried over the function
arguments can have mode $\Wm$. Additionally, the mode \textsf{p} carried over
the rightmost arrow is dependent on the modes of the function arguments.
