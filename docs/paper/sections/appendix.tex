\appendix

\section{Algorithmic Type Checking}
\label{sec:algo-type-check}

Figure~\ref{fig:alg-type-check} gives the algorithmic type checking rules.

\begin{figure*}
\centering
\judgbox{\Delta_{in} ; \Gamma_1 |- e : \Delta_{out}; \Gamma_2; A}{~~Under $\Delta_{in}$ and
  $\Gamma_1$, expression~$e$ results $\Delta_{out}$ and $\Gamma_2$ and has intuitionistic type $A$.}
\begin{mathpar}
\Infer{a-var}
{\Gamma(x) = A}
{\Delta; \Gamma |- x : \Delta; \Gamma; A}
%
\and
%
\Infer{a-unit}
{ }
{\Delta ; \Gamma |- \eUnit : \Delta; \Gamma; \tyUnit}
%
\and
%
\Infer{a-pair}
{\Delta_1; \Gamma |- e_1 : \Delta_2; \Gamma; A_1\\\\
\Delta_2; \Gamma |- e_2 : \Delta_3; \Gamma; A_2}
{\Delta_1; \Gamma |- \ePair{e_1}{e_2} : \Delta_3; \Gamma;  \tyProd{A_1}{A_2}}
%
\and
%
\Infer{a-inj}
{i \in \{1, 2\}\\\\
\Delta_1; \Gamma |- e : \Delta_2; \Gamma; A_i}
{\Delta_1 ; \Gamma |- \eInj{i}{e} : \Delta_2; \Gamma; \tySum{A_1}{A_2}}
%
\and
%
\Infer{a-ref}
{\Delta_1; \Gamma |- e : \Delta_2; \Gamma; A}
{\Delta_1; \Gamma |- \eRef{e} : \Delta_2; \Gamma; \tyRef{A}}
%
\and
%
\Infer{a-split}
{\Delta_1; \Gamma |- e_1 : \Delta_2; \Gamma; \tyProd{A_1}{A_2}\\\\
\Delta_2; \Gamma,x_1:A_1, x_2: A_2 |- e : \Delta_3; \Gamma; B}
{\Delta_1; \Gamma |- \eSplit{e_1}{x_1}{x_2}{e_2} : \Delta_3; \Gamma; B}
%
\and
%  
\Infer{a-case}
{\Delta_1; \Gamma |- e : \Delta_2; \Gamma; \tySum{A_1}{A_2}\\\\
\Delta_2; \Gamma,x_1:A_1 |- e_1 : \Delta_3; \Gamma; B\\\\
\Delta_2; \Gamma,x_2:A_2 |- e_2 : \Delta_3; \Gamma; B}
{\Delta_1; \Gamma |- \eCase{e}{x_1}{e_1}{x_2}{e_2} : \Delta_3; \Gamma; B}
%
\and
%
\Infer{a-get}
{\Delta_1; \Gamma |- e : \Delta_2; \Gamma; \tyRef{A}}
{\Delta_1; \Gamma |- \eGet{e} : \Delta_2; \Gamma; A}
%
\and
%
\Infer{a-set}
{\Delta_1; \Gamma |- e_1 : \Delta_2; \Gamma; \tyRef{A}\\\\
\Delta_2; \Gamma |- e_2 : \Delta_3; \Gamma; A}      
{\Delta_1; \Gamma |- \eSet{e_1}{e_2} : \Delta_3; \Gamma; \tyUnit}
%
\and
%
\Infer{a-fix}
{\emptyctxt; \Gamma,x : \tyArr{A}{\footnotesize m}{A} |- e : \emptyctxt; \Gamma; \tyArr{A}{\footnotesize m}{A}}
{\Delta; \Gamma |- \eFix{x}{e} : \Delta; \Gamma; \tyArr{A}{\footnotesize m}{A}}
%
\and
%
\Infer{a-let}
{\Delta_1; \Gamma |- e_1 : \Delta_2; \Gamma; A\\\\
\Delta_2; \Gamma, x:A |- e_2 : \Delta_3; \Gamma; B}
{\Delta_1; \Gamma |- \eLet{x}{e_1}{e_2} : \Delta_3; \Gamma; B}
%
\and
%
\Infer{a-let!}
{\Delta_1; \Gamma |- e_1 : \Delta_2; \Gamma; A\\\\
\Delta_2; \Gamma, x:A |- e_2 : \Delta_3; \Gamma; B}
{\Delta_1; \Gamma |- \eLetBang{x}{e_1}{e_2} : \Delta_3; \Gamma; B}
%
\and
%
\Infer{a-abs}
{\emptyctxt ; \Gamma, x:A |- e : \emptyctxt; \Gamma; B}
{\Delta ; \Gamma |- \eLam{x}{e} : \Delta; \Gamma; \tyArr{A}{\footnotesize m}{B}}
%
\and
%
\Infer{a-app}
{\Delta_1; \Gamma |- e_2 : \Delta_2; \Gamma; A\\\\
\Delta_2; \Gamma |- e_1 : \Delta_3; \Gamma; \tyArr{A}{\footnotesize m}{B}}
{\Delta_1; \Gamma |- \eApp{e_1}{e_2} : \Delta_3; \Gamma; B}
%
\and
%
\Infer{a-nu}
{\Delta_1, x_1: \tyRd{A} ; \Gamma, x_2 : \tyWr{A} |- e : \Delta_2; \Gamma; B}
{\Delta_1; \Gamma |- \eNu{(x_1, x_2)}{e} : \Delta_2; \Gamma; B}
%
\and
%
\Infer{a-wr}
{\Delta_1; \Gamma |- e_1 : \Delta_2; \Gamma; A\\\\
\Delta_2; \Gamma |- e_2 : \Delta_3; \Gamma; \tyWr{A}}
{\Delta_1; \Gamma|- \eWr{e_1}{e_2} : \Delta_3; \Gamma; \tyUnit}
%
\and
%
\Infer{a-letrd}
{\Delta_1; \Gamma |- e_1 : \Delta_2; \Gamma; \tyRd{A}\\\\
\Delta_2, x_2 : \tyRd{A} ; \Gamma, x_1 : A |- e_2 : \Delta_3; \Gamma; B}
{\Delta_1; \Gamma |- \eLetRd{x_1}{x_2}{e_1}{e_2} : \Delta_3; \Gamma; B}
%
\and
%
\Infer{a-fork}
{\Delta_1; \Gamma |- e_1 : \Delta_2; \Gamma; A\\\\
\Delta_2; \Gamma |- e_2 : \Delta_3; \Gamma; B}
{\Delta_1; \Gamma |- \eFork{e_1}{e_2} : \Delta_3; \Gamma; B}
%
\and
%
\Infer{a-choice}
{\Delta_1; \Gamma |- e_1 : \Delta_2; \Gamma; A\\\\
\Delta_1; \Gamma |- e_2 : \Delta_2; \Gamma; A}
{\Delta_1; \Gamma |- \eChoice{e_1}{e_2} : \Delta_2; \Gamma; A}
\end{mathpar}

\judgbox{\Delta_{in} ; \Gamma_1 |- e : \Delta_{out}; \Gamma_2; A}{~~Under $\Delta_{in}$ and
  $\Gamma_1$, expression~$e$ results $\Delta_{out}$ and $\Gamma_2$ and has affine type $A$.}
\begin{mathpar}
\Infer{a-avar}
{\Delta(x) = X}
{\Delta; \Gamma |- x : \Delta ; \Gamma; X}
%
\and
%
\Infer{a-bang}
{\emptyctxt ; \Gamma |- e : \emptyctxt ; \Gamma ; A}
{\Delta ; \Gamma |- \eBang{e} : \Delta ; \Gamma ; \tyBang{A}}
%
\and
%
\Infer{a-tensor}
{\Delta_1; \Gamma |- e_1 : \Delta_2 ; \Gamma ; X_1\\\\
\Delta_2; \Gamma |- e_2 : \Delta_3 ; \Gamma ; X_2}
{\Delta_1, \Delta_2; \Gamma |- \eLPair{e_1}{e_2} : \Delta_3 ; \Gamma ; \tyTensor{X_1}{X_2}}
%
\and
%  
\Infer{a-asplit}
{\Delta_1; \Gamma |- e_1 : \Delta_2; \Gamma; \tyTensor{X_1}{X_2}\\\\
\Delta_2,x_1:X_1, x_2: X_2; \Gamma |- e : \Delta_3; \Gamma; Y}
{\Delta_1; \Gamma |- \eLsplit{e_1}{x_1}{x_2}{e_2} : \Delta_3; \Gamma; Y}
%
\and
%
\Infer{a-afix}
{\Delta_1, x : \tyLolli{X}{\footnotesize m}{X}; \Gamma |- e : \Delta_ 2 ; \Gamma ; \tyLolli{X}{\footnotesize m}{X}}
{\Delta_1; \Gamma |- \eLfix{x}{e} : \Delta_2 ; \Gamma ; \tyLolli{X}{\footnotesize m}{X}}
%
\and
\Infer{a-lolli}
{\Delta_1,x:X ; \Gamma |- e : \Delta_2 ; \Gamma ; Y}
{\Delta_1 ; \Gamma |- \eLAM{x}{e} : \Delta_2 ; \Gamma ; \tyLolli{X}{\footnotesize m}{Y}}
%
\and
%
\Infer{a-aapp}
{\Delta_1; \Gamma |- e_2 : \Delta_2; \Gamma; X\\\\
\Delta_2; \Gamma |- e_1 : \Delta_3; \Gamma; \tyLolli{X}{\footnotesize m}{Y}}
{\Delta_1; \Gamma |- \eLapp{e_1}{e_2} : \Delta_3; \Gamma; Y}
\end{mathpar}
\caption{Algorithmic typing rules.}
\label{fig:alg-type-check}
\end{figure*}

\section{\textsf{execUC}}
\label{sec:full-execUC}

\begin{figure*}
\lstinputlisting[style=myilc]{listings/suc.ilc}
\caption{Full definition of \textsf{execUC}. The channels follow a uniform
naming scheme. The read end of a channel is prefixed with \textsf{r-} and the
write end of a channel is prefixed with \textsf{w-}. The channel \textsf{rZ2P}
denotes the read end of communications from the environment \textsf{z} to the
party \textsf{p}. First, the random bitstring is split amongst each of the five
parties. Then, the functionality, the adversary, and both protocol parties are
spawned in a child process (given the appropriate channels and parameters), and
the process continues as the environment process. Notice that parties are run in
wrapper functions, which alter their behavior depending on whether or not they
are corrupted. If a party is corrupted, then the adversary masquerades as the
party. The mode carried over the rightmost lollipop is $m \in \{(\mathsf{m}_{\mathsf{f}},\mathsf{m}_{\mathsf{a}},\mathsf{m_{\mathsf{z}}) \mid \mathsf{m}_{\mathsf{f}}
|| (\mathsf{m}_{\mathsf{a}} || (\mathsf{R} || (\mathsf{R}
|| \mathsf{m}_{\mathsf{z}}))) => \mathsf{m}_{\mathsf{e}}}\}$.}
\label{fig:execUC}
\end{figure*}

\begin{figure*}
\lstinputlisting[style=myilc]{listings/wrapper.ilc}
\caption{Party wrapper. The \textsf{wrapper} function takes as parameters a
process \textsf{p}, a boolean \textsf{crupt}, the process' associated
communication channels, the security parameter, and a random bitstring. If the
party is corrupted, then messages from the functionality or the other party are
simply forwarded to the adversary, and messages from the adversary are forwarded
to the functionality. Otherwise, the party runs as expected. \todo{Update
signature and body.}}
\label{fig:wrapper}
\end{figure*}

\begin{figure*}
\lstinputlisting[style=myilc]{listings/dummy.ilc}
\caption{Dummy adversary. The dummy adversary forwards messages from the
environment to either the functionality (if the message has
constructor \textsf{A2F}) or the party \textsf{p} (if the message has
constructor \textsf{A2P}). Similarly, the dummy adversary forwards messages from
the functionality or the procotol parties to the environment. \todo{Update signature.}}
\label{fig:dummy-adversary}
\end{figure*}

\begin{figure*}
\lstinputlisting[style=myilc]{listings/dummyp.ilc}
\caption{Dummy party. The dummy party simply relays information between the
environment and the functionality. \todo{Update signature.}}
\label{fig:dummy-party}
\end{figure*}

\begin{figure*}
\lstinputlisting[style=myilc]{listings/Fcrs.ilc}
\caption{Ideal functionality for common reference string (CRS). \todo{keygen?}}
\label{fig:f-crs}
\end{figure*}

%\begin{figure*}
%\lstinputlisting[style=myilc]{listings/committer.ilc}
%\caption{Universally composable commitment committer.}
%\label{fig:committer}
%\end{figure*}
%
%\begin{figure*}
%\lstinputlisting[style=myilc]{listings/receiver.ilc}
%\caption{Universally composable commitment receiver.}
%\label{fig:receiver}
%\end{figure*}

\begin{figure*}
\lstinputlisting[style=myilc]{listings/sim.ilc}
\caption{Simulator for UC commitment.}
\label{fig:sim}
\end{figure*}

\begin{figure*}
\lstinputlisting[style=myilc]{listings/simR.ilc}
\caption{Simulator for UC commitment.}
\label{fig:simR}
\end{figure*}

\section{Cryptography Definitions}

\begin{comment}
\subsection{Brain Dump}

\begin{definition}[Protocol Emulation]
Let $\pi$ and $\phi$ be probabilistic polynomial time (p.p.t) protocols. We say
that $\pi$ UC-emulates $\phi$ if for any p.p.t. adversary $\mc{A}$ there exists a
p.p.t. ideal-process adversary $\mc{S}$ such that for any balanced PPT environment
$\mc{Z}$ we have:
\begin{equation*}
\textsc{Exec}_{\phi, \mc{S}, \mc{Z}} \approx \textsc{Exec}_{\pi, \mc{A}, \mc{Z}}.
\end{equation*}
\end{definition}

\begin{lemma}[Protocol Emulation w.r.t. the Dummy Adversary]
Let $\pi$ and $\phi$ be probabilistic polynomial time (p.p.t) protocols. We say
that $\pi$ UC-emulates $\phi$ if for the dummy adversary $\mc{D}$ there exists a
p.p.t. ideal-process adversary $\mc{S}$ such that for any balanced PPT environment
$\mc{Z}$ we have:
\begin{equation*}
\textsc{Exec}_{\phi, \mc{S}, \mc{Z}} \approx \textsc{Exec}_{\pi, \mc{D}, \mc{Z}}.
\end{equation*}
\end{lemma}

\begin{theorem}[Universal Composition]
  Let $\rho$, $\pi$, and $\phi$, be p.p.t protocols such that $\pi$ UC-emulates $\phi$ and
  both $\phi$ and $\pi$ are subroutine respecting. Then protocol $\rho^{\phi -> \pi}$
  UC-emulates protocol $\rho$.
\end{theorem}

\begin{corollary}
  Let $\rho$, $\pi$ be p.p.t protocols such that $\pi$ UC-realizes a p.p.t ideal
  functionality $\mc{F}$, and both $\rho$ and $\pi$ are subroutine respecting. Then
  protocol $\rho^{\pi/\mc{F}}$ UC-emulates protocol $\rho$.
\end{corollary}

\begin{corollary}[Universal Composition: Realizing Functionalities]
  Let $\mc{F}$, $\mc{G}$ be ideal functionalities such that $\mc{F}$ is
  p.p.t. Let $\rho$ be a subroutine respecting protocol that UC-realizes $\mc{G}$,
  and let $\pi$ be a subroutine respecting protocol that UC-realizes
  $\mc{F}$. Then the composed protocol $\rho^{\pi/\mc{F}}$ securely realizes $\mc{G}$.
\end{corollary}

\begin{theorem}
  Protocol $\Pi_{\textsc{com}}$ securely realizes functionality
  $\Func_{\textsc{com}}$ in the CRS model.
\end{theorem}

Let {\sf Bit} be the type of single bits (i.e., 0 or 1), and let {\sf Inf} be
the type of infinite bitstrings. The meaning of an ILC term $\tau$ is given by the
denotation $[\![\tau]\!]\sigma$, which returns, for an infinite bitstring $\sigma{:}{\sf Inf}$, a value
$v{:}{\sf Bit}$. The denotation $[\![\tau]\!]$, then, returns a binary
distribution $d$ over the types of return values for all infinite
bitstrings. Let $\Delta(d_1, d_2)$ denote the statistical distance between two
distributions $d_1$ and $d_2$.
%\[ \Delta(d_1, d_2) \defeq max_{A}|d_1 A - d_2 A|\]

\begin{definition}[$\epsilon$-indistinguishability of ILC Terms]
Let $\tau_1$ and $\tau_2$ be ILC terms, which are closed except for an infinite
bitstream free variable $\sigma{:}{\sf Inf}$. Additionally, for any such $\sigma$,
$[\![t_1]\!]\sigma{:}{\sf Bit}$ and $[\![\tau_2]\!]\sigma{:}{\sf Bit}$. We say that $\tau_1$ and
$\tau_2$ are $\epsilon$-indistinguishable iff $\Delta([\![\tau_1]\!], [\![\tau_2]\!]) \leq \epsilon$.
\end{definition}

\begin{definition}[Probability Distribution Ensemble]
An \emph{ensemble} of probability distributions is a family of probability
distributions $\{ X_{\lambda, z} \}_{\lambda \in \mathbb{N}, z \in {\{0,1\}}^{*}}$ with index
set $\mathbb{N} \times \{0,1\}^{*}$.  The ensembles considered in this work are binary probability
distribution ensembles, which describe single bit outputs of computations, where
$\lambda \in \mathbb{N}$ represents the security parameter, and $z \in \{0,1\}^{*}$
represents input.
\end{definition}

\begin{definition}[Indistinguishability]
Let $X$ and $Y$ be two binary probability distribution
ensembles. We say that $X$ and $Y$ are indistinguishable
(written $X \approx Y$) if for any $c, d \in \mathbb{N}$, there exists
$\lambda_0 \in \mathbb{N}$ such that for all $\lambda > \lambda_0$ and all $z \in \cup_{\lambda \leq \lambda^d}\{0,1\}^{\lambda}$,
\[ | \Pr[X_{\lambda, z} = 1] - \Pr[Y_{\lambda, z} = 1] | < \lambda^{-c}. \]
\end{definition}

\begin{definition}[Bit Producing ILC Term]
Let $\tau$ be an ILC term. We say that $\tau$ is bit producing if it is closed except
for an infinite bitstream free variable $\sigma{:}{\sf Inf}$ and $\sigma{:}{\sf Inf} \vdash
\tau{:}{\sf Bit}$.
%security parameter in judgement
%and converging
\end{definition}

\noindent The denotation $[\![\tau]\!]\sigma$, in which a particular $\sigma$ is
given, evaluates to a value of type {\sf Bit}, and the denotation $[\![\tau]\!]$,
in which no $\sigma$ is specified, evaluates to a binary probability distribution
ensemble over types or values?

\begin{definition}[Indistinguishability of Bit Producing ILC Terms]
Let $\tau_1$ and $\tau_2$ be bit producing ILC terms. We say that $\tau_1$ and $\tau_2$ are
indistinguishable terms if the binary probability distribution ensembles
$[\![\tau_1]\!]$ and $[\![\tau_2]\!]$ are indistinguishable.
\end{definition}
%tau is an ILC+ stream w/ infinite streams and security parameter

\begin{definition}[Protocol Emulation in ILC]
Let $(\pi_1, \mc{F}_1)$ and $(\pi_2, \mc{F}_2)$ be two protocol-functionality
pairs. We say that $(\pi_1, \mc{F}_1)$ UC-emulates $(\pi_2, \mc{F}_2)$ iff for all
adversaries $\mc{A}$, there exists an ideal-process adversary $\mc{S}$ such that
for any environment $\mc{Z}$,
$\textsc{ExecUC}_{\mc{Z}, \mc{A}, \pi_1, \mc{F}_1}$ and
$\textsc{ExecUC}_{\mc{Z}, \mc{S}, \pi_2, \mc{F}_2}$ are bit producing and
indistinguishable terms.
% ExecUC should have sigma and parameter as free variables
% bit respecting adversaries and environments
% first define constraints pi and F for divergence wrt A and Z
\end{definition}

\begin{definition}[Protocol Emulation in ILC]
Let $\pi$ and $\phi$ be probabilistic polynomial time (p.p.t.) protocols. We say that
$\pi$ UC-emulates $\phi$ if for any p.p.t. adversary 
$\mc{A}$, there exists a p.p.t. ideal-process adversary $\mc{S}$
such that for any balanced p.p.t. environment $\mc{Z}$,
$\textsc{ExecUC}_{\phi, \mc{S}, \mc{Z}}$ and $\textsc{ExecUC}_{\pi, \mc{A}, \mc{Z}}$
are indistinguishable bit producing terms.
\end{definition}

\begin{definition}[Protocol Emulation in ILC]
Let $\pi$ and $\phi$ be protocols. We say that $\pi$ UC-emulates $\phi$ iff for all
adversaries $\mc{A}$, there exists an ideal-process adversary $\mc{S}$ such that
for any environment $\mc{Z}$,
$\textsc{ExecUC}[\pi, \mc{A}, \mc{Z}, \lambda, \sigma]$ and
$\textsc{ExecUC}[\phi, \mc{S}, \mc{Z}, \lambda, \sigma]$ are bit producing and
indistinguishable terms.
\end{definition}

\begin{definition}[Balanced Environment]
An environment $\mc{Z}$ is balanced if the overall length of inputs given by
$\mc{Z}$ to the parties of the main instance $\pi$ is at most $k$ times the length
of the input to the adversary.
\end{definition}

Environment should activate the adversary to allow sending of messages.
\end{comment}



\begin{definition}[Interactive Turing Machine]
\end{definition}

\begin{definition}[Trapdoor Permutations~\cite{lindell2014introduction}]
  A tuple of polynomial-time algorithms $(\mathsf{Gen}, \mathsf{Samp},
  f, \textsf{Inv})$ is a family of trapdoor permutations if:
  \begin{itemize}[leftmargin=*]
    \item The probabilistic parameter-generation algorithm \textsf{Gen}, on
  input $1^n$, outputs $(I, \mathsf{td})$ with $\left| I \right| \geq n$. Each
  value of $I$ defines a set $D_I$ that constitutes the domain and range of a
  permutation (i.e., bijection) $f_I \colon D_I -> D_I$.
    \item Let $\mathsf{Gen}_1$ denote the algorithm that results by
  running \textsf{Gen} and outputting only $I$. Then
  $(\mathsf{Gen}_1, \mathsf{Samp}, f)$ is a family of one-way permutations.
    \item Let $(I, \mathsf{td})$ be an output of $\mathsf{Gen}(1^n)$. The
  deterministic inverting algorithm \textsf{Inv}, on input \textsf{td} and $y \in
  D_I$, outputs $x \in D_I$. We denote this by
  $x \coloneqq \mathsf{Inv}_{\mathsf{td}}(y)$. It is required that with all but
  negligible probability over $(I, \mathsf{td})$ output by $\mathsf{Gen}(1^n)$
  and uniform choice of $x \in D_I$, we have $\mathsf{Inv}_{\mathsf{td}}(f_I(x)) =
  x$.
  \end{itemize}
\end{definition}

\section{Universally Composable Commitment Protocol}
\begin{algorithm}
\SetAlgorithmName{Protocol}{protocol}{List of Protocols}
\DontPrintSemicolon

\SetKwBlock{Parameters}{\textnormal{\textsf{Public strings}:}}{}
\Parameters{
  $\sigma$: Random string in $\{0,1\}^{4n}$\;
  ${pk}_0, {pk}_1$: Keys for generator $G_{k} \colon \{0,1\}^n \to \{0,1\}^{4n}$
}\smallskip
\SetKwBlock{Commit}{\textnormal{\textsf{Commit}($b$):}}{}
\Commit{
  $r \leftarrow \{0, 1\}^n$\;
  $x \coloneqq G_{{pk}_b}(r)$\;
  if $b=1$ then $x \coloneqq x \oplus \sigma$\;
  Send $(\mathsf{Commit}, x)$ to receiver.\;
  Upon receiving $(\mathsf{Commit}, x)$ from $A$, $B$ outputs $(\mathsf{Receipt})$.
}\smallskip

\SetKwBlock{Decommit}{\textnormal{\textsf{Decommit}($x$):}}{}
\Decommit{
  Send $(b, r)$ to receiver.\;
  Receiver checks $x = G_{{pk}_b}(r)$ for $b = 0$, or $x = G_{{pk}_b}(r) \oplus \sigma$
  for $b = 1$. If verification succeeds, then $B$ outputs $(\mathsf{Open}, b)$.
}
\caption{Universally Composable Commitment}
\label{alg:com}
\end{algorithm}



\section{Type Soundness}
\label{sec:ilcproofs}

We first define syntax for process and channel typings, which each map a kind of
identifier (process name or channel name) to its associated type:

\begin{grammar}
    Process pool typings
    %(maps process names to their types)
    & $\PrTy$
    &$\bnfas$& $\emptyctxt \bnfalt \PrTy,\ProcNm{p} A \bnfalt \PrTy,\ProcNm{p} X$
    \\
    Channel pool typings
    %(maps channel names to their types)
    & $\ChTy$
    &$\bnfas$& $\emptyctxt \bnfalt \ChTy,c:\tyRd{S} \bnfalt \ChTy,c:\tyWr{S}$
\end{grammar}

%\subsection{Configuration Typings}

Using the syntax above, we define configuration typing as a straightforward extension
of single-process typing, given in \Secref{subsec:types}:\smallskip

\judgbox{\JCty{\StTy}{\ChTy}{C}{\PrTy}}{Configuration $C$ is well-typed.}
\begin{mathpar}
\Infer{empty}
{ 
  %\StTy ; \ChTy |- \Store : \StTy
}
{\JCty{\StTy}{\ChTy}{\Config{\Names}{\Store}{\emptyProcs}}{\cdot}}
\and
\Infer{cons}
{ \ChTy |- e : U\\
\JCty{\StTy}{\ChTy}{\Config{\Names}{\Store}{\Procs}}{\PrTy}}
{ \JCty{\StTy}{\ChTy}{\Config{\Names}{\Store}{\Procs,p:e}}{\PrTy,(p:U)}}
%\and
%\Infer{cons}
%{\Delta; \Gamma |- e : U\\
%\JCty{\StTy}{\ChTy}{\Config{\Names}{\Store}{\Procs}}{\PrTy}}
%{ \JCty{\StTy}{\ChTy}{\Config{\Names}{\Store}{\Procs,p:e}}{\PrTy,(p:U)}}
\end{mathpar}

\subsection{Progress}
\label{subsec:label}

Progress for the functional fragment of ILC (local progress) is fairly
standard. We follow the usual recipe, except that we give a special definition
of local process termination:\smallskip

\judgbox{\Lterm{e}}{Expression $e$ is locally terminated.}
\begin{mathpar}
\Infer{val}
{ }
{\Lterm{v}}
\and  
\Infer{rdterm}
{ }
{\Lterm{E[\eLetRd{c}{x}{e}]}}
\and
%\Infer{chterm}
%{\Lterm{e_1} \\ \Lterm{e_2}}
%{\Lterm{E[\eChoice{e_1}{e_2}]}}
\Infer{chterm}
{ }
{\Lterm{E[\eChoicee{c_1}{x_1}{e_1}{c_2}{x_2}{e_2}]}}
\and
\Infer{wrterm}
{ }
{\Lterm{E[\eWr{v}{c}]}}
\end{mathpar}
In other words, $\Lterm{e}$ holds when $e$ is a value, is reading (either as a
standalone read or an external choice), or is writing.

\begin{lemma}[Local Progress]
  If $\ChTy |- e : U$, then either $\Lterm{e}$
  or there exists $e'$ such that $e -> e'$.
  \begin{proof}
    By structural induction on the derivation of $\ChTy |- e : U$.
  \end{proof}
\end{lemma}

To state progress on configurations, we give a special definition of ``program
termination'' that permits deadlocks:\smallskip

\judgbox{\JCterm{C}}{Configuration $C$ is terminated.}
\begin{mathpar}
\Infer{Cterm}
{\forall (p:e) \in \pi.~\Lterm{e}\\\\
\textrm{RdChans}(\pi) = \Sigma_1 \\ \textrm{WrChans}(\pi) = \Sigma_2\\\\
\{ (c_1,c_2) \mid c_1 \in \Sigma_1, c_2 \in \Sigma_2, c_2 \leadsto c_1\} = \varnothing}
{\JCterm{\Config{\Names}{}{\Procs}}}
\end{mathpar}
\begin{align*}
  \textrm{RdChans}(\emptyProcs) &= \emptyctxt
  &\textrm{WrChans}(\emptyProcs) &= \emptyctxt
  \\
  \textrm{RdChans}(\pi, p:E[\eLetRd{c}{x}{e}]) &= \textrm{RdChans}(\pi),c
  &\textrm{WrChans}(\pi, p:E[\eLetRd{c}{x}{e}]) &= \textrm{WrChans}(\pi)
  \\
  \textrm{RdChans}(\pi, p:E[\eChoicee{c_1}{x_1}{e_1}{c_2}{x_2}{e_2}]) &=
  \textrm{RdChans}(\pi),c_1,c_2
  &\textrm{WrChans}(\pi, p:E[\eChoicee{c_1}{x_1}{e_1}{c_2}{x_2}{e_2}]) &= \textrm{WrChans}(\pi)  
  \\
  \textrm{RdChans}(\pi, p:E[\eWr{v}{c}]) &= \textrm{RdChans}(\pi)
  &\textrm{WrChans}(\pi, p:E[\eWr{v}{c}]) &= \textrm{WrChans}(\pi),c
  \\
  \textrm{RdChans}(\pi, p:v) &= \textrm{RdChans}(\pi)
  &\textrm{WrChans}(\pi, p:v) &= \textrm{WrChans}(\pi)
\end{align*}
In other words, $\JCterm{C}$ holds when either:
\begin{enumerate}
 \item $C$ is fully normal: Every process in~$C$ is normalized (consists of a
   value).
 \item $C$ is (at least partially) deadlocked: 
   Some (possibly empty) portion of $C$ is normal, and there exists one or more
   reading processes in $C$, or there exists one or more writing processes in
   $C$, however, no read-write channel pair~$(c_1,c_2)$ exists such that $c_2 \leadsto
   c_1$.
\end{enumerate}

%\begin{lemma}[Non-progress]
%If $\JCty{\StTy}{\ChTy}{C}{\PrTy}$ and $\JCterm{C}$, then there does not exist
%$C'$ such that $\JCred{C}{C'}$.
%\begin{proof}
%    By structural induction on the derivation of $\JCty{\StTy}{\ChTy}{C}{\PrTy}$.
%\end{proof}
%\end{lemma}
%\begin{lemma}[Non-progress]
%For all configurations $C$,
%channel typings~$\ChTy$,
%and process typings~$\PrTy$,
%%
%if $\JCty{\StTy}{\ChTy}{C}{\PrTy}$
%and $\JCterm{C}$,
%then there does not exist $C'$ such that $\JCred{C}{C'}$.
%\begin{proof}
%    By structural induction on the derivation of $\JCty{\StTy}{\ChTy}{C}{\PrTy}$.
%\end{proof}
%\end{lemma}

%\begin{lemma}[Parallel Reduction]
%If $\Config{\Names_1}{\Store}{\Procs_1} -> \Config{\Names_2}{\Store}{\Procs_2}$,
%then there exists $\Names_4 \supseteq \Names_3 \supseteq \Names_2$ such that $\Config{\Names_3}{\Store}{\Procs_1, \Procs_3} ->
%\Config{\Names_4}{\Store}{\Procs_2,\Procs_3}$.
%\begin{proof}
%  By structural induction on the derivation of
%  $\Config{\Names_1}{\Store}{\Procs_1} -> \Config{\Names_2}{\Store}{\Procs_2}$.
%\end{proof}
%\end{lemma}

%To state progress on configurations, we will make use of Lemma~\ref{lem:par},
%which allows a portion of a process pool $\pi$ to take a reduction step. \todo{Check}
%
%\begin{lemma}[Parallel Reduction]\label{lem:par}
%If $\Config{\Names_1}{\Store}{\Procs_1} -> \Config{\Names_2}{\Store}{\Procs_2}$,
%then there exists $\Config{\Names_3}{\Store}{\Procs_3}$ such that
%$\Config{\Names_1,\Names_3}{\Store}{\Procs_1, \Procs_3} ->
%\Config{\Names_2,\Names_3}{\Store}{\Procs_2,\Procs_3}$.
%\begin{proof}
%  By structural induction on the derivation of
%  $\Config{\Names_1}{\Store}{\Procs_1} -> \Config{\Names_2}{\Store}{\Procs_2}$.
%\end{proof}
%\end{lemma}

\begin{theorem}[Progress]
If $\JCty{\StTy}{\ChTy}{C}{\PrTy}$, then either $\JCterm{C}$ or there exists
$C'$ such that $\JCred{C}{C'}$.

%For all configurations $C$,
%channel typings~$\ChTy$,
%and process typings~$\PrTy$,
%%
%if $\JCty{\StTy}{\ChTy}{C}{\PrTy}$
%then 
%either $\JCterm{C}$,
%or $\exists C'$ such that $\JCred{C}{C'}$.
\begin{proof}
    By structural induction on the derivation of
    $\JCty{\StTy}{\ChTy}{C}{\PrTy}$.
    \begin{itemize}[leftmargin=*]
    \item[] \textbf{Case}
      \begin{mathpar}
      \Infer{empty}
      { 
        %\StTy ; \ChTy |- \Store : \StTy
      }
      {\JCty{\StTy}{\ChTy}{\Config{\Names}{\Store}{\emptyProcs}}{\cdot}}
      \end{mathpar}
      \begin{llproof}
        %\Pf{\ChTy}{|-}{{\Config{\Names}{\Store}{\emptyProcs}}: \cdot}{By
        %assumption}
        \Pf{}{}{\forall (p:e) \in \emptyProcs.~\Lterm{e}}{Vacuous}
        \Pf{}{}{\Sigma_1 = \textrm{RdChans}(\emptyProcs)=\emptyctxt}{By definition of RdChans}
        \Pf{}{}{\Sigma_2 = \textrm{WrChans}(\emptyProcs)=\emptyctxt}{By definition of
          WrChans}
        \Pf{}{}{\{ (c_1,c_2) \mid c_1 \in \Sigma_1, c_2 \in \Sigma_2, c_2 \leadsto c_1\} = \varnothing}{}        
        \Pf{}{}{\JCterm{{\Config{\Names}{\Store}{\emptyProcs}}}}{By rule Cterm}
      \end{llproof}

    \item[] \textbf{Case}
      \begin{mathpar}
      \Infer{cons}
      { \ChTy |- e : U\\
      \JCty{\StTy}{\ChTy}{\Config{\Names}{\Store}{\Procs}}{\PrTy}}
      { \JCty{\StTy}{\ChTy}{\Config{\Names}{\Store}{\Procs,p:e}}{\PrTy,(p:U)}}
      \end{mathpar}
      
      \begin{llproof}
        \Pf{}{}{\Lterm{e}~\textrm{or}~\exists~e'~\textrm{s.t.}~e -> e'}{By i.h.}
        
        \Pf{}{}{\JCterm{\Config{\Names}{\Store}{\Procs}}~\textrm{or}~\exists
          \Config{\Names'}{\Store}{\Procs'}~\textrm{s.t.}~\Config{\Names}{\Store}{\Procs}
          -> \Config{\Names'}{\Store}{\Procs'}}{By i.h.}

        \Pf{}{}{\textbf{Subcase}~\exists~e'~\textrm{s.t.}~e -> e'}{}

        \Pf{}{}{\quad\textbf{Subsubcase}~\textrm{local}}{}

        \Pf{}{}{\qquad e = E[e_1]~\textrm{and}~e'= E[e_2]}{Suppose}        
        
        \Pf{}{}{\qquad \Config{\Names}{\Store}{\Procs,p:E[e_1]} ->
          \Config{\Names}{\Store}{\Procs,p:E[e_2]}}{By rule local}

        \Pf{}{}{\quad\textbf{Subsubcase}~\textrm{fork}}{}

        \Pf{}{}{\qquad e = E[\eFork{e_1}{e_2}],~e'= E[e_2],~\textrm{and}~q \not \in
          \Names}{Suppose}
        
        \Pf{}{}{\qquad \Config{\Names}{\Store}{\Procs,p:E[\eFork{e_1}{e_2}]} ->
          \Config{\Names,q}{\Store}{\Procs,q:e_1,p:E[e_2]}}{By rule fork}

        \Pf{}{}{\quad\textbf{Subsubcase}~\textrm{nu}}{}

        \Pf{}{}{\qquad e = E[\eNu{(x_1, x_2)}{e_1}],~e'= E[
            [\eChan{c_1}/x_1][\eChan{c_2}/x_2]e_1 ],~c_1,c_2 \not \in
          \Names,~\textrm{and}~c_2 \leadsto c_1}{Suppose}
        
        \Pf{}{}{\qquad \Config{\Names}{\Store}{\Procs,p:[\eNu{(x_1, x_2)}{e_1}]} ->
          \Config{\Names, c_1, c_2}{\Store}{\Procs, \ProcNm{p} \proc{E[
                [\eChan{c_1}/x_1][\eChan{c_2}/x_2]e_1 ]}}}{By rule nu}

        \Pf{}{}{\quad\textbf{Subsubcase}~\textrm{rw}}{}

        \Pf{}{}{\qquad e = E[ \eLetRd{\eChan{c_1}}{x}{e_1} ],~e'=E[
            [\ePair{!v}{\eChan{c_1}}{1}/x]e_1],~\textrm{and}~c_2 \leadsto
          c_1,~\textrm{or}}{}
        \Pf{}{}{\qquad\quad e = E[ \eWr{v}{\eChan{c_2}}],~e'=E[ \eUnit ],~\textrm{and}~c_2 \leadsto
          c_1}{}

        \Pf{}{}{\qquad \textbf{Subsubsubcase}~e = E[ \eLetRd{\eChan{c_1}}{x}{e_1}
          ],~e'=E[ [\ePair{!v}{\eChan{c_1}}{1}/x]e_1],~\textrm{and}~c_2 \leadsto c_1}{}

        \Pf{}{}{\qquad\quad \exists~(\ProcNm{q} E[ \eWr{v}{\eChan{c_2}}]) \in \pi}{By $c_2 \leadsto c_1$}

        \Pf{}{}{\qquad\quad\Config{\Names}{\Store}{\Procs, \ProcNm{p} E[
              \eLetRd{\eChan{c_1}}{x}{e_1} ]
        -> \Config{\Names}{\Store}{\Procs, \ProcNm{p} E[
            [\ePair{!v}{\eChan{c_1}}{1}/x]e_1]}}}{By rule rw}

        \Pf{}{}{\qquad \textbf{Subsubsubcase}~e = E[ \eWr{v}{\eChan{c_2}}],~e'=E[
            \eUnit ],~\textrm{and}~c_2 \leadsto c_1}{}

        \Pf{}{}{\qquad\quad \exists~(\ProcNm{q} E[ \eLetRd{\eChan{c_1}}{x}{e_1} ]) \in \pi}{By $c_2 \leadsto c_1$}

        \Pf{}{}{\qquad\quad\Config{\Names}{\Store}{\Procs, \ProcNm{p} E[ \eWr{v}{\eChan{c_2}}]
        -> \Config{\Names}{\Store}{\Procs, \ProcNm{p} E[
            \eUnit ]}}}{By rule rw}

        \Pf{}{}{\quad\textbf{Subsubcase}~\textrm{cw}}{}

        \Pf{}{}{\qquad e = E[\eChoicee{c_1}{x_1}{e_1}{c_2}{x_2}{e_2}],~e'=E[ [\ePair{!v}{c_i,
          c_{3-i}}{1}/x_i]e_{i}],~c \leadsto c_i,~i \in \{1, 2\},~\textrm{or}}{}
        \Pf{}{}{\qquad\quad e = E[ \eWr{v}{\eChan{c}}],~e'=E[ \eUnit ],~c \leadsto c_i,~i \in \{1, 2\}}{}

        \Pf{}{}{\qquad \textbf{Subsubsubcase}~e =
          E[\eChoicee{c_1}{x_1}{e_1}{c_2}{x_2}{e_2}],~e'=E[ [\ePair{!v}{c_i,
                c_{3-i}}{1}/x_i]e_{i}],}{}
        \Pf{}{}{\qquad\qquad c \leadsto c_i,~i \in \{1, 2\}}{}

        \Pf{}{}{\qquad\quad \exists~(\ProcNm{q} E[ \eWr{v}{\eChan{c}}]) \in \pi}{By $c \leadsto c_i$}

        \Pf{}{}{\qquad\quad\Config{\Names}{\Store}{\Procs, \ProcNm{p}
            E[\eChoicee{c_1}{x_1}{e_1}{c_2}{x_2}{e_2}] ->
            \Config{\Names}{\Store}{\Procs, \ProcNm{p} E[ [\ePair{!v}{c_i,
                    c_{3-i}}{1}/x_i]e_{i}]}}}{By rule cw}

        \Pf{}{}{\qquad \textbf{Subsubsubcase}~e = E[ \eWr{v}{\eChan{c}}],~e'=E[
            \eUnit ],~c \leadsto c_i,~i \in \{1, 2\}}{}

        \Pf{}{}{\qquad\quad \exists~(\ProcNm{q} E[\eChoicee{c_1}{x_1}{e_1}{c_2}{x_2}{e_2}]) \in
          \pi}{By $c \leadsto c_i$}

        \Pf{}{}{\qquad\quad\Config{\Names}{\Store}{\Procs, \ProcNm{p} E[ \eWr{v}{\eChan{c}}]
        -> \Config{\Names}{\Store}{\Procs, \ProcNm{p} E[
            \eUnit ]}}}{By rule cw}        

        \Pf{}{}{\textbf{Subcase}~\exists
          \Config{\Names'}{\Store}{\Procs'}~\textrm{s.t.}~\Config{\Names}{\Store}{\Procs}
          -> \Config{\Names'}{\Store}{\Procs'}}{}
        
        \Pf{}{}{\quad\Config{\Names}{\Store}{\Procs,p:e} ->
          \Config{\Names'}{\Store}{\Procs',p:e}}{By rules local and congr}

\Pf{}{}{\textbf{Subcase}~\JCterm{\Config{\Names}{\Store}{p:e}}~\textrm{and}~\JCterm{\Config{\Names}{\Store}{\Procs}}}{}
        \Pf{}{}{\quad\Names_1 = \textrm{RdChans}(\Procs,p:e)~\textrm{and}~\Names_2 = \textrm{WrChans}(\Procs,p:e)}{Suppose}
        \Pf{}{}{\quad\{ (c_1,c_2) \mid c_1 \in \Names_1,
          c_2 \in \Names_2, c_2 \leadsto c_1\} =
          \varnothing~\textrm{or}}{}
        \Pf{}{}{\qquad\{ (c_1,c_2) \mid c_1 \in \Names_1,
          c_2 \in \Names_2, c_2 \leadsto c_1\} \neq
          \varnothing}{}
        \Pf{}{}{\quad\textbf{Subsubcase}~\{ (c_1,c_2) \mid c_1 \in \Names_1,
          c_2 \in \Names_2, c_2 \leadsto c_1\} =
          \varnothing}{}
        \Pf{}{}{\qquad\JCterm{\Config{\Names}{\Store}{\Procs,p:e}}}{By rule Cterm}
        \Pf{}{}{\quad\textbf{Subsubcase}~\{ (c_1,c_2) \mid c_1 \in \Names_1,
          c_2 \in \Names_2, c_2 \leadsto c_1\} \neq
          \varnothing}{}
        \Pf{}{}{\qquad \exists~c_2 \leadsto c_1~\textrm{s.t.}~c_1 \in \Sigma_1,
          c_2 \in \Sigma_2}{Above}
        
        \Pf{}{}{\qquad\ProcNm{p} v~\textrm{or}~\ProcNm{p} E[ \eLetRd{\eChan{c_1}}{x}{e}
          ]~\textrm{or}~\ProcNm{p}
          E[\eChoicee{c_1}{x_1}{e_1}{c_3}{x_2}{e_2}]~\textrm{or}}{}
        \Pf{}{}{\qquad\quad\ProcNm{p}
          E[\eChoicee{c_3}{x_1}{e_1}{c_1}{x_2}{e_2}]~\textrm{or}~\ProcNm{p} E[
            \eWr{v}{\eChan{c_2}}]}{By definition of \textbf{lterm}}

        \Pf{}{}{\qquad\textbf{Subsubsubcase}~\ProcNm{p} v}{Impossible}

        \Pf{}{}{\qquad\textbf{Subsubsubcase}~\ProcNm{p} E[ \eLetRd{\eChan{c_1}}{x}{e}
        ]}{}

        \Pf{}{}{\qquad\quad\exists~\ProcNm{q} E[ \eWr{v}{\eChan{c_2}}] \in \pi}{By $c_2 \leadsto c_1$}
        
        \Pf{}{}{\qquad\quad\Config{\Names}{\Store}{\Procs, \ProcNm{p} E[
                \eLetRd{\eChan{c_1}}{x}{e} ]} --->
          \Config{\Names}{\Store}{\Procs, \ProcNm{p} E[
              [\ePair{!v}{\eChan{c_1}}{1}/x]e]}}{By rule rw}

        \Pf{}{}{\qquad\textbf{Subsubsubcase}~\ProcNm{p}
          E[\eChoicee{c_1}{x_1}{e_1}{c_3}{x_2}{e_2}]}{}

        \Pf{}{}{\qquad\quad\exists~\ProcNm{q} E[ \eWr{v}{\eChan{c_2}}] \in \pi}{By $c_2 \leadsto c_1$}        

        \Pf{}{}{\qquad\quad\Config{\Names}{\Store}{\Procs, \ProcNm{p}
          E[\eChoicee{c_1}{x_1}{e_1}{c_3}{x_2}{e_2}]} --->
          \Config{\Names}{\Store}{\Procs, \ProcNm{p} E[ [\ePair{!v}{c_1,
                  c_{3}}{1}/x_1]e_{1}]}}{By rule cw}


        \Pf{}{}{\qquad\textbf{Subsubsubcase}~\ProcNm{p}
          E[\eChoicee{c_3}{x_1}{e_1}{c_1}{x_2}{e_2}]}{}

        \Pf{}{}{\qquad\quad\exists~\ProcNm{q} E[ \eWr{v}{\eChan{c_2}}] \in \pi}{By $c_2 \leadsto c_1$}        

        \Pf{}{}{\qquad\quad\Config{\Names}{\Store}{\Procs, \ProcNm{p}
          E[\eChoicee{c_3}{x_1}{e_1}{c_1}{x_2}{e_2}]} --->
          \Config{\Names}{\Store}{\Procs, \ProcNm{p} E[ [\ePair{!v}{c_1,
          c_3}{1}/x_2]e_{2}]}}{By rule cw}        

        \Pf{}{}{\qquad\textbf{Subsubsubcase}~\ProcNm{p} E[
            \eWr{v}{\eChan{c_2}}]}{}

        \Pf{}{}{\qquad\quad\exists~\ProcNm{q} E[ \eLetRd{\eChan{c_1}}{x}{e}
          ] \in \pi~\textrm{or}~\exists~\ProcNm{q}
          E[\eChoicee{c_1}{x_1}{e_1}{c_3}{x_2}{e_2}] \in \pi~\textrm{or}}{}
        \Pf{}{}{\qquad\qquad\exists~\ProcNm{q}
          E[\eChoicee{c_3}{x_1}{e_1}{c_1}{x_2}{e_2}] \in \pi}{By $c_2 \leadsto c_1$}        
        
        \Pf{}{}{\qquad\quad\Config{\Names}{\Store}{\Procs, \ProcNm{p} E[
              \eWr{v}{\eChan{c_2}}]} --->
          \Config{\Names}{\Store}{\Procs, \ProcNm{p} E[ \eUnit ]}}{By rule rw}
      \end{llproof}
    \end{itemize}    
\end{proof}  
\end{theorem}

\subsection{Preservation}

Preservation for the functional fragment of ILC (local preservation) is standard.

\begin{lemma}[Local Preservation]\label{lem:local-preservation}
  If $\ChTy |- e : U$ and $e -> e'$, then there exists $\ChTy' \supseteq \ChTy$ such
  that $\ChTy |- e' : U$.
  \begin{proof}
    By structural induction on the derivation of $e -> e'$.
  \end{proof}
\end{lemma}

To state preservation on configurations, we first state several auxiliary
results, which follow the formulation of Gay and
Vasconcelos~\cite{gay2010linear}.  Lemma~\ref{lem:equiv} shows that typing of
configurations is preserved under configuration equivalence.

\begin{lemma}[Preservation Modulo Equivalence]\label{lem:equiv}
  If $\ChTy |- C : \PrTy$ and $C \equiv C'$, then $\ChTy |- C' : \PrTy$.
  \begin{proof}
    By structural induction on $\ChTy |- C : \PrTy$.
  \end{proof}
\end{lemma}

Lemma~\ref{lem:subterms} shows that a subterm of a well-typed evaluation context
is typeable with a subset of the type contexts. 

\begin{lemma}[Typeability of Subterms]\label{lem:subterms}
  If $\mathcal{D}$ is a derivation of $\ChTy;\Delta; \Gamma |- E[e] : U$ (written $\mathcal{D}
  :: \ChTy;\Delta;\Gamma |- E[e] : U$), then
  \begin{enumerate}
    \item there exists $\ChTy_1,\ChTy_2;\Delta_1, \Delta_2; \Gamma_1,\Gamma_2$ and $V$ such that
      $\ChTy = \ChTy_1,\ChTy_2$, $\Delta = \Delta_1,\Delta_2$, $\Gamma =
      \Gamma_1,\Gamma_2$,
    \item $\mathcal{D}$ has a subderivation $\mathcal{D}'$ (written
      $\mathcal{D}' \sqsubseteq \mathcal{D}$) concluding $\ChTy_1;\Delta_1;\Gamma_1 |- e : V$,
    \item the position of $\mathcal{D}'$ in $\mathcal{D}$ corresponds to the
      position of the hole in $E$ (written $E[\mathcal{D}' \sqsubseteq \mathcal{D}]$).
  \end{enumerate}
  \begin{proof}
    By structural induction on the structure of $E$.
  \end{proof}
\end{lemma}

%\begin{lemma}[Typeability of Subterms]\label{lem:subterms}
%  If $|- E[e] : U$, then there exists a type $X$ (respectively, a type $A$) such
%  that $x : X; \emptyctxt |- E[x] : U$ and $|- e : X$ (respectively, such that
%  $\emptyctxt; x : A |- E[x] : U$ and $|- e : A$).
%  \begin{proof}
%    By structural induction on the structure of $E$.
%  \end{proof}
%\end{lemma}

Lemma~\ref{lem:replacement} shows that the subterm of a well-typed evaluation
context can be replaced.

%\begin{lemma}[Replacement (Evaluation Contexts)]\label{lem:replacement}
%  If
%  \begin{enumerate}
%  \item $\mathcal{D} :: \Delta_1,\Delta_2;\Gamma_1,\Gamma_2 |- E[e] : U$,
%  \item $\mathcal{D}' \sqsubseteq \mathcal{D}$ such that $\mathcal{D}' :: \Delta_2; \Gamma_2 |- e : V$,
%  \item $E[\mathcal{D}' \sqsubseteq \mathcal{D}]$,
%  \item $\Delta_3;\Gamma_3 |- e' : V$,
%  \item $\Delta_1,\Delta_3;\Gamma_1,\Gamma_3$ is defined,
%  \end{enumerate}
%  then $\Delta_1,\Delta_3;\Gamma_1,\Gamma_3 |- E[e'] : U$.
%  \begin{proof}
%    By structural induction on the structure of $E$.
%  \end{proof}  
%\end{lemma}

\begin{lemma}[Replacement (Evaluation Contexts)]\label{lem:replacement}
  If 
  \begin{enumerate}
  \item $\mathcal{D} :: \ChTy_1,\ChTy_2;\Delta_1,\Delta_2;\Gamma_1,\Gamma_2 |- E[e] : U$,
  \item $\mathcal{D}' \sqsubseteq \mathcal{D}$ such that $\mathcal{D}' :: \ChTy_2;\Delta_2; \Gamma_2 |- e : V$,
  \item $E[\mathcal{D}' \sqsubseteq \mathcal{D}]$,
  \item $\ChTy_3;\Delta_3;\Gamma_3 |- e' : V$,
  \item $\ChTy_1,\ChTy_3;\Delta_1,\Delta_3;\Gamma_1,\Gamma_3$ is defined,
  \end{enumerate}
  then $\ChTy_1,\ChTy_3;\Delta_1,\Delta_3;\Gamma_1,\Gamma_3 |- E[e'] : U$.
  \begin{proof}
    By structural induction on the structure of $E$.
  \end{proof}  
\end{lemma}

Finally, Lemmas~\ref{lem:sub-int} and~\ref{lem:sub-aff} show that typing of
terms is preserved by substitution.

\begin{lemma}[Substitution (Intuitionistic)]\label{lem:sub-int}
  If
  \begin{enumerate}
  \item $\ChTy_1; \Delta_1; \Gamma_1, x : A |- e : U$,
  \item $\ChTy_2; \Delta_2; \Gamma_2 |- e' : A$,
  \item $\ChTy_1,\ChTy_2 ; \Delta_1,\Delta_2 ; \Gamma_1,\Gamma_2$ is defined,
  \end{enumerate}
  then $\ChTy_1,\ChTy_2; \Delta_1,\Delta_2; \Gamma_1,\Gamma_2 |- [e'/x]e : U$.
  \begin{proof}
    By structural induction on the derivation of $\ChTy_1; \Delta_1; \Gamma_1, x : A |- e : U$.
  \end{proof}
\end{lemma}

\begin{lemma}[Substitution (Affine)]\label{lem:sub-aff}
  If
  \begin{enumerate}
  \item $\ChTy_1; \Delta_1, x : X; \Gamma_1 |- e : U$,
  \item $\ChTy_2; \Delta_2; \Gamma_2 |- e' : X$,
  \item $\ChTy_1,\ChTy_2 ; \Delta_1,\Delta_2 ; \Gamma_1,\Gamma_2$ is defined,
  \end{enumerate}
  then $\ChTy_1,\ChTy_2; \Delta_1,\Delta_2; \Gamma_1,\Gamma_2 |- [e'/x]e : U$.
  \begin{proof}
    By structural induction on the derivation of $\ChTy_1; \Delta_1, x : X; \Gamma_1 |- e : U$.
  \end{proof}
\end{lemma}

\begin{lemma}[Substitution (Read Channel)]\label{lem:sub-rd}
  If
  \begin{enumerate}
  \item $\ChTy_1; \Delta_1, x : \tyRd{S}; \Gamma_1 |- e : U$,
  \item $\ChTy_2; \Delta_2; \Gamma_2 |- c : \tyRd{S}$,
  \item $\ChTy_1,\ChTy_2 ; \Delta_1,\Delta_2 ; \Gamma_1,\Gamma_2$ is defined,
  \end{enumerate}
  then $\ChTy_1,\ChTy_2; \Delta_1,\Delta_2; \Gamma_1,\Gamma_2 |- [c/x]e : U$.
  \begin{proof}
    By structural induction on the derivation of $\ChTy_1; \Delta_1, x : \tyRd{S}; \Gamma_1 |- e : U$.
  \end{proof}
\end{lemma}

\begin{lemma}[Substitution (Write Channel)]\label{lem:sub-wr}
  If
  \begin{enumerate}
  \item $\ChTy_1; \Delta_1; \Gamma_1, x : \tyWr{S} |- e : U$,
  \item $\ChTy_2; \Delta_2; \Gamma_2 |- c : \tyWr{S}$,
  \item $\ChTy_1,\ChTy_2 ; \Delta_1,\Delta_2 ; \Gamma_1,\Gamma_2$ is defined,
  \end{enumerate}
  then $\ChTy_1,\ChTy_2; \Delta_1,\Delta_2; \Gamma_1,\Gamma_2 |- [c/x]e : U$.
  \begin{proof}
    By structural induction on the derivation of $\ChTy_1; \Delta_1; \Gamma_1, x : \tyWr{S} |- e : U$.
  \end{proof}
\end{lemma}

\begin{theorem}[Preservation]
If $\JCty{\StTy}{\ChTy}{C}{\PrTy}$ and $\JCred{C}{C'}$, then there exists
$\ChTy' \supseteq \ChTy$ and $\PrTy' \supseteq \PrTy$ such that
$\JCty{\StTy'}{\ChTy'}{C'}{\PrTy'}$.
\begin{proof}
    By structural induction on the derivation of $\JCred{C}{C'}$.
  \begin{itemize}[leftmargin=*]
  \item[] \textbf{Case}
    \begin{mathpar}
      \Infer{local}{e_1 ---> e_2 }
      { \Config{\Names}{\Store_1}{\Procs, \ProcNm{p} \proc{E[e_1]}} --->
        \Config{\Names}{\Store_2}{\Procs, \ProcNm{p} \proc{E[e_2]}} }
    \end{mathpar}
    \begin{llproof}
      \Pf{\ChTy}{|-}{\Config{\Names}{\Store_1}{\Procs, \ProcNm{p} \proc{E[e_1]}}
        : \PrTy~\textrm{s.t.}~\PrTy = \PrTy_{\pi},p : U,}{}
      \Pf{}{}{\quad \ChTy = \ChTy_1,\ChTy_2, ~\textrm{and}~\mathcal{D} ::
        \ChTy_1,\ChTy_2|- E[e_1] : U}{Assumption}

      \Pf{}{}{\exists~\mathcal{D}'\sqsubseteq\mathcal{D}~\textrm{s.t.}~\mathcal{D}' :: \ChTy_2 |-
        e_1 : V~\textrm{and}~E[\mathcal{D}'\sqsubseteq\mathcal{D}]}{By
        Lemma~\ref{lem:subterms}}

      \Pf{\ChTy_2}{|-}{e_2 : V}{By i.h. and Lemma~\ref{lem:local-preservation}}

      \Pf{\ChTy_1,\ChTy_2}{|-}{E[e_2] : U}{By Lemma~\ref{lem:replacement}}

      \Pf{\ChTy}{|-}{E[e_2] : U}{By above equalities}      

      \Pf{\ChTy}{|-}{\Config{\Names}{\Store}{\Procs} : \PrTy_{\pi}}{Above}      

      \Pf{\ChTy}{|-}{\Config{\Names}{\Store}{\Procs, \ProcNm{p} E[e_2]} :
        (\PrTy_{\pi}, p : U)}{By rule cons}

      \Pf{\ChTy}{|-}{\Config{\Names}{\Store}{\Procs, \ProcNm{p} E[e_2]} :
        \PrTy}{By above equalities}      
      
      \Pf{}{}{\ChTy' = \ChTy~\textrm{and}~\PrTy' = \PrTy}{Suppose}

      \Pf{\ChTy'}{|-}{\Config{\Names}{\Store}{\Procs, \ProcNm{p} E[e_2]} :
        \PrTy'}{By above equalities}      
    \end{llproof}

  \item[] \textbf{Case}
    \begin{mathpar}
      \Infer{fork}{ q \notin \Names }
      { \Config{\Names}{\Store}{\Procs, \ProcNm{p} \proc{E[ \eFork{e_1}{e_2} }] } --->
        \Config{\Names,q}{\Store}{\Procs, \ProcNm{q} \proc{e_1}, \ProcNm{p} \proc{E[ e_2 ]}}}      
    \end{mathpar}
    \begin{llproof}
      \Pf{\ChTy}{|-}{\Config{\Names}{\Store_1}{\Procs, \ProcNm{p} \proc{E[\eFork{e_1}{e_2}]}}
        : \PrTy~\textrm{s.t.}~\PrTy = \PrTy_{\pi},p : U,}{}
      \Pf{}{}{\quad\ChTy = \ChTy_1,\ChTy_2,~\textrm{and}~\mathcal{D} ::
        \ChTy_1,\ChTy_2|- E[\eFork{e_1}{e_2}] : U}{Assumption}

      \Pf{}{}{\exists~\mathcal{D}'\sqsubseteq\mathcal{D}~\textrm{s.t.}~\mathcal{D}' :: \ChTy_2 |-
        \eFork{e_1}{e_2} : V_2~\textrm{and}~E[\mathcal{D}'\sqsubseteq\mathcal{D}]}{By
        Lemma~\ref{lem:subterms}}

      \Pf{\ChTy_2}{|-}{e_1 : V_1}{By inversion on fork}            

      \Pf{\ChTy_2}{|-}{e_2 : V_2}{By inversion on fork}

      \Pf{\ChTy_1,\ChTy_2}{|-}{E[e_2] : U}{By Lemma~\ref{lem:replacement}}

      \Pf{\ChTy}{|-}{E[e_2] : U}{By above equalities}      

      \Pf{\ChTy}{|-}{\Config{\Names}{\Store}{\Procs} : \PrTy_{\pi}}{Above}

      \Pf{\ChTy}{|-}{\Config{\Names,q}{\Store}{\Procs} : \PrTy_{\pi}}{By $q \not \in
        \Sigma$}

%      \Pf{\ChTy}{|-}{\Config{\Names,q}{\Store}{\ProcNm{p} \proc{E[e_2]}} : p :
%        U_p}{Above}

      \Pf{\ChTy}{|-}{\Config{\Names,q}{\Store}{\Procs, \ProcNm{q} \proc{e_1}} : (\PrTy_{\pi},
        q : V_1)}{By rule cons}

      \Pf{\ChTy}{|-}{\Config{\Names,q}{\Store}{\Procs, \ProcNm{q} \proc{e_1},
          \ProcNm{p} \proc{E[e_2]}} : (\PrTy_{\pi}, q : V_1, p : U)}{By
        rule cons}

      \Pf{\ChTy}{|-}{\Config{\Names,q}{\Store}{\Procs, \ProcNm{q} \proc{e_1},
          \ProcNm{p} \proc{E[e_2]}} : \PrTy,q : V_1}{By above equalities}            

      \Pf{}{}{\ChTy' = \ChTy~\textrm{and}~\PrTy' = \PrTy,q:V_1}{Suppose}

      \Pf{\ChTy'}{|-}{\Config{\Names,q}{\Store}{\Procs, \ProcNm{q} \proc{e_1},
          \ProcNm{p} \proc{E[e_2]}} : \PrTy'}{By above equalities}            
    \end{llproof}

  \item[] \textbf{Case}
    \begin{mathpar}
      \Infer{congr}{
      C_1 \equiv C_1' 
      \\
      C_1' ---> C_2'
      \\
      C_2' \equiv C_2
      }
      { C_1 ---> C_2 }
    \end{mathpar}
    \begin{llproof}
      \Pf{\ChTy}{|-}{C_1 : \PrTy}{Assumption}
      \Pf{}{}{C_1 \equiv C_1'}{Given}
      \Pf{\ChTy}{|-}{C_1' : \PrTy}{By Lemma~\ref{lem:equiv}}
      \Pf{}{}{\ChTy'\supseteq\ChTy~\textrm{and}~\PrTy'\supseteq\PrTy}{Suppose}
      \Pf{\ChTy'}{|-}{C_2' : \PrTy'}{By i.h.}
      \Pf{\ChTy'}{|-}{C_2 : \PrTy'}{By Lemma~\ref{lem:equiv}}
    \end{llproof}

  \item[] \textbf{Case}
    \begin{mathpar}
      \Infer{nu}{ c_1, c_2 \notin \Names \\ c_2 \leadsto c_1}
      { \Config{\Names}{\Store}{\Procs, \ProcNm{p} \proc{E[ \eNu{(x_1, x_2)}{e} ]}} --->
        \Config{\Names, c_1, c_2}{\Store}{\Procs, \ProcNm{p} \proc{E[ [\eChan{c_1}/x_1][\eChan{c_2}/x_2]e ]}}}
    \end{mathpar}
    \begin{llproof}
      \Pf{\ChTy}{|-}{\Config{\Names}{\Store_1}{\Procs, \ProcNm{p} \proc{E[
              \eNu{(x_1, x_2)}{e} ]}} : \PrTy~\textrm{s.t.}~\PrTy = \PrTy_{\pi}, p:U,}{}
      \Pf{}{}{\quad \ChTy = \ChTy_1,\ChTy_2~\textrm{and}~\mathcal{D} ::
        \ChTy_1,\ChTy_2|- \proc{E[ \eNu{(x_1, x_2)}{e} ]} : U}{Assumption}

      \Pf{}{}{\exists~\mathcal{D}'\sqsubseteq\mathcal{D}~\textrm{s.t.}~\mathcal{D}' :: \ChTy_2|-
        \eNu{(x_1, x_2)}{e} : V~\textrm{and}~E[\mathcal{D}'\sqsubseteq\mathcal{D}]}{By
        Lemma~\ref{lem:subterms}}

%      \Pf{}{}{c_2 \leadsto c_1~\text{s.t.}~\ChTy(c_2) =
%        \tyWr{S}~\textrm{and}~\ChTy(c_1) = \tyRd{S}}{Given}

      \Pf{\ChTy_2;\Gamma;\Delta}{|-}{e : U~\textrm{where}~\Gamma;\Delta=x_1: \tyRd{S} ; x_2 :
        \tyWr{S}}{By inversion on nu}

      \Pf{c_1 : \tyRd{S}}{|-}{c_1 : \tyRd{S}}{By $c_2 \leadsto c_1$}

      \Pf{c_2 : \tyWr{S}}{|-}{c_2 : \tyWr{S}}{By $c_2 \leadsto c_1$}

      \Pf{\ChTy_3}{|-}{[\eChan{c_1}/x_1][\eChan{c_2}/x_2]e :
        U~\textrm{where}~\ChTy_3=\ChTy_2,c_1:\tyRd{S},c_2:\tyRd{S}}{By
        Lemmas~\ref{lem:sub-rd} and~\ref{lem:sub-wr}}

      \Pf{\ChTy_1,\ChTy_3}{|-}{\proc{E[ [\eChan{c_1}/x_1][\eChan{c_2}/x_2]e ]}
        : U_p}{By Lemma~\ref{lem:replacement}}

      \Pf{\ChTy,\ChTy_4}{|-}{\proc{E[ [\eChan{c_1}/x_1][\eChan{c_2}/x_2]e ]} :
        U_p~\textrm{where}~\ChTy_4=c_1:\tyRd{S},c_2:\tyWr{S}}{By above
        equalities}

      \Pf{\ChTy,\ChTy_4}{|-}{\Config{\Names}{\Store}{\Procs} : \PrTy_{\pi}}{Above}

      \Pf{\ChTy,\ChTy_4}{|-}{\Config{\Names}{\Store}{\Procs, \ProcNm{p} \proc{E[
              [\eChan{c_1}/x_1][\eChan{c_2}/x_2]e ]}} : (\PrTy_{\pi}, p : U)}{By
        rule cons}

      \Pf{\ChTy,\ChTy_4}{|-}{\Config{\Names}{\Store}{\Procs, \ProcNm{p}
          \proc{E[ [\eChan{c_1}/x_1][\eChan{c_2}/x_2]e ]}} : \PrTy}{Above}      

      \Pf{}{}{\ChTy' = \ChTy,\ChTy_4~\textrm{and}~\PrTy' = \PrTy}{Suppose}

      \Pf{\ChTy'}{|-}{\Config{\Names}{\Store}{\Procs, \ProcNm{p}
          \proc{E[ [\eChan{c_1}/x_1][\eChan{c_2}/x_2]e ]}} : \PrTy'}{By above equalities}      
    \end{llproof}    
    
  \item[] \textbf{Case}
    \begin{mathpar}
    \Infer{rw}
    { c_2 \leadsto c_1 }
    { \Config{\Names}{\Store}{\Procs, \ProcNm{p} E_1[ \eLetRd{\eChan{c_1}}{x}{e} ], \ProcNm{q} E_2[ \eWr{v}{\eChan{c_2}}]} --->
      \Config{\Names}{\Store}{\Procs, \ProcNm{p} E_1[ [\ePair{!v}{\eChan{c_1}}{1}/x]e], \ProcNm{q}
        E_2[ \eUnit ]} }
    \end{mathpar}
    \begin{llproof}
      \Pf{\ChTy}{|-}{\Config{\Names}{\Store_1}{\Procs, \ProcNm{p} E_1[
              \eLetRd{\eChan{c_1}}{x}{e} ], \ProcNm{q} E_2[
            \eWr{v}{\eChan{c_2}}]} : \PrTy~\textrm{s.t.}~\PrTy=\PrTy_{\pi},p:U,q:V,}{}
      \Pf{}{}{\quad \ChTy =\ChTy_1,\ChTy_2, \mathcal{D}_p :: \ChTy_1,\ChTy_2|-
        \proc{E_1[ \eLetRd{\eChan{c_1}}{x}{e} ]} : U,}{}
      \Pf{}{}{\quad \ChTy =\ChTy_3,\ChTy_4,~\textrm{and}~\mathcal{D}_q ::
        \ChTy_3,\ChTy_4|- \proc{E_2[ \eWr{v}{\eChan{c_2}}]} : V}{Assumption}

      \Pf{}{}{\exists~\mathcal{D}_p'\sqsubseteq\mathcal{D}_p~\textrm{s.t.}~\mathcal{D}_p' :: \ChTy_2|-
        \eLetRd{\eChan{c_1}}{x}{e} : U'~\textrm{and}~E_1[\mathcal{D}_p'\sqsubseteq\mathcal{D}_p]}{By
        Lemma~\ref{lem:subterms}}

      \Pf{}{}{\exists~\mathcal{D}_q'\sqsubseteq\mathcal{D}_q~\textrm{s.t.}~\mathcal{D}_q' :: \ChTy_4|-
        \eWr{v}{\eChan{c_2}} : \tyUnit~\textrm{and}~E_2[\mathcal{D}_q'\sqsubseteq\mathcal{D}_q]}{By
        Lemma~\ref{lem:subterms}}      

      \Pf{}{}{c_2 \leadsto c_1~\text{s.t.}~\ChTy(c_2) =
        \tyWr{S}~\textrm{and}~\ChTy(c_1) = \tyRd{S}}{Given}

      \Pf{\ChTy_2;\Delta;\emptyctxt}{|-}{e : U'~\textrm{where}~\Delta=\wrtok,x :
        \tyTensor{\tyBang{S}}{\tyRd{S}}}{By inversion on rd}

      \Pf{}{|-}{v : S}{By inversion on wr}

      \Pf{}{|-}{\eBang{v} : \tyBang{S}}{By rule bang}

      \Pf{}{|-}{\ePair{!v}{\eChan{c_1}}{1} :
        \tyTensor{\tyBang{S}}{\tyRd{S}}}{By rule apair}

      \Pf{\ChTy_2;\wrtok;\emptyctxt}{|-}{[\ePair{!v}{\eChan{c_1}}{1}/x]e : U'}{By
          Lemma~\ref{lem:sub-rd}}

      \Pf{\ChTy_1,\ChTy_2}{|-}{E_1[ [\ePair{!v}{\eChan{c_1}}{1}/x]e] : U}{By
        Lemma~\ref{lem:replacement}}

      \Pf{\ChTy}{|-}{E_1[ [\ePair{!v}{\eChan{c_1}}{1}/x]e] : U}{By above equalities}      

      \Pf{\ChTy}{|-}{\Config{\Names}{\Store}{\Procs} : \PrTy_{\pi}}{Above}

      \Pf{\ChTy}{|-}{\Config{\Names}{\Store}{\Procs, \ProcNm{p} E_1[
            [\ePair{!v}{\eChan{c_1}}{1}/x]e]} : (\PrTy_{\pi}, p : U)}{By rule cons}

      \Pf{\ChTy_4}{|-}{\eUnit : \tyUnit }{By rule unit}

      \Pf{\ChTy_3,\ChTy_4}{|-}{E_2[ \eUnit ] : V}{By
        Lemma~\ref{lem:replacement}}

      \Pf{\ChTy}{|-}{E_2[ \eUnit ] : V}{By above equalities}      

      \Pf{\ChTy}{|-}{\Config{\Names}{\Store}{\Procs, \ProcNm{p} E_1[
            [\ePair{!v}{\eChan{c_1}}{1}/x]e], \ProcNm{q} E_2[ \eUnit ]} :
        (\PrTy_{\pi}, p:U, q:V)}{By rule cons}

      \Pf{\ChTy}{|-}{\Config{\Names}{\Store}{\Procs, \ProcNm{p} E_1[
            [\ePair{!v}{\eChan{c_1}}{1}/x]e], \ProcNm{q} E_2[ \eUnit ]} :
        \PrTy}{By above equalities}

      \Pf{}{}{\ChTy' = \ChTy~\textrm{and}~\PrTy' = \PrTy}{Suppose}      
      
      \Pf{\ChTy'}{|-}{\Config{\Names}{\Store}{\Procs, \ProcNm{p} E_1[ [\ePair{!v}{\eChan{c_1}}{1}/x]e], \ProcNm{q}
        E_2[ \eUnit ]} : \PrTy'}{By above equalities}
    \end{llproof}

  \item[] \textbf{Case}
    \begin{mathpar}
    \Infer{cw}
    { c \leadsto c_i \\ i \in \{1, 2\} }
    { \Config{\Names}{\Store}{\Procs, \ProcNm{p} E_1[\eChoicee{c_1}{x_1}{e_1}{c_2}{x_2}{e_2}], \ProcNm{q} E_2[ \eWr{v}{\eChan{c}}]} --->
      \Config{\Names}{\Store}{\Procs, \ProcNm{p} E_1[ [\ePair{!v}{c_i,
              c_{3-i}}{1}/x_i]e_{i}], \ProcNm{q} E_2[ \eUnit ]} }      
    \end{mathpar}
    \begin{llproof}
      \Pf{\ChTy}{|-}{\Config{\Names}{\Store_1}{\Procs, \ProcNm{p}
          E_1[\eChoicee{c_1}{x_1}{e_1}{c_2}{x_2}{e_2}], \ProcNm{q} E_2[
            \eWr{v}{\eChan{c}}]} : \PrTy}{}
      \Pf{}{}{\quad\textrm{s.t.}~\PrTy=\PrTy_{\pi},p:U,q:V,}{}
      \Pf{}{}{\quad \ChTy=\ChTy_1,\ChTy_2,\mathcal{D}_p :: \ChTy_1,\ChTy_2 |-
        \proc{E_1[\eChoicee{c_1}{x_1}{e_1}{c_2}{x_2}{e_2}]} : U,}{}
      \Pf{}{}{\quad \ChTy=\ChTy_3,\ChTy_4,~\textrm{and}~\mathcal{D}_q ::
        \ChTy_3,\ChTy_4|- \proc{E_2[ \eWr{v}{\eChan{c}}]} : V}{Assumption}

      \Pf{}{}{\exists~\mathcal{D}_p'\sqsubseteq\mathcal{D}_p~\textrm{s.t.}~\mathcal{D}_p' ::
        \ChTy_2|- \eChoicee{c_1}{x_1}{e_1}{c_2}{x_2}{e_2} :
        U'~\textrm{and}~E_1[\mathcal{D}_p'\sqsubseteq\mathcal{D}_p]}{By
        Lemma~\ref{lem:subterms}}

      \Pf{}{}{\exists~\mathcal{D}_q'\sqsubseteq\mathcal{D}_q~\textrm{s.t.}~\mathcal{D}_q' :: \ChTy_4|-
        \eWr{v}{\eChan{c_2}} : \tyUnit~\textrm{and}~E_2[\mathcal{D}_q'\sqsubseteq\mathcal{D}_q]}{By
        Lemma~\ref{lem:subterms}}
      
      \Pf{}{}{c \leadsto c_1~\text{s.t.}~\ChTy(c) =
        \tyWr{S},~\ChTy(c_1) = \tyRd{S},~\ChTy(c_2) = \tyRd{T}~\textrm{or}}{}
      \Pf{}{}{\quad c \leadsto c_2~\text{s.t.}~\ChTy(c) =
        \tyWr{T},~\ChTy(c_2) = \tyRd{T},~\ChTy(c_1) = \tyRd{S}}{Given}

      \Pf{}{}{\textbf{Subcase}~c \leadsto c_1}{}      

      \Pf{\ChTy_2;\Delta;\emptyctxt}{|-}{e : U'~\textrm{where}~\Delta = \wrtok, x_1 :
        \tyTensor{\tyBang{S}}{\tyTensor{\tyRd{S}}{\tyRd{T}}}}{By
        inversion on choice}

      \Pf{}{|-}{v : S}{By inversion on wr}

      \Pf{}{|-}{\eBang{v} : \tyBang{S}}{By rule bang}

      \Pf{}{|-}{\ePair{!v}{\eChan{c_1},\eChan{c_2}}{1} :
        \tyTensor{\tyBang{S}}{\tyTensor{\tyRd{S}}{\tyRd{T}}}}{By rule apair}

      \Pf{\ChTy_2;\wrtok;\emptyctxt}{|-}{\ePair{!v}{\eChan{c_1},\eChan{c_2}}{1}/x_1]e_1
      : U'}{By Lemma~\ref{lem:sub-rd}}

      \Pf{\ChTy_1,\ChTy_2}{|-}{E_1[
          [\ePair{!v}{\eChan{c_1},\eChan{c_2}}{1}/x_1]e_1] : U}{By
        Lemma~\ref{lem:replacement}}

      \Pf{\ChTy}{|-}{E_1[
          [\ePair{!v}{\eChan{c_1},\eChan{c_2}}{1}/x_1]e_1] : U}{By above equalities}      

      \Pf{\ChTy}{|-}{\Config{\Names}{\Store}{\Procs} : \PrTy_{\pi}}{Above}

      \Pf{\ChTy}{|-}{\Config{\Names}{\Store}{\Procs, \ProcNm{p} E_1[
            [\ePair{!v}{\eChan{c_1},\eChan{c_2}}{1}/x_1]e_1]} : (\PrTy_{\pi}, p :
        U)}{By rule cons}

      \Pf{\ChTy_4}{|-}{\eUnit : \tyUnit }{By rule unit}

      \Pf{\ChTy_3,\ChTy_4}{|-}{E_2[ \eUnit ] : V}{By
        Lemma~\ref{lem:replacement}}

      \Pf{\ChTy}{|-}{E_2[ \eUnit ] : V}{By above equalities}      

      \Pf{\ChTy}{|-}{\Config{\Names}{\Store}{\Procs, \ProcNm{p} E_1[
            [\ePair{!v}{\eChan{c_1},\eChan{c_2}}{1}/x_1]e_1], \ProcNm{q} E_2[
            \eUnit ]} : (\PrTy_{\pi}, p : U, q : V)}{By rule cons}

      \Pf{\ChTy}{|-}{\Config{\Names}{\Store}{\Procs, \ProcNm{p} E_1[
            [\ePair{!v}{\eChan{c_1},\eChan{c_2}}{1}/x_1]e_1], \ProcNm{q} E_2[
            \eUnit ]} : \PrTy}{By above equalities}            

      \Pf{}{}{\ChTy' = \ChTy~\textrm{and}~\PrTy' = \PrTy}{Suppose}      
      
      \Pf{\ChTy'}{|-}{\Config{\Names}{\Store}{\Procs, \ProcNm{p} E_1[
            [\ePair{!v}{\eChan{c_1},\eChan{c_2}}{1}/x_1]e_1], \ProcNm{q} E_2[
            \eUnit ]} : \PrTy'}{By above equalities}

      \Pf{}{}{\textbf{Subcase}~c \leadsto c_2}{}      

      \Pf{\ChTy_2;\Delta;\emptyctxt}{|-}{e : U'~\textrm{where}~\Delta = \wrtok, x_2 :
        \tyTensor{\tyBang{T}}{\tyTensor{\tyRd{T}}{\tyRd{S}}}}{By inversion on
        choice}

      \Pf{}{|-}{v : T}{By inversion on wr}

      \Pf{}{|-}{\eBang{v} : \tyBang{T}}{By rule bang}

      \Pf{}{|-}{\ePair{!v}{\eChan{c_2},\eChan{c_1}}{1} :
        \tyTensor{\tyBang{T}}{\tyTensor{\tyRd{T}}{\tyRd{S}}}}{By rule apair}

      \Pf{\ChTy_2,\wrtok;\emptyctxt}{|-}{\ePair{!v}{\eChan{c_2},\eChan{c_1}}{1}/x_2]e_2
      : U'}{By Lemma~\ref{lem:sub-rd}}

      \Pf{\ChTy_1,\ChTy_2}{|-}{E_1[ [\ePair{!v}{\eChan{c_2},\eChan{c_1}}{1}/x_2]e_2] : U}{By
        Lemma~\ref{lem:replacement}}

      \Pf{\ChTy}{|-}{E_1[ [\ePair{!v}{\eChan{c_2},\eChan{c_1}}{1}/x_2]e_2] : U}{By
        above equalities}      

      \Pf{\ChTy}{|-}{\Config{\Names}{\Store}{\Procs} : \PrTy_{\pi}}{Above}

      \Pf{\ChTy}{|-}{\Config{\Names}{\Store}{\Procs, \ProcNm{p} E_1[
            [\ePair{!v}{\eChan{c_2},\eChan{c_1}}{1}/x_2]e_2]} : (\PrTy_{\pi}, p :
        U)}{By rule cons}

      \Pf{\ChTy_4}{|-}{\eUnit : \tyUnit }{By rule unit}

      \Pf{\ChTy_3,\ChTy_4}{|-}{E_2[ \eUnit ] : V}{By
        Lemma~\ref{lem:replacement}}

      \Pf{\ChTy}{|-}{E_2[ \eUnit ] : V}{By above equalities}

      \Pf{\ChTy}{|-}{\Config{\Names}{\Store}{\Procs, \ProcNm{p} E_1[
            [\ePair{!v}{\eChan{c_2},\eChan{c_1}}{1}/x_2]e_2], \ProcNm{q} E_2[
            \eUnit ]} : (\PrTy_{\pi}, p : U, q : V)}{By rule cons}

      \Pf{\ChTy}{|-}{\Config{\Names}{\Store}{\Procs, \ProcNm{p} E_1[
            [\ePair{!v}{\eChan{c_2},\eChan{c_1}}{1}/x_2]e_2], \ProcNm{q} E_2[
            \eUnit ]} : \PrTy}{By above equalities}      

      \Pf{}{}{\ChTy' = \ChTy~\textrm{and}~\PrTy' = \PrTy}{Suppose}      
      
      \Pf{\ChTy'}{|-}{\Config{\Names}{\Store}{\Procs, \ProcNm{p} E_1[
            [\ePair{!v}{\eChan{c_2},\eChan{c_1}}{1}/x_2]e_2], \ProcNm{q} E_2[
            \eUnit ]} : \PrTy'}{By above equalities}
    \end{llproof}    
  \end{itemize}
\end{proof}
\end{theorem}

\section{Confluence}

The following lemmas state structural invariants over write effects and read
channels of a well-typed configuration: at most one process owns the write token
$\wrtok$, and every read channel end is a non-duplicable (affine) resource.

%\begin{lemma}[Unique writer process]
%\label{lem:UniqueWriter}
%If $C$ is a well-typed configuration with process pool~$\pi$, 
%then there exists at most one write-mode process in $\pi$.
%\begin{proof}
%By structural induction over the typing derivation for $C$.
%\end{proof}
%\end{lemma}

\begin{lemma}[Unique writer process]
\label{lem:UniqueWriter}
If $C$ is a well-typed configuration with process pool~$\pi$, then there exists at
most one process in $\pi$ that owns the write token $\wrtok$ (i.e., has $\wrtok$
in its affine context).
\begin{proof}
By structural induction over the typing derivation for $C$.
\end{proof}
\end{lemma}

\begin{lemma}[Unique reader process]
\label{lem:UniqueReader}
If $C$ is a well-typed configuration with process pool~$\pi$, 
and $c$ is a read channel in this configuration,
then there exists at most one process in $\pi$ where $c$ appears.
\begin{proof}
By structural induction over the typing derivation for $C$.
\end{proof}
\end{lemma}

\begin{theorem}[Single-step confluence]\label{lem:single-step-confluence}
For all well-typed configurations $C$,
%
 if $\JCred{C}{C_1}$ 
and $\JCred{C}{C_2}$ 
then 
there exists renaming a function~$f$ 
such that either:
\begin{enumerate}
\item %$\JCterm{C_1}$ and 
$C_1 = f(C_2)$,
or
\item there exists $C_3$ such that $\JCred{C_1}{C_3}$ and $\JCred{f(C_2)}{C_3}$.
\end{enumerate}
\begin{proof}
   By induction on the pair of steps 
   $\left< \JCred{C}{C_1}\right.$, 
   $\left.\JCred{C}{C_2} \right>$.

   We consider the following cases:
   \begin{itemize}[leftmargin=*]
   \item[] \textbf{Case} congruence
     \begin{itemize}[leftmargin=*]
       \item[] If either step uses \Rule{congr}, we apply
     the inductive hypothesis.
     \end{itemize}     

   \item[] \textbf{Case} independent processes
     \begin{itemize}[leftmargin=*]
       \item[] If both steps advance distinct processes, using any of the rules
         \Rule{local}, \Rule{fork} and \Rule{nu}, we produce $C_3$ by combining
         those two (independent) steps.
     \end{itemize}


   \item[] \textbf{Case} one process
     \begin{itemize}[leftmargin=*]
       \item[] If both steps advance the same process, we show that this is
         deterministic (up to naming) by constructing the naming function $f$
         such that $C_2 = f(C_1)$.  Most cases are straightforward since they
         perform no nondeterministic choices.  The only source of nondeterminism
         is the name choices, in rules \Rule{nu} and \Rule{fork}. In each case,
         we map the name choice from the second step to that of the first step.
     \end{itemize}

   \item[] \textbf{Case} interaction
     \begin{itemize}[leftmargin=*]
       \item[] If either step uses \Rule{rw} or \Rule{cw}, we rely on Lemmas
         \ref{lem:UniqueWriter} and \ref{lem:UniqueReader} to show that both
         steps use either \Rule{rw} or \Rule{cw}, and that the reader-writer
         process pair is unique.
     \end{itemize}
   \end{itemize}
\end{proof}
\end{theorem}

By composing multiple uses of this theorem
we prove multi-step confluence.
However, to carry forth this composition, we need a more general
notion of single-step confluence, which is parameteric in a renaming
function for the initial configurations.

\begin{theorem}[Single-step confluence, generalized]
For all well-typed configurations $C$ 
and renaming functions $f$,
%
 if $\JCred{C}{C_1}$ 
and $\JCred{f(C)}{C_2}$ 
then 
there exists renaming function~$g$ 
such that either:
\begin{enumerate}
\item %$\JCterm{C_1}$ and 
$C_1 = g(C_2)$,
or
\item there exists $C_3$ such that $\JCred{C_1}{C_3}$ and $\JCred{g(C_2)}{C_3}$.
\end{enumerate}
\begin{proof}
  Analogous to the proof of Theorem~\ref{lem:single-step-confluence}
  (single-step confluence).
\end{proof}
\end{theorem}

We prove a full confluence theorem that is generalized similarly, by
accepting a renaming function~$f$ to produce a new function~$g$:

\begin{theorem}[Full confluence]
For all well-typed configurations $C$,
and renaming functions $f$,
%
 if $\JCredm{C}{C_1}$ 
and $\JCredm{f(C)}{C_2}$ 
and $\JCterm{C_1}$
and $\JCterm{C_2}$
then 
there exists a renaming function~$g$ 
such that $C_1 = g(C_2)$.
\begin{proof}
  By induction on the reduction sequence pair
  $\left< \JCredm{C}{C_1}\right.$, 
  $\left.\JCredm{f(C)}{C_2} \right>$.
  Because of single-step confluence, we know that
  if either reduction sequence is empty, then the other must be empty,
  and that
  if either takes a step, the other must take a step.

   \begin{itemize}[leftmargin=*]
   \item[] \textbf{Case} empty
     \begin{itemize}[leftmargin=*]
       \item[] When empty, we have the resulting renaming function~$g$ via
         single-step confluence.
     \end{itemize}

   \item[] \textbf{Case} step
     \begin{itemize}[leftmargin=*]
       \item[] We consider the case where each reduction consists of at least
         one step: $\JCred{C}{C_1'}$ and $\JCredm{C_1'}{C_1}$ and
         $\JCred{f(C)}{C_2'}$ and $\JCredm{C_2'}{C_2}$.  By single-step
         confluence, we have that there exists $g_0$ such that $g_0(C_2') =
         C_1'$.  By the inductive hypothesis, we have that there exists $g$ such
         that $C_1 = g(C_2)$.
     \end{itemize}     
   \end{itemize}
\end{proof}
\end{theorem}

