\section{Verifying Properties Using Session Types}
\label{sec:session}

\begin{figure*}
  \begin{grammar}
    Intuitionistic Types
    & $A,B$
    &$\bnfas$& $\tyUnit \bnfalt \tyProd{A}{B} \bnfalt \tySum{A}{B} \bnfalt
    \tyWr{A} \bnfalt \tyRef{A} \bnfalt \tyArr{A}{\footnotesize $m$}{B}$
    \\
    Linear Types
    & $X,Y$
    &$\bnfas$& $\tyRd{A} \bnfalt !\,A \bnfalt X \otimes Y \bnfalt X
    \lollipop{\footnotesize m} Y$
    \\
    Intuitionistic Typing Contexts
    & $\Gamma$
    &$\bnfas$& $\emptyctxt \bnfalt \Gamma,x:A$
    \\
    Linear Typing Contexts
    & $\Delta$
    &$\bnfas$& $\emptyctxt \bnfalt \Delta,x:A$
  \end{grammar}
\caption{Syntax of types.}
\label{fig:syntax--types}
\end{figure*}


\begin{figure*}[t]
  \begin{grammar}
    All types
    & $U,V$
    &$\bnfas$& $A \bnfalt X$
    \\
    Intuitionistic types
    & $A,B$
    &$\bnfas$& $\tyUnit \bnfalt \tyProd{A}{B} \bnfalt \tySum{A}{B} \bnfalt
    \tyWr{A} \bnfalt \tyArr{A}{\footnotesize m}{U}$
    \\
    Affine types
    & $X,Y$
    &$\bnfas$& $\tyRd{A} \bnfalt \tyBang{A} \bnfalt \tyTensor{X}{Y} \bnfalt X \oplus
    Y \bnfalt \tyLolli{X}{\footnotesize m}{U}$
  \end{grammar}
  \vspace{-.1cm}
  \begin{minipage}{.4\textwidth}
    \begin{grammar}
      Modes & $m,n,p$ &$\bnfas$& $\Wm \bnfalt \Rm \bnfalt \Vm$
      \\
      Multiplicity & $\pi$ &$\bnfas$& $1 \bnfalt \infty$      
    \end{grammar}
  \end{minipage}%
  \begin{minipage}{.4\textwidth}
    \begin{grammar}
      Intuitionistic typings
      & $\Gamma$
      &$\bnfas$& $\emptyctxt \bnfalt \Gamma,x:A$
      \\
      Affine typings
      & $\Delta$
      &$\bnfas$& $\emptyctxt \bnfalt \Delta,x:X$
    \end{grammar}    
  \end{minipage}
  \begin{grammar}
    Expressions
    & $e$
        &$\bnfas$&
        $\eVar{x} \bnfalt \eUnit \bnfalt \ePair{e_1}{e_2}{\pi} \bnfalt \eInj{\pi}{i}{e} \bnfalt \eSplit{\pi}{e_1}{x_1}{x_2}{e_2} \bnfalt
    \eCase{\pi}{e}{x_1}{e_1}{x_2}{e_2}$
    \\ &&& $\bnfaltbrk \eFix{\pi}{x}{e}
    \bnfalt \eLet{x}{e_1}{e_2} \bnfalt \eLetBang{x}{e_1}{e_2} \bnfalt \eBang{e}$
    \\ &&& $\bnfaltbrk \eLam{x}{e} \bnfalt \eApp{e_1}{e_2} \bnfalt \eLAM{x}{e}
    \bnfalt \eLapp{e_1}{e_2}$
    \\ &&& $\bnfaltbrk \eNu{(x_1,x_2)}{e} \bnfalt \eWr{e_1}{e_2}
    \bnfalt \eLetRd{x_1}{x_2}{e_1}{e_2} \bnfalt \eFork{e_1}{e_2} \bnfalt \eChoice{e_1}{e_2}$
  \end{grammar}
  \caption{Syntax of ILC.}
  \label{fig:ilc-syntax}
\end{figure*}


\textbf{Notes on multiparty session types:}

The work by Honda \etal~\cite{honda2008multiparty} introduces multiparty
asynchronous session types, a typed calculus for mobile processes. Multiparty
session types extend binary session types, which can be used to represent a
conversation structure among two-party protocols, to arbitrarily many
parties. To achieve this, multiparty session types directly abstract a
conversation structure among $n$-parties as a \emph{global type}, which can be
mapped, or \emph{projected}, onto each party's local view of the conversation
structure called its \emph{local type}. Then, the code of each party is checked
for conformance with its local type by efficient type-checking.

Differences between multiparty session types and our work:
\begin{enumerate}
  \item \emph{Preventing race conditions at channels.} Multiparty session types
    require that when a common channel is used in two communications, the
    sending actions and the receiving actions should respectively be ordered
    temporally (through causality), so that no confusion arises. Since the
    execution of a system of ITMs is essentially single-threaded, we can avoid
    race conditions by disallowing parallel writes and requiring that read
    channels are linearly typed.
  \item \emph{Defining asynchrony.} In their process calculus, a sending action
    is never blocked (asynchrony), and two messages sent from the same sender to
    the same channel must arrive in the sending order (order
    preservation). Arbitrary reordering of messages can be simulated by
    including an adversary process that permutes messages locally.
\end{enumerate}

\begingroup
\setlength\intextsep{0pt}
\begin{wrapfigure}{r}{0.3\textwidth}
  \lstinputlisting[style=myilc]{listings/session_counterexample.ilc}
\end{wrapfigure}

Why we use session types as effects: A common pattern in ideal functionalities
is to read on some channel, and then write to one of several distinct channels
based on the read value. Consider the expression to the right, which first reads
from a channel \textsf{c}, binding the read value to \textsf{v} and rebinding
the channel to \textsf{a}. Then, depending on the value \textsf{v}, the process
writes an integer on either channel \textsf{a} or channel \textsf{b}. The
session type of each match expression branch would be
$\mathsf{a}!\langle\mathsf{int}\rangle$ and $\mathsf{b}!\langle\mathsf{int}\rangle$, which is not
typeable. This is a common pattern and it would be useful to be able to express
this naturally. In ILC, the expression is well-typed: each branch has type
$\tyUnit |> \Wm$.\mypar \endgroup

\todo{Three types annotations: optionally occur, eventually occur, constrained
  occur.}

\subsection{Examples}
\label{subsec:examples}

In certain cases, we may not be able to prove that an event eventually happens
with absolute certainty, but it suffices to prove that it happens with
non-negligible probability. In the example below, let \textsf{ed\_loop} be the
same looping construct defined before, except the function argument has the
refinement $\lozenge \mathsf{W@a2r}$, i.e., eventually channel \textsf{a2r} is written
to. Process \textsf{A} runs \textsf{ed\_loop} with read channel \textsf{z2a} and
the function \textsf{f}, which writes the unit value to \textsf{a2z} if the value read
on \textsf{z2a} contains a 0 and writes a string to \textsf{a2r} if the value
read on \textsf{z2a} contains a 1. Process \textsf{Z} begins by writing to
\textsf{A}, which will send back the unit value. Then, \textsf{Z} either sends
back $(\mathsf{'Tick}, 0)$ or $(\mathsf{'Tick}, 1)$ depending on the value
returned by \textsf{rand\_bit}. Although writing $(\mathsf{'Tick}, 1)$ is not
guaranteed to happen, so it cannot satisfy the refinement, it happens with
exponentially increasing probability.

\lstinputlisting[style=myilc]{listings/effects.ilc}

%\lstinputlisting[style=myilc]{listings/constraints.ilc}

%\lstinputlisting[style=myilc]{listings/responsive.ilc}

%\lstinputlisting[style=myilc]{listings/sesh_channels.ilc}
