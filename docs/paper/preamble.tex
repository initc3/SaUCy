%\usepackage{algorithm}
\usepackage[linesnumbered,algoruled,boxed,noline]{algorithm2e}
\usepackage[noend]{algpseudocode}
\usepackage{amsfonts}
\usepackage{amsmath}
\usepackage{amssymb}
\usepackage{amsthm}
\usepackage{bbm}
\usepackage{boxedminipage}
\usepackage{color}
\usepackage{xcolor}
\usepackage{comment}
\usepackage{enumitem}
\usepackage{joshuadunfield}
\usepackage{lipsum}
\usepackage{mathpartir}
\usepackage{mathtools}
\usepackage{semantic}
\usepackage{slashed}
\usepackage{soul}
\usepackage{subcaption}
\usepackage[most]{tcolorbox}
\usepackage{upquote}
\usepackage{wrapfig}
\usepackage{xspace}

\makeatletter
\newcommand{\oset}[3][0ex]{%
  \mathrel{\mathop{#3}\limits^{
    \vbox to#1{\kern-0.75\ex@
    \hbox{$\scriptstyle#2$}\vss}}}}
\makeatother

\definecolor{Green}{rgb}{0.13, 0.55, 0.13}
\definecolor{Purple}{rgb}{0.6, 0.2, 0.8}

\newcommand{\mc}[1]{\mathcal{{#1}}}
\newcommand{\todo}[1]{\emph{\hl{TODO:} {#1}}}
\newcommand{\saucy}{SaUCy\xspace}
\newcommand{\fstar}{$\text{F}^{\star}$}
\newcommand\myeq{\stackrel{\mathclap{\scriptsize\mbox{def}}}{=}}
\newcommand{\mypar}{\par}
\newcommand{\concat}{%
  \mathbin{{+}\mspace{-8mu}{+}}%
}
\newcommand\doubleplus{+\kern-1.3ex+\kern0.8ex}
\newcommand\mdoubleplus{\ensuremath{\mathbin{+\mkern-10mu+}}}
\newcommand{\yrightarrow}[1]{\xrightarrow{\raisebox{-2.5pt}[0pt][0pt]{#1}}}
\newcommand{\lollipop}[1]{\oset{{\rule{0ex}{0ex}\mkern-3mu{#1}}}{\lolli}}
\newcommand{\lollipopp}[1]{\oset{{\rule{0ex}{0ex}\mkern-3mu{#1}}}{\lolli}}
\newcommand*{\lolli}{%
  {\text{--}\mkern-2.4mu\text{---}\mkern-2.5mu\circ}% subformula acts as \mathord
}
\newcommand{\myheader}[1]{\noindent{\textit{\textbf{#1}}}}
\newcommand{\myheaderi}[1]{\textit{\textbf{#1}}}

% Functionality and ILC boxes
\newcommand{\Func}{\mc{F}}

\newenvironment{sketch}{%
  \renewcommand{\proofname}{Proof Sketch}\proof}{\endproof}

\newtcolorbox{func}[1][]{title={\textbf{Functionality}~{#1}},enhanced,
  coltitle=black,
  top=0.2in,
  attach boxed title to top left=
  {xshift=0.5cm,yshift=-\tcboxedtitleheight/2},
  boxed title style={size=small, colback=white},colback=white}

\lstdefinestyle{myilc}
{
    language=Caml,
    keywordstyle={\bfseries},
    morekeywords={let, letrec, in, wr, rd, match, with, set},
    basicstyle={\sffamily},
    captionpos=b,
    columns=fullflexible,
    upquote = true,
    mathescape=true,
    showstringspaces=false,
}

\newcommand{\mywr}[2]{\mathsf{w{#1}2{#2}}}

\theoremstyle{definition}
\newtheorem{definition}{Definition}[section]

% ILC Expressions
\newcommand{\eVar}[1]{#1}
\newcommand{\eChan}[1]{#1}
\newcommand{\eUnit}{\Lparen\hspace{1px}\Rparen}
\newcommand{\ePair}[3]{{\Lparen{#1},{#2}\Rparen}_{#3}}
\newcommand{\eInj}[3]{\keyword{inj}_{#1}^{#2}\Lparen#3\Rparen}
\newcommand{\eRef}[1]{\keyword{ref}\Lparen{#1}\Rparen}
\newcommand{\eCase}[6]{\keyword{case}_{#1}\Lparen{#2},{#3}.{#4},{#5}.{#6}\Rparen}
\newcommand{\eSplit}[5]{\keyword{split}_{#1}\Lparen{#2},{#3}.{#4}.{#5}\Rparen}
\newcommand{\eGet}[1]{\keyword{get}\Lparen{#1}\Rparen}
\newcommand{\eSet}[2]{\keyword{set}\Lparen{#1},{#2}\Rparen}
\newcommand{\eFix}[3]{\keyword{fix}_{#1}\Lparen{#2}.{#3}\Rparen}
\newcommand{\eLet}[4]{\keyword{let}_{#1}\ {#2} = {#3}\ \keyword{in}\ {#4}}
\newcommand{\eLetRd}[4]{\eLet{}{\langle{#1},{#2}\rangle}{\eRd{#3}}{{#4}}}
\newcommand{\eLetBang}[3]{\keyword{let!}\ {#1} = {#2}\ \keyword{in}\ {#3}}
\newcommand{\eLam}[3]{\lambda_{#1}{#2}.\,{#3}}
\newcommand{\eApp}[3]{{\Lparen#1\,#2\Rparen}_{#3}}
\newcommand{\eNu}[2]{\nu#1.\,#2}
\newcommand{\eWr}[2]{\keyword{wr}\Lparen{#1},{#2}\Rparen}
\newcommand{\eRd}[1]{\keyword{rd}\Lparen{#1}\Rparen}
\newcommand{\eFork}[2]{\ensuremath{#1 \xFork #2}}
\newcommand{\eChoice}[2]{\ensuremath{#1 \xChoice #2}}
\newcommand{\eSeq}[2]{{#1}\,;\,{#2}}

\newcommand{\eBang}[1]{{!}\,{#1}}
\newcommand{\eUnbang}[1]{\textnormal{!`}\,{#1}}

% Extended ILC
\newcommand{\eKGen}[1]{\keyword{kgen}\Lparen{#1}\Rparen}
\newcommand{\eTdp}[2]{\keyword{tdp}\Lparen{#1},{#2}\Rparen}
\newcommand{\eInv}[2]{\keyword{inv}\Lparen{#1},{#2}\Rparen}
\newcommand{\eHc}[1]{\keyword{hc}\Lparen{#1}\Rparen}

% ILC Types
\newcommand{\tyNat}{\tyname{Nat}}
\newcommand{\tyBool}{\tyname{Bool}}
%\newcommand{\tyUnit}{\keyword{Unit}}
\newcommand{\tyUnit}{\mathbbm{1}}
\newcommand{\tyProd}[2]{{#1} ** {#2}}
\newcommand{\tySum}[2]{{#1} + {#2}}
\newcommand{\tyRd}[1]{\keyword{Rd}\,#1}
\newcommand{\tyWr}[1]{\keyword{Wr}\,#1}
\newcommand{\tyRef}[1]{\keyword{Ref}\,#1}
\newcommand{\tyArr}[3]{{#1}\rightarrow_{#2}{#3}}
\newcommand{\emptyctxt}{\cdot}
\newcommand{\tyBang}[1]{{!}\,{#1}}
\newcommand{\tyTensor}[2]{{#1}\otimes{#2}}
%\newcommand{\tyLolli}[3]{{#1}\lollipop{#2}{#3}}
\newcommand{\tyLolli}[3]{{#1}\multimap_{#2}{#3}}

\newcommand{\tyBit}{\keyword{Bit}}

% ILC Modes
\newcommand{\Rm}{\textsf{R}}
\newcommand{\Wm}{\textsf{W}}
\newcommand{\Vm}{\textsf{V}}

% ILC Semantics
%\newcommand{\Chans}{\Sigma}
%\newcommand{\emptyChans}{\varepsilon}
\newcommand{\Names}{\Sigma}
\newcommand{\emptyNames}{\varepsilon}
\newcommand{\Store}{\sigma}
\newcommand{\Procs}{\pi}
\newcommand{\ProcNm}[1]{#1{:}}
\newcommand{\proc}[1]{#1}
\newcommand{\emptyProcs}{\varepsilon}
\newcommand{\Config}[3]{\left<#1;#3\right>}

% ILC Misc
\newcommand{\xFork}{\mathrel{|\rhd}}
\newcommand{\xChoice}{\mathrel{\oplus}}
\newcommand{\e}{\epsilon}
\newcommand{\Lparen}{\textsf{(}}
\newcommand{\Rparen}{\textsf{)}}

%% Math ligatures (thanks to the semantic package) that make it
%% easier to typeset math using readable LaTeX text.
%\mathlig{|-->}{\longmapsto}
\mathlig{::=}{\bnfas}
\mathlig{:=}{\coloneqq}
\mathlig{|}{\;|\;}
% \mathlig{[[}{\mbsf{[}}
% \mathlig{]]}{\mbsf{]}}
\mathlig{[[}{\textsf{\upshape[}}
\mathlig{]]}{\textsf{\upshape]}}
\mathlig{**}{\times}
\mathlig{|>}{\rhd}
\mathlig{->}{\arr}
\mathlig{-->}{\rightarrow}
\mathlig{--->}{\longrightarrow}
\mathlig{=>}{\Rightarrow}
\mathlig{*!}{\boldsf{!}}
\mathlig{||}{\mathrel{|\!|}}
\mathlig{;;}{\mathrel{;}}
\mathlig{*&&}{\mathrel{\xFork}}
\mathlig{*||}{\mathrel{\oplus}}
\mathlig{!!}{\Downarrow}


% SaUCy specific
\newcommand{\msf}[1]{\ensuremath{{\mathsf {#1}}}}
\newcommand{\mtt}[1]{\ensuremath{\mathtt {#1}}}
\newcommand{\mcal}[1]{\ensuremath{\mathcal {#1}}}
\newcommand{\val}{\msf{val}}
\newcommand{\sn}{\msf{sn}}
\newcommand{\tn}{\textnormal}
\newcommand{\codeb}[1]{\textsf{#1}}
\newcommand{\hash}{\ensuremath{\mathcal{H}}}
\newcommand{\adv}{\ensuremath{{\mathcal A}}\xspace}
\newcommand{\Adv}{\adv}
\newcommand{\A}{\adv}
\newcommand{\samples}{\overset{\$}{\leftarrow}}
\newcommand{\SA}{\msf{SA}}
\newcommand{\SaUCy}{\xspace{\msf{SaUCy}}\xspace}
\newcommand{\leak}{\color{blue}\mtt{leak}}
\newcommand{\eventually}[1]{\color{blue}\mtt{leak}}
\newcommand{\poly}{\textnormal{poly}}
\newcommand{\negl}{\textnormal{negl}}
\newcommand{\execUC}[4]{\mathsf{execUC}({#1},{#2},{#3},{#4})}
\newcommand{\chan}[1]{{\underline{\smash{\msf{#1}}}}}
\newcommand{\functionality}[1]{\ensuremath{\mathcal{F}_{\textnormal{\msf{{#1}}}}}}
\newcommand{\F}{\functionality}
%\newcommand{\G}[1]{\mathcal{G}_{\textnormal{\tiny {\uppercase{#1}}}}}
\newcommand{\GG}[1]{{\overline \mathcal{G}}_{\textnormal{\tiny {\uppercase{#1}}}}}
\renewcommand{\C}[1]{\mathcal{C}_{\textnormal{\tiny {\uppercase{#1}}}}}
\renewcommand{\P}{\ensuremath{\mathcal P}}
\newcommand{\Ps}{\ensuremath{\{\mathcal{P}_i\}_{i \in [N]}}}
\renewcommand{\S}{\ensuremath{\mathcal S}}
\newcommand{\env}{\Z}
\newcommand{\Z}{{\mcal{E}}}
\newcommand{\R}{{\mathcal R}}


%% ILC judgement forms
%% ---------------------------------------------------------------------
%% Used to write the line-by-line reasoning of inductive proofs.

%% Channel typing meta var: each one maps channel names to types
\newcommand{\ChTy}{\ensuremath{\Psi}}

%% Process typing meta var: each one maps process names to types
\newcommand{\PrTy}{\ensuremath{\Phi}}

%% Process typing meta var: each one maps process names to types
\newcommand{\StTy}{\ensuremath{\Omega}}

%% Affine typing judgement
\newcommand{\JtyAff}[6]{#1 ; #2 ; #3 |- #4 : #5 |> #6}

%% Intuitionisitic typing judgement
\newcommand{\JtyInt}[6]{#1 ; #2 ; #3 |- #4 : #5 |> #6}

%% Configuration typing judgement: Each process has a mode, with at most one write-mode process, and each process has a return type, 
%% captured in the final mapping (#3).
\newcommand{\JCty}[4]{#2 |- #3 : #4}

%% Configurations~$C_1$ and $C_2$ are equivalent.
\newcommand{\JCequiv}[2]{#1 \equiv #2}

%% Configuration~$C_1$ reduces to $C_2$. single step.
\newcommand{\JCred}[2]{#1 ---> #2}

%% Configuration~$C_1$ reduces to $C_2$. multiple steps.
\newcommand{\JCredm}[2]{#1 --->^\ast #2}

%% Configuration has terminated, and all processes consist of their final values.
\newcommand{\JCterm}[1]{#1~\textbf{term}}

% Under store $\Store_1$, expression~$e_1$ reduces to~$\Store_2 ; e_2$.
\newcommand{\Jered}[4]{#1 ; #2 ---> #3 ; #4}

\newcommand{\Rule}[1]{\textrm{#1}}
