\usepackage{algorithm}
\usepackage[noend]{algpseudocode}
\usepackage{amsfonts}
\usepackage{amsmath}
\usepackage{amssymb}
\usepackage{amsthm}
\usepackage{boxedminipage}
\usepackage{color}
\usepackage{comment}
\usepackage{enumitem}
\usepackage{joshuadunfield}
\usepackage{lipsum}
\usepackage{mathpartir}
\usepackage{mathtools}
\usepackage{semantic}
\usepackage{soul}
\usepackage{subcaption}
\usepackage[most]{tcolorbox}
\usepackage{upquote}
\usepackage{wrapfig}
\usepackage{xspace}

% Commands
\newcommand{\mc}[1]{\mathcal{{#1}}}
\newcommand{\myheader}[1]{\noindent\textbf{#1}}
\newcommand{\myheaderi}[1]{\textbf{#1}}
\newcommand{\todo}[1]{\emph{\hl{TODO:} {#1}}}
\newcommand{\saucy}{SaUCy\xspace}
\newcommand{\package}{Pack\xspace}
\newcommand{\fstar}{$\text{F}^{\star}$}
\newcommand\myeq{\stackrel{\mathclap{\scriptsize\mbox{def}}}{=}}
\newcommand{\mypar}{\par}
\newcommand\doubleplus{+\kern-1.3ex+\kern0.8ex}

% Smart contracts
\newcommand{\money}[1]{\xspace{\${#1}}}
\newcommand{\timeval}[1]{\textbf{\texttt{#1}}}
\newcommand{\Xmoney}{{\textnormal{\money{\ensuremath{X}}}}}
\def\BState{\State\hskip-\ALG@thistlm}
\algdef{SE}[SUBALG]{Indent}{EndIndent}{}{\algorithmicend\ }%
\algtext*{Indent}
\algtext*{EndIndent}

% Functionality and ILC boxes
\newcommand{\Func}{\mc{F}}

\newtcolorbox{func}[1][]{title={\textbf{Functionality}~{#1}},enhanced,%drop shadow={black!50!white},
  coltitle=black,
  top=0.2in,
  attach boxed title to top left=
  {xshift=0.5cm,yshift=-\tcboxedtitleheight/2},
  boxed title style={size=small, colback=white},colback=white}

\newtcolorbox{ilc}[1][]{title={\textbf{ILC}~$\Func_{\textsc{#1}}$},enhanced,%drop shadow={black!50!white},
  coltitle=black,
  top=0.1in,
  attach boxed title to top left=
  {xshift=0.5cm,yshift=-\tcboxedtitleheight/2},
  boxed title style={size=small,colback=red!25},colback=white}

\lstdefinestyle{myilc}
{
    language=Caml,
    keywordstyle={\bfseries},
    morekeywords={let, letrec, in, wr, rd, match, with},
    basicstyle={\sffamily},
    captionpos=b,
    columns=fullflexible,
    upquote = true,
    mathescape=true,
    showstringspaces=false,
}

\theoremstyle{definition}
\newtheorem{definition}{Definition}[section]

% ILC


\newcommand{\runonboldsf}{\sffamily\bfseries\selectfont}
\newcommand{\boldsf}[1]{\text{\runonboldsf #1}}

% ILC Values
\newcommand{\vUnit}{\textsf{()}}

% ILC Types
\newcommand{\tyUnit}{\tyname{unit}}
\newcommand{\tyNat}{\tyname{nat}}
\newcommand{\tyBool}{\tyname{bool}}

% ILC Computation types
\newcommand{\xF}{\boldsf{F}}
\newcommand{\xU}{\textsf{U}}
\newcommand{\xM}{\textsf{M}}
\newcommand{\tyFp}[1]{\xF(#1)}
\newcommand{\tyUp}[1]{\xU(#1)}
\newcommand{\tyMp}[1]{\xM(#1)}
\newcommand{\tyF}[1]{\xF\,#1}
\newcommand{\tyU}[1]{\xU\,#1}
\newcommand{\tyM}[1]{\xM\,#1}

% ILC Read and Write channel types
\newcommand{\xRd}{\textsf{Rd}}
\newcommand{\xWr}{\textsf{Wr}}
\newcommand{\xRdL}{\textsf{RdL}}
\newcommand{\xWrL}{\textsf{WrL}}
\newcommand{\tyRdp}[1]{\xRd(#1)}
\newcommand{\tyWrp}[1]{\xWr(#1)}
\newcommand{\tyRd}[1]{\xRd\,#1}
\newcommand{\tyWr}[1]{\xWr\,#1}
\newcommand{\tyRdL}[1]{\xRdL\,#1}
\newcommand{\tyWrL}[1]{\xWrL\,#1}

% ILC Modes
\newcommand{\Rm}{\textsf{R}}
\newcommand{\Wm}{\textsf{W}}
\newcommand{\Vm}{\textsf{V}}

%\newcommand{\Split}[4]{\keyword{split}\Lparen{#1}, {#2}.{#3}.{#4}\Rparen}
%\newcommand{\Case}[5]{\keyword{case}\Lparen{#1}, {#2}.{#3}, {#4}.{#5}\Rparen}

\newcommand{\inj}[1]{\keyword{inj}_{#1}\,}
\newcommand{\Inj}[1]{\inj{#1}}

\newcommand{\vPair}[2]{\Lparen{#1},{#2}\Rparen}
\newcommand{\vInj}[2]{\keyword{inj}_{#1}\Lparen#2\Rparen}
\newcommand{\vThunk}[1]{\keyword{thunk}\Lparen#1\Rparen}

\newcommand{\eFork}[2]{\ensuremath{#1 *&& #2}}
\newcommand{\eChoose}[2]{\ensuremath{#1 *|| #2}}


\newcommand{\xFork}{\mathrel{|\rhd}}

%% Math ligatures (thanks to the semantic package) that make it
%% easier to typeset math using readable LaTeX text.
%\mathlig{|-->}{\longmapsto}
\mathlig{::=}{\bnfas}
\mathlig{:=}{\coloneqq}
\mathlig{|}{\;|\;}
% \mathlig{[[}{\mbsf{[}}
% \mathlig{]]}{\mbsf{]}}
\mathlig{[[}{\textsf{\upshape[}}
\mathlig{]]}{\textsf{\upshape]}}
\mathlig{**}{\times}
\mathlig{|>}{\rhd}
\mathlig{->}{\arr}
\mathlig{-->}{\rightarrow}
\mathlig{--->}{\longrightarrow}
\mathlig{=>}{\Rightarrow}
\mathlig{*!}{\boldsf{!}}
\mathlig{||}{\mathrel{|\!|}}
\mathlig{;;}{\mathrel{;}}
\mathlig{*&&}{\mathrel{\xFork}}
\mathlig{*||}{\mathrel{\oplus}}
\mathlig{!!}{\Downarrow}

\mathlig{@>}{\arr_{\namesort}}

\newcommand{\e}{\epsilon}
\newcommand{\hist}{\mathcal{H}}

\newcommand{\ambns}{M}

\newcommand{\disjoint}{\mathrel{\bot}}

\newcommand{\tbrack}[1]{\textsf{\upshape[}{#1}\textsf{\upshape]}}

\newcommand{\rootname}{\textvtt{root}}

%\newcommand{\tv}{tv}
\newcommand{\Mv}{V}
\newcommand{\ntevalsym}{\Downarrow_{\mathsf{M}}}
\newcommand{\nteval}{\mathrel{\ntevalsym}}

\newcommand{\xrefv}{\keyword{ref}}
\newcommand{\xthunk}{\keyword{thunk}}
\newcommand{\xunthunk}{\keyword{unthunk}}
\newcommand{\xname}{\keyword{name}}
%\newcommand{\xName}{\tyname{Name}}
\newcommand{\xName}{\tyname{Nm}}

\newcommand{\xRef}{\tyname{Ref}}
%\newcommand{\xThunk}{\tyname{Thunk}}
\newcommand{\xThunk}{\tyname{Thk}}
\newcommand{\xUnthunk}{\tyname{Unthunk}}

\newcommand{\refv}[1]{\xrefv\;{#1}}
\newcommand{\thunk}[1]{\xthunk\;{#1}}
\newcommand{\unthunk}[1]{\xunthunk\;{#1}}
\newcommand{\name}[1]{\xname\;{#1}}
\newcommand{\Name}[1]{\xName\tbrack{#1}}


\newcommand{\Thk}[1]{\xThunk\tbrack{#1}\,}
\newcommand{\Unthk}[1]{\xUnthunk\tbrack{#1}\,}
\newcommand{\Thunk}[2]{\keyword{thunk}\Lparen{#1},{#2}\Rparen}
\newcommand{\Unthunk}[1]{\keyword{unthunk}\Lparen{#1}\Rparen}
\newcommand{\Ref}[1]{\keyword{ref}\Lparen{#1}\Rparen}
\newcommand{\Get}[1]{\keyword{get}\Lparen{#1}\Rparen}
\newcommand{\Set}[2]{\keyword{set}\Lparen{#1},{#2}\Rparen}
\newcommand{\Seq}[2]{\keyword{seq}\Lparen{#1},{#2}\Rparen}
%\newcommand{\If}[3]{\keyword{if}\Lparen{#1},{#2},{#3}\Rparen}
\newcommand{\Case}[5]{\keyword{case}\Lparen{#1},{#2}.{#3},{#4}.{#5}\Rparen}
\newcommand{\Split}[4]{\keyword{split}\Lparen{#1},{#2}.{#3}.{#4}\Rparen}

\newcommand{\ftv}[1]{{\rm ftv}\Lparen{#1}\Rparen}

\newcommand{\List}[3]{\tyname{List}\tbrack{#1;#2}~{#3}}
\newcommand{\Art}[2]{\tyname{Art}\tbrack{#1}~{#2}}
\newcommand{\Int}[0]{\tyname{Int}}

\let\Force\undefined
\newcommand{\eForce}[1]{\keyword{force}\Lparen{#1}\Rparen}

\newcommand{\eFix}[1]{\keyword{fix}\Lparen{#1}\Rparen}

\newcommand{\App}[2]{#1\,#2}
\newcommand{\eApp}[2]{#1\,#2}

\newcommand{\Lparen}{\textsf{(}}
\newcommand{\Rparen}{\textsf{)}}

\newcommand{\Ret}[1]{\keyword{ret}\Lparen{#1}\Rparen}
\newcommand{\Let}[3]{\keyword{let}\Lparen{#1},{#2}.{#3}\Rparen}
\newcommand{\eLet}[3]{\keyword{let}\Lparen{#1},{#2}.{#3}\Rparen}
\newcommand{\LetBang}[3]{\keyword{let!}\Lparen{#1},{#2}.{#3}\Rparen}

\newcommand{\fix}[2]{\keyword{fix}\Lparen{#1}.{#2}\Rparen}

% Channel read and Channel write expressions
\newcommand{\eRd}[1]{\keyword{rd}\Lparen{#1}\Rparen}
\newcommand{\eWr}[2]{\keyword{wr}\Lparen{#1},{#2}\Rparen}
\newcommand{\eNu}[2]{\nu#1.\,#2}

\newcommand{\xstoretype}{\textsf{has-store-type}}
\newcommand{\storetype}{\;\xstoretype\;}

\newcommand{\xterminal}{\textsf{terminal}}
\newcommand{\terminal}{\;\xterminal}

\newcommand{\Config}[2]{\left<#1;#2\right>}

\newcommand{\Chans}{\Sigma}
%\newcommand{\vChan}[1]{\keyword{chan}\Lparen#1\Rparen}
\newcommand{\vChan}[1]{#1}
\newcommand{\emptyChans}{\varepsilon}

\newcommand{\emptyctxt}{\cdot}

\newcommand{\Procs}{\pi}
\newcommand{\proc}[2]{#1;#2}
\newcommand{\emptyProcs}{\varepsilon}

\newcommand{\msf}[1]{\ensuremath{{\mathsf {#1}}}}
\newcommand{\mtt}[1]{\ensuremath{\mathtt {#1}}}
\newcommand{\mcal}[1]{\ensuremath{\mathcal {#1}}}
\newcommand{\val}{\msf{val}}
\newcommand{\sn}{\msf{sn}}

\newcommand{\tn}{\textnormal}
\newcommand{\codeb}[1]{\textsf{#1}}
\newcommand{\hash}{\ensuremath{\mathcal{H}}}
\newcommand{\adv}{\ensuremath{{\mathcal A}}\xspace}
\newcommand{\Adv}{\adv}
\newcommand{\A}{\adv}
\newcommand{\samples}{\overset{\$}{\leftarrow}}
\newcommand{\SA}{\msf{SA}}
% SaUCy specific
\newcommand{\SaUCy}{\xspace{\msf{SaUCy}}\xspace}
\newcommand{\leak}{\color{blue}\mtt{leak}}
\newcommand{\eventually}[1]{\color{blue}\mtt{leak}}
\newcommand{\poly}{\textnormal{poly}}
\newcommand{\negl}{\textnormal{negl}}
\newcommand{\execUC}[4]{\mathsf{execUC}({#1},{#2},{#3},{#4})}
\newcommand{\chan}[1]{{\underline{\smash{\msf{#1}}}}}
\newcommand{\functionality}[1]{\ensuremath{\mathcal{F}_{\textnormal{\msf{{#1}}}}}}
\newcommand{\F}{\functionality}
%\newcommand{\G}[1]{\mathcal{G}_{\textnormal{\tiny {\uppercase{#1}}}}}
\newcommand{\GG}[1]{{\overline \mathcal{G}}_{\textnormal{\tiny {\uppercase{#1}}}}}
\renewcommand{\C}[1]{\mathcal{C}_{\textnormal{\tiny {\uppercase{#1}}}}}
\renewcommand{\P}{\ensuremath{\mathcal P}}
\newcommand{\Ps}{\ensuremath{\{\mathcal{P}_i\}_{i \in [N]}}}
\renewcommand{\S}{\ensuremath{\mathcal S}}
\newcommand{\env}{\Z}
\newcommand{\Z}{{\mcal{E}}}
\newcommand{\R}{{\mathcal R}}
